\documentclass[10pt]{article}
\usepackage{../pplmanual}
%%% Commonly Needed packages
\usepackage{graphicx,color,calc}
\usepackage{fancyvrb}
\usepackage{makeidx}
\usepackage{alltt}
%\usepackage{html}
\usepackage{listings}
\usepackage{hyperref}
\hypersetup{
    colorlinks,%
    citecolor=black,%
    filecolor=black,%
    linkcolor=black,%
    urlcolor=magenta
}

%%\usepackage{xspace} <- creates problems with other hyperlink packages like "html"

%%% Commands for uniform looks of C++, Charm++, and Projections
\newcommand{\CC}{C\hbox{++}}
\newcommand{\emCC}{C\hbox{\em++}}
\newcommand{\charmpp}{\textsc{Charm++}}
\newcommand{\charmc}{\texttt{charmc}}
\newcommand{\projections}{\textsc{Projections}}
\newcommand{\converse}{\textsc{Converse}}
\newcommand{\ampi}{\textsc{AMPI}}
\newcommand{\tempo}{\textsc{TeMPO}}
\newcommand{\irecv}{\textsl{iRecv}}
\newcommand{\sdag}{\textsl{Structured Dagger}}
\newcommand{\jade}{Jade}

%%% Commands to produce margin symbols
\newcommand{\new}{\marginpar{\fbox{\bf$\mathcal{NEW}$}}}
\newcommand{\important}{\marginpar{\fbox{\bf\Huge !}}}
\newcommand{\experimental}{\marginpar{\fbox{\bf\Huge $\beta$}}}

%%% Commands for manual elements
\newcommand{\zap}[1]{ }
\newcommand{\function}[1]{{\noindent{\textsf{#1}}\\}}
\newcommand{\cmd}[1]{{\noindent{\textsf{#1}}\\}}
\newcommand{\args}[1]{\hspace*{2em}{\texttt{#1}}\\}
\newcommand{\prototype}[1]{\vspace{0.2in}\index{#1}}
\newcommand{\param}[1]{{\texttt{#1}}}
\newcommand{\kw}[1]{{\textsf{#1}\index{#1}}}
\newcommand{\uw}[1]{{\textsl{#1}}}
\newcommand{\desc}[1]{\indent{#1}}
\newcommand{\note}[1]{(\textbf{Note:} #1)}
\newcommand{\term}[1]{{\bf #1}\index{#1}}

\makeindex


\makeindex

\title{\charmpp\\ MultiBlock Framework\\ Manual}
\version{1.0}
\credits{
This version of \charmpp{} MultiBlock Framework was developed
by Orion Lawlor and Milind Bhandarkar.
}

\begin{document}

\maketitle

\section{Motivation}



\section{Introduction/Terminology}



\section{Structure of a MultiBlock Framework Program}

A MultiBlock framework program consists of three subroutines: \kw{init}, \kw{driver}, and \kw{finalize}.  \kw{init} and \kw{finalize} are called by the MultiBlock framework only on the first processor -- these routines typically do specialized I/O, startup and shutdown tasks.  \kw{driver} is called for every chunk on every processor, and does the main work of the program.

\begin{alltt}
     subroutine init
          read the configuration data llike block prefix, 
	  number of blocks and dimension 
     end subroutine

     subroutine driver
	  allocate and initialize the grid
	  register boundary condition functions
          time loop
               MultiBlock computations
               update shared node fields
               more MultiBlock computations
          end time loop
     end subroutine

     subroutine finalize
           write results
     end subroutine
\end{alltt}

\section{Compilation and Execution}

A Multiblock framework program is a \charmpp\ program, so you must begin by
downloading the latest source version of \charmpp\ from
{\tt http://charm.cs.uiuc.edu/}.  Build the source with 
{\tt ./build MBLOCK version} or {\tt cd} into the build directory, 
{\tt version/tmp}, and type {\tt make MBLOCK}.
To compile a MULTIBLOCK program, pass the {\tt -language mblock} (for C) or 
{\tt -language mblockf} (for Fortran) option to {\tt charmc}.

In a charm installation, see charm/version/pgms/charm++/mblock/
for example and test programs.


\section{MultiBlock Framework API Reference}

\subsection{Initialization}
All these methods should be called from the \kw{init} function by the user.The values
passed to these functions are typically read from a file.
\function{int MBLK\_Set\_prefix(const char *prefix);}
\function{subroutine MBLK\_Set\_prefix(prefix,err)}
\args{ Character, intent(in)::prefix}
\args{integer, intent(out)::err}

This function is called to set the prefix. It returns MBLK\_SUCCESS in case of 
success else it returns MBLK\_FAILURE.

\vspace{0.2in}

\function{int MBLK\_Set\_nblocks(const int n);}
\function{ subroutine MBLK\_Set\_nblocks(n,err)}
\args{integer, intent(in)::n}
\args{integer, intent(out)::err}
This call is made to set the number of blocks to be used in the application.It 
returns MBLK\_SUCCESS in case of success else it returns MBLK\_FAILURE.
\vspace{0.2in}

\function{int MBLK\_Set\_dim(const int n);}
\function{subroutine MBLK\_Set\_dim(n, err)}
\args{integer, intent(in)::n}
\args{integer, intent(out):: err}
This call is made to set the dimension. It returns MBLK\_SUCCESS in case of 
success else it returns MBLK\_FAILURE.

\subsection{Utility}

\function{int MBLK\_Get\_nblocks(int* n);}
\function{subroutine MBLK\_Get\_nblocks(n,err)}
\args{	integer,intent(out)::n
	integer,intent(out)::err}
     Get the number of blocks in the current computation.  Can
     only be called from the driver routine. Returns MBLK\_SUCCESS in case of
     success and MBLK\_FAILURE in case of error.
\vspace{0.2in}

\function{int MBLK\_Get\_myblock(int* m);}
\function{subroutine MBLK\_Get\_myblock(m,err)}
\args{	integer,intent(out)::m
	integer,intent(out)::err }

     Get the id of the current block. Can only be called from the driver 
     routine. Returns MBLK\_SUCCESS in case of success and MBLK\_FAILURE in 
     case of error.
\vspace{0.2in}

\function{int MBLK\_Get\_blocksize(int* dims);}
\function{subroutine MBLK\_Get\_blocksize(dimsm,err)}
\args{	integer,intent(out)::dims(3)
	integer,intent(out)::err }
     Get the Interior dimensions, in voxels. The size of the array dims should
     be 3. Can only be called from the driver routine. Returns MBLK\_SUCCESS 
     in case of success and MBLK\_FAILURE in case of error.
\vspace{0.2in}

\function{double MBLK\_Timer(void);}
\function{function double precision :: MBLK\_Timer()}

     Return the current wall clock time, in seconds.  Resolution is
     machine-dependent, but is at worst 10ms.
\vspace{0.2in}

\function{void MBLK\_Print\_block(void);}
\function{subroutine MBLK\_Print\_block()}

     Print a debugging representation of the current block.
     Prints the entire blocks array, and data associated with
     each block.
\vspace{0.2in}

\function{void MBLK\_Print(const char *str);}
\function{subroutine MBLK\_Print(str)}
\args{  character*, intent(in) :: str}

     Print the given string, prepended by the block id if called from the 
     driver. Works on all machines; unlike \kw{printf} or
     \kw{print *}, which may not work on all parallel machines.


\subsection{Block Fields}


The MultiBlock framework handles the updating of the values of blocks.
The basic mechanism to do this update is the field-- numeric data items associated with each block. We make no assumptions about the meaning of the block data, and allow various data types and non-communicated data associated with each
block.  To do this, the framework must be able to find the data items
associated with each block in memory.

Each field represents a (set of) block data items stored in a contiguous array,
indexed by block number.  You create a field once, with \kw{MBLK\_Create\_Field}, then pass the resulting field ID to \kw{MBLK\_Update\_Field} (which does the
overlapping block communication) and/or \kw{MBLK\_Reduce\_Field} (which applies a reduction over block values).
\vspace{0.2in}

\function{int MBLK\_Create\_Field(int *dimensions,int isVoxel,const int base\_type,const int vec\_len,const int offset,const int dist, int *fid);}
\function{subroutine MBLK\_Create\_Field(dimensions, isVoxel,base\_type, vec\_len, offset, dist, err)}
  \args{integer, intent(in)  :: dimensions, isVoxel, base\_type, vec\_len, offset, dist
	integer, intent(out) :: fid, err}

     Creates and returns a MultiBlock field ID, which can be passed to
\kw{MBLK\_Update\_Field} and \kw{MBLK\_Reduce\_Field.}  Can only be called from
\kw{driver().}  A field is a range of values associated with each local block--
the Multiblock framework uses the information you pass to find the values associated with overlapping blocks (for \kw{MBLk\_Update\_Field}) and primary blocks (for \kw{MBLK\_Reduce\_Field}). Returns MBLK\_SUCCESS in case of success and MBLK\_FAILURE in case of error.

     \kw{dimensions} describes the number of dimensions should be in array of size 3
     \kw{isVoxel} describes whether the dimenstions passed in are in voxel(=11) or not(= 0).
     \kw{base\_type} describes the kind of data item associated with each
     node, one of:

     \begin{itemize}
        \item \kw{MBLk\_BYTE}-- unsigned char, INTEGER*1, or CHARACTER*1
        \item \kw{MBLK\_INT}-- int or INTEGER*4
        \item \kw{MBLK\_REAL}-- float or REAL*4
        \item \kw{MBLK\_DOUBLE}-- double, DOUBLE PRECISION, or REAL*8
     \end{itemize}

     \kw{vec\_len} describes the number of data items associated with each
     node, an integer at least 1.

     \kw{offset} is the byte offset from the start of the nodes array to the
     data items, a non-negative integer.

     \kw{dist} is the byte offset from the first node's data item to the
     second, a positive integer.
     \kw{fid} is the identifier for the field that is created by the function.
     \kw{err} is returned as  MBLK\_SUCCESS in case of success otherwise MBLK\_FAILURE is returned.
\vspace{0.2in}

     For example, if each block has a 3D grid, over which we are performing Successive over relaxation.You can register the ghost regions for update with:
\begin{alltt}
	!In Fortran
	
	integer :: size(3), ni,nj,nk
	integer :: si,sj,sk
 	integer :: fid, err
	integer, parameter::ghostwidth=1;	
	!Find the dimensions of the grid, from the framework to allocateit
	MBLk_Get_blocksize(size,err);

	!Add ghost region width to the dimensions obtained from the framework
	ni=size(1)+2*ghostWidth; 
	nj=size(2)+2*ghostWidth; 
	nk=size(3)+2*ghostWidth;
 	si=1+ghostWidth; sj=1+ghostWidth; sk=1+ghostWidth;

	...allocate and initialize the grid 

	!Create the field that needs to be updated
	size(1)=ni; size(2)=nj; size(3)=nk;
	call MBLK_Create_field(&
	       &size,1, MBLK_DOUBLE,1,&
	       &offsetof(grid(1,1,1),grid(si,sj,sk)),&
	       &offsetof(grid(1,1,1),grid(2,1,1)),fid,err)	
	

\end{alltt}
     This example uses the Fortran-only helper routine \kw{offsetof}, which
     returns the offset in bytes of memory between its two given
     variables.  The C version uses pointer arithmetic to achieve the
     same result.
\vspace{0.2in}
     

\function{void MBLK\_Update\_field(const int fid,int ghostwidth, void *grid);}
\function{subroutine MBLK\_Update\_field(fid,ghostwidth, grid,err)}
  \args{integer, intent(in)  :: fid, ghostwidth}
  \args{integer,intent(out) :: err}
  \args{varies, intent(inout) :: grid}

     Update the values in the ghost regions which specified when the
     field was created. For the example above the ghost regions will be 
     updated once for each step in the time loop.

     \kw{MBLK\_Update\_field} can only be called from driver, and to be useful,
     must be called from every block's driver routine.

     \kw{MBLK\_Update\_field} blocks till the field has been updated.
     After this routine returns, the given field will updated.
     If the update was successful MBLK\_SUCCESS is returned and 
     MBLk\_FAILURE is returned in case of error.
\vspace{0.2in}

\function{void MBLK\_Iupdate\_field(const int fid,int ghostwidth, void *ingrid, void* outgrid);}
\function{subroutine MBLK\_Iupdate\_field(fid,ghostwidth, ingrid, outgrid,err)}
  \args{integer, intent(in)  :: fid, ghostwidth}
  \args{integer,intent(out) :: err}
  \args{varies,intent(in) :: ingrid}
  \args{varies,intent(out) :: outgrid}


     Update the values in the ghost regions which were specified when the
     field was created. For the example above the ghost regions will be 
     updated once for each step in the time loop.

     \kw{MBLK\_Iupdate\_field} can only be called from driver, and to be useful,
     must be called from every block's driver routine.

     \kw{MBLK\_Iupdate\_field} is a non blocking call like MPI\_IRecv
     After the routine returns the update is not complete but is guranteed
     to be complete in future. So before using the values the status of the
     update should be checked using \kw{MBLk\_Test\_update} or should wait
     for the completion of the call using \kw{MBLK\_Wait\_update}
     
     If the\kw{MBLK\_Iupdate\_field} call was successful MBLK\_SUCCESS is 
     returned and MBLK\_FAILURE is returned in case of error.
\vspace{0.2in}

\function{int MBLK\_Test\_update(int *status);}
\function{subroutine MBLK\_Test\_update(status,err)}
\args{integer, intent(out) :: status,err}

     \kw{MBLK\_Test\_update} is a call that is used in assosiation with 
     \kw{MBLk\_Iupdate\_field} from the driver sub routine.It tests whether
      the field has been updated or not.
     \kw{status} is returned as MBLK\_DONE if the update was completed or 
      MBLK\_NOTDONE if the update is still pending.
     \kw{err} is returned as MBLK\_SUCCESS if the call was successful or 
     MBLK\_FAILURE if there was an error.


ORION: I think this function at this time is not updating the status I think it is returning MBLK\_DONE or MBLK\_NOTDONE or MBLK\_FAILURE in error instead of using the 
status. Please have a look at that
\vspace{0.2in}

\function{void MBLk\_Wait\_update(void);}
\function{subroutine MBLk\_Wait\_update()}

     \kw{MBLk\_Wait\_update} call is a blocking call and is used in assoisation 
     with \kw{MBLK\_Iupdate\_field} call. It blocks till the update is completed.
\vspace{0.2 in}


\function{void MBLK\_Reduce\_field(int fid,void *grid, void *out,int op);}
\function{subroutine MBLK\_Reduce\_field(fid,grid,outVal,op)}
  \args{integer, intent(in)  :: fid,op}
  \args{varies, intent(in) :: grid}
  \args{varies, intent(out) :: outVal}

     Combine a field from each block, according to op, across all blocks.
     After \kw{Reduce\_Field} returns, all blocks will have identical
values in \kw{outVal,} which must be \kw{vec\_len} copies of \kw{base\_type}.

     May only be called from driver, and to complete, must be called
     from every chunk's driver routine.

     \kw{op} must be one of:

\begin{itemize}
        \item \kw{MBLk\_SUM}-- each element of \kw{outVal} will be the sum 
of the corresponding fields of all blocks
        \item \kw{MBLK\_MIN}-- each element of \kw{outVal} will be the 
smallest value among the corresponding field of all blocks
        \item \kw{MBLK\_MAX}-- each element of \kw{outVal} will be the largest 
value among the corresponding field of all blocks
\end{itemize}
\vspace{0.2in}

\function{void MBLK\_Reduce(int fid,void *inVal,void *outVal,int op);}
\function{subroutine MBLK\_Reduce(fid,inVal,outVal,op)}
  \args{integer, intent(in)  :: fid,op}
  \args{varies, intent(in) :: inVal}
  \args{varies, intent(out) :: outVal}

     Combine a field from each block, acoording to \kw{op}, across all blocks.
\kw{Fid} is only used for the \kw{base\_type} and \kw{vec\_len}-- offset and
\kw{dist} are not used.  After this call returns, all blocks will have
identical values in \kw{outVal}.  Op has the same values and meaning as
\kw{MBLK\_Reduce\_Field}.

     May only be called from driver, and to complete, must be called
     from every blocks driver routine.

\vspace{0.2in}

\subsection{Boundary Conditions}
In most of the applications using the MultiBlock Framework the blocks have 
boundary conditions for eaach block depending on the geometery. Various calls 
are provided in the framework to register and apply the boundry conditions.

\function{int MBLK\_Register\_bc(const int bcnum, int ghostWidth, const MBLK\_BcFn bcfn);}
\function{subroutine MBLK\_Register\_bc(bcnum, ghostwidth, bcfn, err)}
\args{integer,intent(in) :: bcnum, ghostWidth}
\args{integer,intent(out) :: err}
\args{subroutine :: bcfn}

This is call is used to register the boundry condition function, for a block,
with the framework.
\begin{itemize}
	\item \kw{bcnum} The boundry condtion number to be associated with the 
	function.
	\item \kw{ghostWidth} The width of the ghostcells where this boundry
	condition is going to be applied.
	\item \kw{bcfn} The user function that is to be used for applying the
	the boundry conditions.
\end{itemize}
\kw{MBLK\_Register\_bc} should only be called from the driver.
This call returns MBLK\_SUCCESS in case of success or else returns MBLK\_FAILURE
in case of an error.
\vspace{0.2in}

\function{ int MBLK\_Apply\_bc(const int bcnum, void *grid,void *size);}
\function{subroutine MBLK\_Apply\_bc(bcnum, grid, size,err)}
\args{ integer,intent(in)::bcnum}
\args{integer,intent(out)::err}
\args{varies,intent(inout)::grid}
\args{varies, intent(in)::size}

\kw{MBLK\_Apply\_bc} call is made to apply the boundry condition function
associated to \kw{bcnum} to the block.The grid specifies the place where 
the boundary condition are to be applied adn sizes array gives the dimensions
of the grid.
It returns MBLK\_SUCCESS if the call is successful else it returns
MBLK\_FAILURE in case of error. 

\function{ int MBLK\_Apply\_bc\_all(void* grid, void* size);} 
\function{subroutine MBLK\_Apply\_bc\_all(grid, size, err)}
\args{integer,intent(out)::err}
\args{varies,intent(inout)::grid}
\args{varies, intent(in)::size}
This call is same as \kw{MBLK\_Apply\_bc} except it applies all the boundary 
functions to the block.

\subsection{Migration}

The \charmpp\ runtime framework includes an automated, run-time load balancer,
which will automatically monitor the performance of your parallel program.
If needed, the load balancer can ``migrate'' mesh chunks from heavily-loaded
processors to more lightly-loaded processors, improving the load balance and
speeding up the program.  For this to be useful, pass the \kw{+vpN} argument
with a larger number of blocks \kw{N} than processors
Because this is somewhat involved, you may refrain from calling 
\kw{MBLK\_Migrate} and migration will never take place.

The runtime system can automatically move your thread stack to the new
processor, but you must write a PUP function to move any global or
heap-allocated data to the new processor (global data is declared at file scope
or \kw{static} in C and \kw{COMMON} in Fortran77; heap allocated data comes
from C \kw{malloc}, C++ \kw{new}, or Fortran90 \kw{ALLOCATE}).  A PUP
(Pack/UnPack) function performs both packing (converting heap data into a
message) and unpacking (converting a message back into heap data).  All your
global and heap data must be collected into a single block (\kw{struct} in C;
user-defined \kw{TYPE} in Fortran) so the PUP function can access it all.

Your PUP function will be passed a pointer to your heap data block and a
special handle called a ``pupper'', which contains the network message to be
sent.  Your PUP function returns a pointer to your heap data block.  In a PUP
function, you pass all your heap data to routines named \kw{pup\_type}, where
type is either a basic type (such as int, char, float, or double) or an array
type (as before, but with a ``s'' suffix).  Depending on the direction of
packing, the pupper will either read from or write to the values you pass--
normally, you shouldn't even know which.  The only time you need to know the
direction is when you are leaving a processor or just arriving.
Correspondingly, the pupper passed to you may be deleting (indicating that you
are leaving the processor, and should delete your heap storage after packing),
unpacking (indicating you've just arrived on a processor, and should allocate
your heap storage before unpacking), or neither (indicating the system is
merely sizing a buffer, or checkpointing your values).

PUP functions are much easier to write than explain-- a simple C heap block
and the corresponding PUP function is:

\begin{alltt}
     typedef struct {
       int n1;/*Length of first array below*/
       int n2;/*Length of second array below*/
       double *arr1; /*Some doubles, allocated on the heap*/
       int *arr2; /*Some ints, allocated on the heap*/
     } my_block;
 
     my_block *pup_my_block(pup_er p,my_block *m)
     {
       if (pup_isUnpacking(p)) m=malloc(sizeof(my_block));
       pup_int(p,\&m->n1);
       pup_int(p,\&m->n2);
       if (pup_isUnpacking(p)) {
         m->arr1=malloc(m->n1*sizeof(double));
         m->arr2=malloc(m->n2*sizeof(int));
       }
       pup_doubles(p,m->arr1,m->n1);
       pup_ints(p,m->arr2,m->n2);
       if (pup_isDeleting(p)) {
         free(m->arr1);
         free(m->arr2);
         free(m);
       }
       return m;
     }
\end{alltt}

This single PUP function can be used to copy the \kw{my\_block} data into a
message buffer and free the old heap storage (deleting pupper); allocate
storage on the new processor and copy the message data back (unpacking pupper);
or save the heap data for debugging or checkpointing.

A Fortran block TYPE and corresponding PUP routine is as follows:

\begin{alltt}
     MODULE my_block_mod
       TYPE my_block
         INTEGER :: n1,n2x,n2y
         REAL*8, POINTER, DIMENSION(:) :: arr1
         INTEGER, POINTER, DIMENSION(:,:) :: arr2
       END TYPE
     END MODULE
 
     SUBROUTINE pup_my_block(p,m)
       IMPLICIT NONE
       USE my_block_mod
       USE pupmod
       INTEGER :: p
       TYPE(my_block) :: m
       call pup_int(p,m%n1)
       call pup_int(p,m%n2x)
       call pup_int(p,m%n2y)
       IF (pup_isUnpacking(p)) THEN
         ALLOCATE(m%arr1(m%n1))
         ALLOCATE(m%arr2(m%n2x,m%n2y))
       END IF
       call pup_doubles(p,m%arr1,m%n1)
       call pup_ints(p,m%arr2,m%n2x*m%n2y)
       IF (pup_isDeleting(p)) THEN
         DEALLOCATE(m%arr1)
         DEALLOCATE(m%arr2)
       END IF
     END SUBROUTINE
\end{alltt}

\function{int MBLK\_Register(void *block, MBLk\_PupFn pup\_ud, int* rid)}
\function{subroutine MBLK\_Register(block,pup\_ud, rid)}
    \args{integer, intent(out)::rid}
    \args{TYPE(varies), POINTER :: block}
    \args{SUBROUTINE :: pup\_ud}

     Associates the given data block and PUP function.  Returns a block
     ID, which can be passed to \kw{MBLK\_Get\_registered} later.  Can only be
     called from driver.  It returns MBLK\_SUCESS if the call was successful
     and MBLK\_FAILURE in case of error. For the declarations above, you call
     \kw{MBLK\_Register} as:

\begin{alltt}
          /*C/C++ driver() function*/
	  int myId, err;
          my_block *m=malloc(sizeof(my_block));
          err =MBLK_Register(m,(MBLK_PupFn)pup_my_block,&rid);
 
          !- Fortran driver subroutine
          use my_block_mod
          interface
            subroutine pup_my_block(p,m)
              use my_block_mod
              INTEGER :: p
              TYPE(my_block) :: m
            end subroutine
          end interface
          TYPE(my_block) :: m
          INTEGER :: myId,err
          MBLK_Register(m,pup_my_block,myId,err)
\end{alltt}

     Note that Fortran blocks must be allocated on the stack in driver;
     while C/C++ blocks may be allocated on the heap.
\vspace{0.2in}

\function{void MBLK\_Migrate()}
\function{subroutine MBLK\_Migrate()}

     Informs the load balancing system that you are ready to be
     migrated, if needed.  If the system decides to migrate you, the
     PUP function passed to \kw{MBLK\_Register} will be called with a sizing
     pupper, then a packing, deleting pupper.  Your stack (and pupped
     data) will then be sent to the destination machine, where your PUP
     function will be called with an unpacking pupper.  \kw{MBLK\_Migrate}
     will then return, whereupon you should call \kw{MBLK\_Get\_registered} to
     get your unpacked data block.  Can only be called from driver.



\function{int MBLK\_Get\_Userdata(int n, void** block)}

     Return your unpacked userdata after migration-- that is, the
     return value of the unpacking call to your PUP function.  Takes
     the userdata ID returned by \kw{MBLK\_Register}.  Can be called from
     driver at any time.

     Since Fortran blocks are always allocated on the stack, the system
     migrates them to the same location on the new processor, so no
     \kw{Get\_registered} call is needed from Fortran.
\input{index}
\end{document}
