\documentclass[10pt]{article}
\usepackage{../pplmanual}
%%% Commonly Needed packages
\usepackage{graphicx,color,calc}
\usepackage{fancyvrb}
\usepackage{makeidx}
\usepackage{alltt}
%\usepackage{html}
\usepackage{listings}
\usepackage{hyperref}
\hypersetup{
    colorlinks,%
    citecolor=black,%
    filecolor=black,%
    linkcolor=black,%
    urlcolor=magenta
}

%%\usepackage{xspace} <- creates problems with other hyperlink packages like "html"

%%% Commands for uniform looks of C++, Charm++, and Projections
\newcommand{\CC}{C\hbox{++}}
\newcommand{\emCC}{C\hbox{\em++}}
\newcommand{\charmpp}{\textsc{Charm++}}
\newcommand{\charmc}{\texttt{charmc}}
\newcommand{\projections}{\textsc{Projections}}
\newcommand{\converse}{\textsc{Converse}}
\newcommand{\ampi}{\textsc{AMPI}}
\newcommand{\tempo}{\textsc{TeMPO}}
\newcommand{\irecv}{\textsl{iRecv}}
\newcommand{\sdag}{\textsl{Structured Dagger}}
\newcommand{\jade}{Jade}

%%% Commands to produce margin symbols
\newcommand{\new}{\marginpar{\fbox{\bf$\mathcal{NEW}$}}}
\newcommand{\important}{\marginpar{\fbox{\bf\Huge !}}}
\newcommand{\experimental}{\marginpar{\fbox{\bf\Huge $\beta$}}}

%%% Commands for manual elements
\newcommand{\zap}[1]{ }
\newcommand{\function}[1]{{\noindent{\textsf{#1}}\\}}
\newcommand{\cmd}[1]{{\noindent{\textsf{#1}}\\}}
\newcommand{\args}[1]{\hspace*{2em}{\texttt{#1}}\\}
\newcommand{\prototype}[1]{\vspace{0.2in}\index{#1}}
\newcommand{\param}[1]{{\texttt{#1}}}
\newcommand{\kw}[1]{{\textsf{#1}\index{#1}}}
\newcommand{\uw}[1]{{\textsl{#1}}}
\newcommand{\desc}[1]{\indent{#1}}
\newcommand{\note}[1]{(\textbf{Note:} #1)}
\newcommand{\term}[1]{{\bf #1}\index{#1}}

\makeindex


\newcommand{\tcharm}{\textsc{TCharm}}

\makeindex

\title{Threaded \charmpp Manual}
\version{1.0}
\credits{
The initial version of Threaded \charmpp{} was developed
in late 2001 by Orion Lawlor.
}

\begin{document}

\maketitle

\section{Motivation}

\charmpp{} includes several application frameworks, such as the 
Finite Element Framework, the Multiblock Framework, and AMPI.  
These frameworks do almost all their work in load balanced, 
migratable threads.  

The Threaded \charmpp{} Framework, \tcharm{}, provides both
common runtime support for these threads and facilities for
combining multiple frameworks within a single program.
For example, you can use \tcharm{} to create a Finite Element
Framework application that also uses AMPI to communicate between
Finite Element chunks.

Specifically, \tcharm{} provides language-neutral interfaces for:
\begin{enumerate}
\item{Program startup, including read-only global data setup and the
configuration of multiple frameworks.}
\item{Run-time load balancing, including migration.}
\item{Program shutdown.}
\end{enumerate}

The first portion of this manual describes the general properties 
of \tcharm{} common to all the application frameworks, such as 
program contexts and how to write migratable code.
The second portion describes in detail how to combine separate 
frameworks into a single application.


\section{Basic \tcharm{} Programming}
Any routine in a \tcharm{} program runs in one of two contexts:

\begin{description}

\item[Serial Context] Routines that run on only one processor
and with only one set of data. There are absolutely
no limitations on what a serial context routine can do---it 
is as if the code were running in an ordinary serial program.
Startup and shutdown routines usually run in the serial context.

\item[Parallel Context] Routines that run on several processors,
and may run with several different sets of data on a single processor.
This kind of routine must obey certain restrictions.  The program's
main computation routines always run in the parallel context.

\end{description}

Parallel context routines run in a migratable, 
user-level thread maintained by \tcharm{}.  
Since there are normally several of these threads per processor,
any code that runs in the parallel context has to be thread-safe.
However, \tcharm{} 
is non-preemptive, so it will only switch threads when you make
a blocking call, like ``MPI\_Recv" or ``FEM\_Update\_field".



\subsection{Global Variables}
\label{sec:global}
By ``global variables'', we mean anything that is stored at a fixed, 
preallocated location in memory.  In C, this means variables declared 
at file scope or with the \kw{static} keyword.  In Fortran, this is
% variables that are part of a \kw{COMMON} block or declared inside 
% a \kw{MODULE}.
either variables that are part of a \kw{COMMON} block, declared inside 
a \kw{MODULE} or variables with the \kw{SAVE} attribute.

Global variables are shared by all the threads on a processor, which
makes using global variables extremely error prone.
To see why this is a problem, consider a program fragment like:

\begin{alltt}
  foo=a
  call MPI_Recv(...)
  b=foo
\end{alltt}

After this code executes, we might expect \uw{b} to always be equal to \uw{a}.
but if \uw{foo} is a global variable, \kw{MPI\_Recv} may block and 
\uw{foo} could be changed by another thread.

For example, if two threads execute this program, they could interleave like:

\vspace{0.1in}
\begin{tabular}{|l|l|}\hline
\em{Thread 1} & \em{Thread 2}\\
\hline
\uw{foo}=1 & \\
block in MPI\_Recv & \\
 & \uw{foo}=2 \\
 & block in MPI\_Recv \\
\uw{b}=\uw{foo} & \\
\hline\end{tabular}
\vspace{0.1in}

At this point, thread 1 might expect \uw{b} to be 1; but it will actually be 2.
From the point of view of thread 1, the global variable \uw{foo} suddenly
changed its value during the call to \kw{MPI\_Recv}.

There are several possible solutions to this problem:

\begin{itemize}
\item Never use global variables---only use parameters or local variables.  
This is the safest and most general solution.
One standard practice is to collect all the globals into a C struct or 
Fortran type named ``Globals", and pass a pointer to this object to all
your subroutines.  This also combines well with the pup method for doing
migration-based load balancing, as described in Section~\ref{sec:pup}.

\item Never write {\em different} values to global variables.  If every thread
writes the same value, global variables can be used safely.  For example,
you might store some parameters read from a configuration file like the 
simulation timestep $\Delta t$.  See Section~\ref{sec:readonlyglobal}
for another, more convenient way to set such variables.

\item Never issue a blocking call while your global variables are set.
This will not work on a SMP version of Charm++, where several processors
may share a single set of global variables.
Even on a non-SMP version, this is a dangerous solution, because someday 
someone might add a blocking call while the variables are set.  
This is only a reasonable solution when calling legacy code or 
using old serial libraries that might use global variables.
\end{itemize}

The above only applies to routines that run in the parallel context.
There are no restrictions on global variables for serial context
code.



\subsection{Input/Output}
\label{sec:io}

In the parallel context, there are several limitations on open
files.  First, several threads may run on one processor, so
Fortran Logical Unit Numbers are shared by all the threads on
a processor.  Second, open files are left behind when a thread 
migrates to another processor---it is a crashing error to open a 
file, migrate, then try to read from the file.

Because of these restrictions, it is best to open files only when
needed, and close them as soon as possible.  In particular, it
is best if there are no open files whenever you make blocking calls.


\subsection{Migration-Based Load Balancing}
\label{sec:migration}
\label{sec:isomalloc}

The \charmpp\ runtime framework includes an automatic run-time load balancer,
which can monitor the performance of your parallel program.
If needed, the load balancer can ``migrate'' threads from heavily-loaded
processors to more lightly-loaded processors, improving the load balance and
speeding up the program.  For this to be useful, you need to pass the 
link-time argument \kw{-balancer} \uw{B} to set the load balancing algorithm,
and the run-time argument \kw{+vp} \uw{N} (use \uw{N} virtual processors)
to set the number of threads.
The ideal number of threads per processor depends on the problem, but
we've found five to a hundred threads per processor to be a useful range.

When a thread migrates, all its data must be brought with it.
``Stack data'', such as variables declared locally in a subroutine,
will be brought along with the thread automatically.  Global data,
as described in Section~\ref{sec:global}, is never brought with the thread
and should generally be avoided.

``Heap data'' in C is structures and arrays allocated using \kw{malloc} or \kw{new};
in Fortran, heap data is TYPEs or arrays allocated using \kw{ALLOCATE}.
To bring heap data along with a migrating thread, you have two choices:
write a pup routine or use isomalloc.  Pup routines are described in 
Section~\ref{sec:pup}.

{\em Isomalloc} is a special mode which controls the allocation of heap data.  
You enable isomalloc allocation using the link-time flag ``-memory isomalloc''.  
With isomalloc, migration is completely transparent---all your allocated data 
is automatically brought to the new processor.  The data will be unpacked at the same
location (the same virtual addresses) as it was stored originally; so even
cross-linked data structures that contain pointers still work properly.

The limitations of isomalloc are:
\begin{itemize}
\item Wasted memory.  Isomalloc uses a special interface\footnote{
The interface used is \kw{mmap}.} to aquire memory, and the finest granularity
that can be aquired is one page, typically 4KB.  This means if you allocate
a 2-entry array, isomalloc will waste an entire 4KB page.  We should eventually 
be able to reduce this overhead for small allocations.

\item Limited space on 32-bit machines.  Machines where pointers are 32 bits
long can address just 4GB ($2^32$ bytes) of virtual address space.  Additionally, 
the operating system and conventional heap already use a significant amount 
of this space; so the total virtual address space available is typically under 1GB.  
With isomalloc, all processors share this space, so with just 20 processors
the amount of memory per processor is limited to under 50MB!  This is an 
inherent limitation of 32-bit machines; to run on more than a few processors you 
must use 64-bit machines or avoid isomalloc.
\end{itemize}



\section{Advanced \tcharm{} Programming}
The preceeding features are enough to write simple programs
that use \tcharm{}-based frameworks.  These more advanced techniques
provide the user with additional capabilities or flexibility.


\subsection{Writing a Pup Routine}
\label{sec:pup}

The runtime system can automatically move your thread stack to the new
processor, but unless you use isomalloc, you must write a pup routine to 
move any global or heap-allocated data to the new processor.  A pup
(Pack/UnPack) routine can perform both packing (converting your data into a
network message) and unpacking (converting the message back into your data).  
A pup routine is passed a pointer to your data block and a
special handle called a ``pupper'', which contains the network message.  

In a pup routine, you pass all your heap data to routines named \kw{pup\_}\uw{type} or \kw{fpup\_}\uw{type}, where
\uw{type} is either a basic type (such as int, char, float, or double) or an array
type (as before, but with a ``s'' suffix).  Depending on the direction of
packing, the pupper will either read from or write to the values you pass--
normally, you shouldn't even know which.  The only time you need to know the
direction is when you are leaving a processor, or just arriving.
Correspondingly, the pupper passed to you may be deleting (indicating that you
are leaving the processor, and should delete your heap storage after packing),
unpacking (indicating you've just arrived on a processor, and should allocate
your heap storage before unpacking), or neither (indicating the system is
merely sizing a buffer, or checkpointing your values).

pup functions are much easier to write than explain-- a simple C heap block
and the corresponding pup function is:

\begin{alltt}
     typedef struct \{
       int n1;/*Length of first array below*/
       int n2;/*Length of second array below*/
       double *arr1; /*Some doubles, allocated on the heap*/
       int *arr2; /*Some ints, allocated on the heap*/
     \} my_block;
 
     void pup_my_block(pup_er p,my_block *m)
     \{
       if (pup_isUnpacking(p)) \{ /*Arriving on new processor*/
         m->arr1=malloc(m->n1*sizeof(double));
         m->arr2=malloc(m->n2*sizeof(int));
       \}
       pup_doubles(p,m->arr1,m->n1);
       pup_ints(p,m->arr2,m->n2);
       if (pup_isDeleting(p)) \{ /*Leaving old processor*/
         free(m->arr1);
         free(m->arr2);
       \}
     \}
\end{alltt}

This single pup function can be used to copy the \kw{my\_block} data into a
message buffer and free the old heap storage (deleting pupper); allocate
storage on the new processor and copy the message data back (unpacking pupper);
or save the heap data for debugging or checkpointing.

A Fortran block TYPE and corresponding pup routine is as follows:

\begin{alltt}
     MODULE my_block_mod
       TYPE my_block
         INTEGER :: n1,n2x,n2y
         DOUBLE PRECISION, ALLOCATABLE, DIMENSION(:) :: arr1
         INTEGER, ALLOCATABLE, DIMENSION(:,:) :: arr2
       END TYPE
     END MODULE
 
     SUBROUTINE pup_my_block(p,m)
       IMPLICIT NONE
       USE my_block_mod
       USE pupmod
       INTEGER :: p
       TYPE(my_block) :: m
       IF (fpup_isUnpacking(p)) THEN
         ALLOCATE(m%arr1(m%n1))
         ALLOCATE(m%arr2(m%n2x,m%n2y))
       END IF
       call fpup_doubles(p,m%arr1,m%n1)
       call fpup_ints(p,m%arr2,m%n2x*m%n2y)
       IF (fpup_isDeleting(p)) THEN
         DEALLOCATE(m%arr1)
         DEALLOCATE(m%arr2)
       END IF
     END SUBROUTINE
\end{alltt}


You indicate to \tcharm{} that you want a pup routine called using
the routine below.  An arbitrary number of blocks can be registered
in this fashion.

\vspace{0.2in}
\function{void TCHARM\_Register(void *block, TCharmPupFn pup\_fn)}
\function{SUBROUTINE TCHARM\_Register(block,pup\_fn)}
    \args{TYPE(varies), POINTER :: block}
    \args{SUBROUTINE :: pup\_fn}

     Associate the given data block and pup function.  Can only be
     called from the parallel context.  For the declarations above, you call
     \kw{TCHARM\_Register} as:

\begin{alltt}
          /*In C/C++ driver() function*/
          my_block m;
          TCHARM_Register(m,(TCharmPupFn)pup_my_block);
 
          !- In Fortran driver subroutine
          use my_block_mod
          interface
            subroutine pup_my_block(p,m)
              use my_block_mod
              INTEGER :: p
              TYPE(my_block) :: m
            end subroutine
          end interface
          TYPE(my_block), TARGET :: m
          call TCHARM_Register(m,pup_my_block)
\end{alltt}

     Note that the data block must be allocated on the stack. 
     Also, in Fortran, the "TARGET" attribute must be used on the 
     block (as above) or else the compiler may not update values during 
     a migration, because it believes only it can access the block.

\vspace{0.2in}
\function{void TCHARM\_Migrate()}
\function{subroutine TCHARM\_Migrate()}

     Informs the load balancing system that you are ready to be
     migrated, if needed.  If the system decides to migrate you, the
     pup function passed to \kw{TCHARM\_Register} will first be called with 
     a sizing pupper, then a packing, deleting pupper.  Your stack and pupped
     data will then be sent to the destination machine, where your pup
     function will be called with an unpacking pupper.  \kw{TCHARM\_Migrate}
     will then return.  Can only be called from in the parallel context.



\subsection{Readonly Global Variables}
\label{sec:readonlyglobal}

You can also use a pup routine to set up initial values for global
variables on all processors.  This pup routine is called with only
a pup handle, just after the serial setup routine, and just before 
any parallel context routines start.  The pup routine is never
called with a deleting pup handle, so you need not handle that case.

A C example is:
\begin{alltt}
     int g_arr[17];
     double g_f;
     int g_n; /*Length of array below*/
     float *g_allocated; /*heap-allocated array*/
 
     void pup_my_globals(pup_er p)
     \{
       pup_ints(p,g_arr,17);
       pup_double(p,&g_f);
       pup_int(p,&g_n);
       if (pup_isUnpacking(p)) \{ /*Arriving on new processor*/
         g_allocated=malloc(g_n*sizeof(float));
       \}
       pup_floats(p,g_allocated,g_n);
     \}
\end{alltt}

A fortran example is:
\begin{alltt}
     MODULE my_globals_mod
       INTEGER :: g_arr(17)
       DOUBLE PRECISION :: g_f
       INTEGER :: g_n
       SINGLE PRECISION, ALLOCATABLE :: g_allocated(:)
     END MODULE
 
     SUBROUTINE pup_my_globals(p)
       IMPLICIT NONE
       USE my_globals_mod
       USE pupmod
       INTEGER :: p
       call fpup_ints(p,g_arr,17)
       call fpup_double(p,g_f)
       call fpup_int(p,g_n)
       IF (fpup_isUnpacking(p)) THEN
         ALLOCATE(g_allocated(g_n))
       END IF
       call fpup_floats(p,g_allocated,g_n)
     END SUBROUTINE
\end{alltt}


You register your global variable pup routine using the method below.
Multiple pup routines can be registered the same way.

\vspace{0.2in}
\function{void TCHARM\_Readonly\_globals(TCharmPupGlobalFn pup\_fn)}
\function{SUBROUTINE TCHARM\_Readonly\_globals(pup\_fn)}
    \args{SUBROUTINE :: pup\_fn}



\section{Combining Frameworks}
\label{sec:combining}

This section describes how to combine multiple frameworks in a 
single application.  You might want to do this, for example,
to use AMPI communication inside a finite element method solver.

You specify how you want the frameworks to be combined by writing
a special setup routine that runs when the program starts.
The setup routine must be named TCHARM\_User\_setup.  If you declare a 
user setup routine, the standard framework setup routines (such
as the FEM framework's \kw{init} routine) are bypassed, and you
do all the setup in the user setup routine.

The setup routine creates a set of threads and then attaches frameworks
to the threads.  Several different frameworks can be attached to one thread set,
and there can be several sets of threads; however, the most frameworks
cannot be attached more than once to single set of threads. That is, a single
thread cannot have two attached AMPI frameworks, since the MPI\_COMM\_WORLD 
for such a thread would be indeterminate.

\vspace{0.2in}
\function{void TCHARM\_Create(int nThreads, TCharmThreadStartFn thread\_fn)}
\function{SUBROUTINE TCHARM\_Create(nThreads,thread\_fn)}
    \args{INTEGER, INTENT(in) :: nThreads}
    \args{SUBROUTINE :: thread\_fn}

Create a new set of \tcharm{} threads of the given size.  The threads will
execute the given function, which is normally your user code.  
You should call \kw{TCHARM\_Get\_num\_chunks()} 
to get the number of threads from the command line.  This routine can 
only be called from your \kw{TCHARM\_User\_setup} routine.

You then attach frameworks to the new threads.  The order in which
frameworks are attached is irrelevant, but attach commands always apply
to the current set of threads.
  
To attach a chare array to the \tcharm{} array, use:

\function{CkArrayOptions TCHARM\_Attach\_start(CkArrayID *retTCharmArray,int *retNumElts)}
This function returns a CkArrayOptions object that will bind your chare array to the \tcharm{} array, in addition to returning the \tcharm{} array proxy and number of elements by reference. If you are using frameworks like AMPI, they will automatically attach themselves to the \tcharm{} array in their initialization routines.

\section{Command-line Options}
\label{sec:cla}

The complete set of link-time arguments relevant to \tcharm{} is:
\begin{description}
\item[-memory isomalloc] Enable memory allocation that will automatically
migrate with the thread, as described in Section~\ref{sec:isomalloc}.

\item[-balancer \uw{B}] Enable this load balancing strategy.  The
current set of balancers \uw{B} includes RefineLB (make only small changes
each time), MetisLB (remap threads using graph partitioning library), 
HeapCentLB (remap threads using a greedy algorithm), and RandCentLB
(remap threads to random processors).  You can only have one balancer.

\item[-module \uw{F}] Link in this framework.  The current set of frameworks
\uw{F} includes ampi, collide, fem, mblock, and netfem.  You can link in 
multiple frameworks.

\end{description}

The complete set of command-line arguments relevant to \tcharm{} is:

\begin{description}
\item[+p \uw{N}] Run on \uw{N} physical processors.

\item[+vp \uw{N}] Create \uw{N} ``virtual processors'', or threads.  This is
the value returned by TCharmGetNumChunks.

\item[++debug] Start each program in a debugger window.  See Charm++
Installation and Usage Manual for details.

\item[+tcharm\_stacksize \uw{N}] Create \uw{N}-byte thread stacks.  This
value can be overridden using TCharmSetStackSize().

\item[+tcharm\_nomig] Disable thread migration.  This can help determine
whether a problem you encounter is caused by our migration framework.

\item[+tcharm\_nothread] Disable threads entirely.  This can help determine
whether a problem you encounter is caused by our threading framework.
This generally only works properly when using only one thread.

\item[+tcharm\_trace \uw{F}] Trace all calls made to the framework \uw{F}.
This can help to understand a complex program.  This feature is not
available if \charmpp{} was compiled with CMK\_OPTIMIZE.

\end{description}

%%%%%%%%%%%%%%%%%%%%%%%%%%%%%%%%%%%%%%%%%%%%%%%%%%%%%%%%%%%%%
\section{Writing a library using TCharm}
\label{sec:tlib}

Until now, things were presented from the perspective of a user---one 
who writes a program for a library written on TCharm. This section gives an
overview of how to go about writing a library in Charm++ that uses TCharm.

\begin{itemize}
\item Compared to using plain MPI, TCharm provides the ability to 
access all of Charm++, including arrays and groups.

\item Compared to using plain Charm++,
using TCharm with your library automatically provides your users with a 
clean C/F90 API (described in the preceeding chapters) for basic thread
memory  managment, I/O, and migration.  It also allows you to use a convenient
"thread->suspend()" and "thread->resume()" API for blocking a thread,
and works properly with the load balancer, unlike CthSuspend/CthAwaken.
\end{itemize}

The overall scheme for writing a TCharm-based library "Foo" is:

\begin{enumerate}
\item You must provide a FOO\_Init routine that creates anything
you'll need, which normally includes a Chare Array of your own
objects.  The user will call your FOO\_Init routine
from their main work routine; and normally FOO\_Init routines 
are collective.

\item In your FOO\_Init routine, create your array bound it to the 
running TCharm threads, by creating it using the CkArrayOptions 
returned by TCHARM\_Attach\_start.  Be sure to only create the 
array once, by checking if you're the master before creating the
array.  

One simple way to make the non-master threads block until the corresponding
local array element is created is to use TCharm semaphores.  
These are simply a one-pointer slot you can assign using TCharm::semaPut
and read with TCharm::semaGet.  They're useful in this context
because a TCharm::semaGet blocks if a local TCharm::semaGet hasn't
yet executed.  

\begin{alltt}
//This is either called by FooFallbackSetuo mentioned above, or by the user
//directly from TCHARM\_User\_setup (for multi-module programs)
void FOO\_Init(void)
\{
  if (TCHARM\_Element()==0) \{
    CkArrayID threadsAID; int nchunks;
    CkArrayOptions opts=TCHARM\_Attach\_start(&threadsAID,&nchunks);
  
  //actually create your library array here (FooChunk in this case)
    CkArrayID aid = CProxy\_FooChunk::ckNew(opt);
  \}
  FooChunk *arr=(FooChunk *)TCharm::semaGet(FOO_TCHARM_SEMAID);
\}
\end{alltt}

\item Depending on your library API, you may have to
set up a thread-private variable(Ctv) to point to your library object. 
This is needed to regain context when you are called by the user.
A better design is to avoid the Ctv, and instead hand the user 
an opaque handle that includes your array proxy.

\begin{alltt}
//\_fooptr is the Ctv that points to the current chunk FooChunk and is only valid in 
//routines called from fooDriver()
CtvStaticDeclare(FooChunk *, \_fooptr);

/* The following routine is listed as an initcall in the .ci file */
/*initcall*/ void fooNodeInit(void)
\{
  CtvInitialize(FooChunk*, \_fooptr);
\}
\end{alltt}

\item Define the array used by the library

\begin{alltt}
class FooChunk: public TCharmClient1D \{
   CProxy\_FooChunk thisProxy;
protected:
   //called by TCharmClient1D when thread changes
   virtual void setupThreadPrivate(CthThread forThread)
   \{
      CtvAccessOther(forThread, \_fooptr) = this;
   \}
   
   FooChunk(CkArrayID aid):TCharmClient1D(aid)
   \{
      thisProxy = this;
      tCharmClientInit();
      TCharm::semaPut(FOO_TCHARM_SEMAID,this);
      //add any other initialization here
   \}

   virtual void pup(PUP::er &p) \{
     TCharmClient1D::pup(p);
     //usual pup calls
   \}
   
   // ...any other calls you need...
   int doCommunicate(...);
   void recvReply(someReplyMsg *m);
   ........
\}
\end{alltt}


\item Block a thread for communication using thread->suspend and
thread->resume

\begin{alltt}
int FooChunk::doCommunicate(...)
\{
   replyGoesHere = NULL;
   thisProxy[destChunk].sendRequest(...);
   thread->suspend(); //wait for reply to come back
   return replyGoesHere->data;
\}

void FooChunk::recvReply(someReplyMsg *m)
\{
  if(replyGoesHere!=NULL) CkAbort("FooChunk: unexpected reply\\n");
  replyGoesHere = m;
  thread->resume(); //Got the reply -- start client again
\}
\end{alltt}


\item Add API calls. This is how user code running in the thread interacts
with the newly created library. Calls to TCHARM\_API\_TRACE macro must be 
added to the start of every user-callable method. In addition to tracing,
these disable isomalloc allocation. 

The charm-api.h macros CDECL, FDECL and
FTN\_NAME should be used to provide both C and FORTRAN versions of each 
API call.  You should use the "MPI capitalization standard", where the library
name is all caps, followed by a capitalized first word, with all subsequent 
words lowercase, separated by underscores.  This capitalization system is 
consistent, and works well with case-insensitive languages like Fortran.

Fortran parameter passing is a bit of an art, but basically for simple 
types like int (INTEGER in fortran), float (SINGLE PRECISION or REAL*4),
and double (DOUBLE PRECISION or REAL*8), things work well.  Single parameters
are always passed via pointer in Fortran, as are arrays.  Even though 
Fortran indexes arrays based at 1, it will pass you a pointer to the 
first element, so you can use the regular C indexing.  The only time Fortran
indexing need be considered is when the user passes you an index--the
int index will need to be decremented before use, or incremented before
a return.

\begin{alltt}
CDECL void FOO\_Communicate(int x, double y, int * arr) \{
   TCHARM\_API\_TRACE("FOO\_Communicate", "foo"); //2nd parameter is the name of the library
   FooChunk *f = CtvAccess(\_fooptr);
   f->doCommunicate(x, y, arr);
\}

//In fortran, everything is passed via pointers
FDECL void FTN\_NAME(FOO_COMMUNICATE, foo_communicate)
     (int *x, double *y, int *arr)
\{
   TCHARM\_API\_TRACE("FOO_COMMUNICATE", "foo");
   FooChunk *f = CtvAccess(\_fooptr);
   f->doCommunicate(*x, *y, arr); 
\}
\end{alltt}

\end{enumerate}

\input{index}
\end{document}
