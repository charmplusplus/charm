\chapter{Machine Interface and Scheduler}

This chapter describes two of \converse{}'s modules: the CMI, and the
CSD.  Together, they serve to transmit messages and schedule the
delivery of messages. First, we describe the machine model assumed by
\converse{}.

\section{Machine Model}
\label{model}

\converse{} treats the parallel machine as a collection of {\em nodes}, where
each node is comprised of a number of {\em processors} that share memory 
In some cases, the number of processors per node may be exactly one  
(e.g. Distributed memory multicomputers such as IBM SP.)  
In addition, each of the processors may have multiple {\em threads} running on
them which share code and data but have different stacks.
Functions and macros are provided for handling shared memory across
processors and querying node information. These are discussed in section
\ref{globalvars}

\section{Defining Handler Numbers}
\label{handler1}

When a message arrives at a processor, it triggers the execution of a
{\em handler function}, not unlike a UNIX signal handler.  The handler
function receives, as an argument, a pointer to the message.
The message itself specifies which handler function is to be
called when the message arrives.  Messages are contiguous sequences of
bytes.  The message has two parts: the header, and the data.  The data
may contain anything you like.  The header contains a {\em handler
number}, which specifies which handler function is to be executed when
the message arrives.  Before you can send a message, you have to
define the handler numbers.

\converse{} maintains a table mapping handler numbers to function
pointers.  Each processor has its own copy of the mapping.  There is a
caution associated with this approach: it is the user's responsibility
to ensure that all processors have identical mappings.  This is easy
to do, nonetheless, and the user must be aware that this is (usually)
required.

The following functions are provided to define the handler numbers:

\function{typedef void (*CmiHandler)(void *)}
\index{CmiHandler}
\desc{Functions that handle \converse{} messages must be of this type.}

\function{int CmiRegisterHandler(CmiHandler h)}
\index{CmiRegisterHandler}
\desc{This represents the standard technique for associating numbers
with functions.  To use this technique, the \converse{} user registers
each of his functions, one by one, using \kw{CmiRegisterHandler}.  One must
register exactly the same functions in exactly the same order on all
processors.  The system assigns monotonically increasing numbers to
the functions, the same numbers on all processors.  This insures
global consistency.  \kw{CmiRegisterHandler} returns the number which was
chosen for the function being registered.}

\function {int CmiRegisterHandlerGlobal(CmiHandler h)}
\index{CmiRegisterHandlerLocal}
\desc{This represents a second registration technique.   The \converse{}
user registers his functions on processor zero, using
\kw{CmiRegisterHandlerGlobal}.  The \converse{} user is then responsible for
broadcasting those handler numbers to other processors, and installing
them using \kw{CmiNumberHandler} below.  The user should take care not to
invoke those handlers until they are fully installed.}

\function {int CmiRegisterHandlerLocal(CmiHandler h)}
\index{CmiRegisterHandlerLocal}
\desc{This function is used when one wishes to register functions
in a manner that is not consistent across processors.  This function
chooses a locally-meaningful number for the function, and records it
locally.  No attempt is made to ensure consistency across processors.}

\function {void CmiNumberHandler(int n, CmiHandler h)}
\index{CmiNumberHandler}
\desc{Forces the system to associate the specified handler number \uw{n}
with the specified handler function \uw{h}.  If the function number
\uw{n} was previously mapped to some other function, that old mapping
is forgotten.  The mapping that this function creates is local to the
current processor.  \kw{CmiNumberHandler} can be useful in combination with
\kw{CmiRegisterGlobalHandler}.  It can also be used to implement
user-defined numbering schemes: such schemes should keep in mind that
the size of the table that holds the mapping is proportional to the
largest handler number --- do not use big numbers!}

\note{Of the three registration methods, the \kw{CmiRegisterHandler} method
is by far the simplest, and is strongly encouraged.  The others are
primarily to ease the porting of systems that already use similar
registration techniques.  One may use all three registration methods
in a program.  The system guarantees that no numbering conflicts will
occur as a result of this combination.}

\section{Writing Handler Functions}
\label{handler2}

A message handler function is just a C function that accepts a void
pointer (to a message buffer) as an argument, and returns nothing.  The
handler may use the message buffer for any purpose, but is responsible
for eventually deleting the message using CmiFree.

\section{Building Messages}

To send a message, one first creates a buffer to hold the message.
The buffer must be large enough to hold the header and the data.
The buffer can be in any kind of memory: it could be a local variable,
it could be a global, it could be allocated with {\tt malloc}, and
finally, it could be allocated with \kw{CmiAlloc}.  The \converse{} user
fills the buffer with the message data.  One puts a handler number
in the message, thereby specifying which handler function the message
should trigger when it arrives.  Finally, one uses a message-transmission
function to send the message.

The following functions are provided to help build message buffers:

\function{void *CmiAlloc(int size)}
\index{CmiAlloc}
\desc{Allocates memory of size \uw{size} in heap and returns pointer to 
the usable space.  There are some message-sending functions that
accept only message buffers that were allocated with \kw{CmiAlloc}.  Thus,
this is the preferred way to allocate message buffers. The returned pointer
point to the message header, the user data will follow it. See
\kw{CmiMsgHeaderSizeBytes} for this.}

\function{void CmiFree(void *ptr)}
\index{CmiFree}
\desc{This function frees the memory pointed to by \uw{ptr}. \uw{ptr}
should be a pointer that was previously returned by \kw{CmiAlloc}.}

\function {\#define CmiMsgHeaderSizeBytes}
\index{CmiMsgHeaderSizeBytes}
\desc{This constant contains the size of the message header.  When one
allocates a message buffer, one must set aside enough space for the header
and the data.  This macro helps you to do so. For example, if one want to
allocate an array of 100 int, he should call the function this way:
\tt{``char *myMsg = CmiAlloc(100*sizeof(int) + CmiMsgHeaderSizeBytes)''}}

\function {void CmiSetHandler(int *MessageBuffer, int HandlerId)}
\index{CmiSetHandler}
\desc{This macro sets the handler number of a message to \uw{HandlerId}.}

\function {int CmiGetHandler(int *MessageBuffer)}
\index{CmiGetHandler}
\desc{This call returns the handler of a message in the form of a
handler number.}
 
\function {CmiHandler CmiGetHandlerFunction(int *MessageBuffer)}
\index{CmiGetHandlerFunction}
\desc{This call returns the handler of a message in the form of a
function pointer.}

\section{Sending Messages}

The following functions allow you to send messages.  Our model is that
the data starts out in the message buffer, and from there gets
transferred ``into the network''.  The data stays ``in the network''
for a while, and eventually appears on the target processor.  Using
that model, each of these send-functions is a device that transfers
data into the network.  None of these functions wait for the data to
be delivered.

On some machines, the network accepts data rather slowly.  We don't
want the process to sit idle, waiting for the network to accept the
data.  So, we provide several variations on each send function:

\begin{itemize}

\item{{\bf sync}: a version that is as simple as possible, pushing the
data into the network and not returning until the data is ``in the
network''.  As soon as a sync function returns, you can reuse the
message buffer.}

\item{{\bf async}: a version that returns almost instantaneously, and then
continues working in the background.  The background job transfers the
data from the message buffer into the network.  Since the background job
is still using the message buffer when the function returns, you can't
reuse the message buffer immediately.  The background job sets a flag
when it is done and you can then reuse the message buffer.}

\item{{\bf send and free}: a version that returns almost instantaneously,
and then continues working in the background.  The background job
transfers the data from the message buffer into the network.  When the
background job finishes, it \kw{CmiFree}s the message buffer.  In
this situation, you can't reuse the message buffer at all. 
To use a function of this type, you must allocate the message buffer
using \kw{CmiAlloc}.}

\item{{\bf node}\experimental{}: a version that send a message to a node 
instead of a
specific processor. This means that when the message is received, any ``free''
processor within than node can handle it.}

\end{itemize}

\function{void CmiSyncSend(unsigned int destPE, unsigned int size, void *msg)}
\index{CmiSyncSend}
\desc{Sends \uw{msg} of size \uw{size} bytes to processor
\uw{destPE}.  When it returns, you may reuse the message buffer.}

\function{void CmiSyncNodeSend(unsigned int destNode, unsigned int size, void *msg)}
\index{CmiSyncNodeSend}
\desc{Sends\experimental{} \uw{msg} of size \uw{size} bytes to node
\uw{destNode}.  When it returns, you may reuse the message buffer.}

\function{void CmiSyncSendAndFree(unsigned int destPE, unsigned int size, void *msg)}
\index{CmiSyncSendAndFree}
\desc{Sends \uw{msg} of size \uw{size} bytes to processor
\uw{destPE}.  When it returns, the message buffer has been freed
using \kw{CmiFree}.}

\function{void CmiSyncNodeSendAndFree(unsigned int destNode, unsigned int size, void *msg)}
\index{CmiSyncNodeSendAndFree}
\desc{Sends\experimental{} \uw{msg} of size \uw{size} bytes to node
\uw{destNode}.  When it returns, the message buffer has been freed
using \kw{CmiFree}.}

\function{CmiCommHandle CmiAsyncSend(unsigned int destPE, unsigned int size, void *msg)}
\index{CmiAsyncSend}
\desc{Sends \uw{msg} of size \uw{size} bytes to processor
\uw{destPE}.  It returns a communication handle which can be
tested using \kw{CmiAsyncMsgSent}: when this returns true, you may reuse
the message buffer. If the returned communication handle is 0, message buffer
can be reused immediately, thus saving a call to \kw{CmiAsyncMsgSent}.}

\function{CmiCommHandle CmiAsyncNodeSend(unsigned int destNode, unsigned int size, void *msg)}
\index{CmiAsyncNodeSend}
\desc{Sends\experimental{} \uw{msg} of size \uw{size} bytes to node
\uw{destNode}.  It returns a communication handle which can be
tested using \kw{CmiAsyncMsgSent}: when this returns true, you may reuse
the message buffer. If the returned communication handle is 0, message buffer
can be reused immediately, thus saving a call to \kw{CmiAsyncMsgSent}.}

\function{void CmiSyncVectorSend(int destPE, int len, int sizes[], char *msgComps[])}
\desc{Concatenates several pieces of data and sends them to processor
\uw{destPE}.  The data consists of \uw{len} pieces residing in
different areas of memory, which are logically concatenated.  The
\uw{msgComps} array contains pointers to the pieces; the size of
\uw{msgComps[i]} is taken from \uw{sizes[i]}. 
When it returns, \uw{sizes}, \uw{msgComps} and the message
components specified in \uw{msgComps} can be immediately reused.}

\function{void CmiSyncVectorSendAndFree(int destPE, int len, int sizes[], char *msgComps[])}
\desc{Concatenates several pieces of data and sends them to processor
\uw{destPE}.  The data consists of \uw{len} pieces residing in
different areas of memory, which are logically concatenated.  The
\uw{msgComps} array contains pointers to the pieces; the size of
\uw{msgComps[i]} is taken from \uw{sizes[i]}. 
The message components specified in \uw{msgComps} are \kw{CmiFree}d 
by this function therefore, they should be dynamically
allocated using \kw{CmiAlloc}.  However, the \uw{sizes} and
\uw{msgComps} array themselves are not freed.}

\function{CmiCommHandle CmiAsyncVectorSend(int destPE, int len, int sizes[], char *msgComps[])}
\desc{Concatenates several pieces of data and sends them to processor
\uw{destPE}.  The data consists of \uw{len} pieces residing in
different areas of memory, which are logically concatenated.  The
\uw{msgComps} array contains pointers to the pieces; the size of
\uw{msgComps[i]} is taken from \uw{sizes[i]}. 
The individual pieces of data as well as the arrays \uw{sizes} and
\uw{msgComps} should not be overwritten or freed before the
communication is complete.  This function returns a communication
handle which can be tested using \kw{CmiAsyncMsgSent}: when this returns
true, the input parameters can be reused. If the returned communication 
handle is 0, message buffer
can be reused immediately, thus saving a call to \kw{CmiAsyncMsgSent}.}

\function{int CmiAsyncMsgSent(CmiCommHandle handle)}
\index{CmiAsyncMsgSent}
\desc{Returns true if the communication specified by the given
\kw{CmiCommHandle} has proceeded to the point where the message buffer can
be reused.}

\function{void CmiReleaseCommHandle(CmiCommHandle handle)}
\index{CmiReleaseCommHandle}
\desc{Releases the communication handle \uw{handle} and
associated resources. It does not free the message buffer.}

\function{void CmiMultipleSend(unsigned int destPE, int len, int sizes[], char
*msgComps[])}
\index{CmiMultipleSend}
\desc{This function\experimental{} allows the user to send 
multiple messages that may be
destined for the SAME PE in one go. This is more efficient than sending
each message to the destination node separately. This function assumes
that the handlers that are to receive this message have already been set.
If this is not done, the behavior of the function is undefined.

In the function, The \uw{destPE} parameter identifies the destination
processor.
The \uw{len} argument identifies the {\it number} of messages that are to
be sent in one go. 
The \uw{sizes[]} array is an array of sizes of each of these messages.
The \uw{msgComps[]} array is the array of the messages. 
The indexing in each array is from 0 to len - 1.
\note{
Before calling this function, the program needs to initialise the system
to be able to provide this service. This is done by calling the function
\kw{CmiInitMultipleSendRoutine}. Unless this function is
called, the system will not be able to provide the service to the user.}
}

\section{Broadcasting Messages}

\function{void CmiSyncBroadcast(unsigned int size, void *msg)}
\index{CmiSyncBroadcast}
\desc{Sends \uw{msg} of length \uw{size} bytes to all processors
excluding the processor on which the caller resides. }

\function{void CmiSyncNodeBroadcast(unsigned int size, void *msg)}
\index{CmiSyncNodeBroadcast}
\desc{Sends \uw{msg} of length \uw{size} bytes to all nodes
excluding the node on which the caller resides. }

\function{void CmiSyncBroadcastAndFree(unsigned int size, void *msg)}
\index{CmiSyncBroadcastAndFree}
\desc{Sends \uw{msg} of length \uw{size} bytes to all processors
excluding the processor on which the caller resides.  Uses \kw{CmiFree} to 
deallocate the message buffer for \uw{msg} when the
broadcast completes. Therefore \uw{msg} must point to a buffer
allocated with \kw{CmiAlloc}.}

\function{void CmiSyncNodeBroadcastAndFree(unsigned int size, void *msg)}
\index{CmiSyncNodeBroadcastAndFree}
\desc{Sends \uw{msg} of length \uw{size} bytes to all nodes
excluding the node on which the caller resides.  Uses \kw{CmiFree} to 
deallocate the message buffer for \uw{msg} when the
broadcast completes. Therefore \uw{msg} must point to a buffer
allocated with \kw{CmiAlloc}.}

\function{void CmiSyncBroadcastAll(unsigned int size, void *msg)}
\index{CmiSyncBroadcastAll}
\desc{Sends \uw{msg} of length \uw{size} bytes to all processors
including the processor on which the caller resides. This function
does not free the message buffer for \uw{msg}.}

\function{void CmiSyncNodeBroadcastAll(unsigned int size, void *msg)}
\index{CmiSyncNodeBroadcastAll}
\desc{Sends \uw{msg} of length \uw{size} bytes to all nodes
including the node on which the caller resides. This function
does not free the message buffer for \uw{msg}.}

\function{void CmiSyncBroadcastAllAndFree(unsigned int size, void *msg)}
\index{CmiSyncBroadcastAllAndFree}
\desc{Sends \uw{msg} of length \uw{size} bytes to all processors
including the processor on which the caller resides. This function
frees the message buffer for \uw{msg} before returning, so
\uw{msg} must point to a dynamically allocated buffer.}

\function{void CmiSyncNodeBroadcastAllAndFree(unsigned int size, void *msg)}
\index{CmiSyncNodeBroadcastAllAndFree}
\desc{Sends \uw{msg} of length \uw{size} bytes to all nodes
including the node on which the caller resides. This function
frees the message buffer for \uw{msg} before returning, so
\uw{msg} must point to a dynamically allocated buffer.}

\function{CmiCommHandle CmiAsyncBroadcast(unsigned int size, void *msg)}
\index{CmiAsyncBroadcast}
\desc{Initiates asynchronous broadcast of message \uw{msg} of
length \uw{size} bytes to all processors excluding the processor on
which the caller resides. It returns a communication handle which
could be used to check the status of this send using
\kw{CmiAsyncMsgSent}. If the returned communication handle is 0, 
message buffer can be reused immediately, thus saving a call to 
\kw{CmiAsyncMsgSent}. \uw{msg} should not be overwritten or
freed before the communication is complete.}

\function{CmiCommHandle CmiAsyncNodeBroadcast(unsigned int size, void *msg)}
\index{CmiAsyncNodeBroadcast}
\desc{Initiates asynchronous broadcast of message \uw{msg} of
length \uw{size} bytes to all nodes excluding the node on
which the caller resides. It returns a communication handle which
could be used to check the status of this send using
\kw{CmiAsyncMsgSent}. If the returned communication handle is 0, 
message buffer can be reused immediately, thus saving a call to 
\kw{CmiAsyncMsgSent}. \uw{msg} should not be overwritten or
freed before the communication is complete.}

\function{CmiCommHandle CmiAsyncBroadcastAll(unsigned int size, void *msg)}
\index{CmiAsyncBroadcastAll}
\desc{Initiates asynchronous broadcast of message \uw{msg} of
length \uw{size} bytes to all processors including the processor on
which the caller resides. It returns a communication handle which
could be used to check the status of this send using
\kw{CmiAsyncMsgSent}. If the returned communication handle is 0, 
message buffer can be reused immediately, thus saving a call to 
\kw{CmiAsyncMsgSent}. \uw{msg} should not be overwritten or
freed before the communication is complete.}

\function{CmiCommHandle CmiAsyncNodeBroadcastAll(unsigned int size, void *msg)}
\index{CmiAsyncNodeBroadcastAll}
\desc{Initiates asynchronous broadcast of message \uw{msg} of
length \uw{size} bytes to all nodes including the node on
which the caller resides. It returns a communication handle which
could be used to check the status of this send using
\kw{CmiAsyncMsgSent}. If the returned communication handle is 0, 
message buffer can be reused immediately, thus saving a call to 
\kw{CmiAsyncMsgSent}. \uw{msg} should not be overwritten or
freed before the communication is complete.}

\section{Multicasting Messages}
\label{sec:multicast}

\function{typedef ... CmiGroup;}
\index{CmiGroup}
\desc{A \kw{CmiGroup} represents a set of processors.  It is an opaque type.
Group IDs are useful for the multicast functions below.}

\function{CmiGroup CmiEstablishGroup(int npes, int *pes);}
\index{CmiGroup}
\index{CmiEstablishGroup}
\desc{Converts an array of processor numbers into a group ID.  Group
IDs are useful for the multicast functions below.  Caution: this call
uses up some resources.  In particular, establishing a group uses 
some network bandwidth (one broadcast's worth) and a small amount of
memory on all processors.}

\function{void CmiSyncMulticast(CmiGroup grp, unsigned int size, void *msg)}
\index{CmiSyncMulticast}
\desc{Sends \uw{msg} of length \uw{size} bytes to all members
of the specified group.  Group IDs are created using
\kw{CmiEstablishGroup}.}

\function{void CmiSyncMulticastAndFree(CmiGroup grp, unsigned int size, void *msg)}
\index{CmiSyncMulticastAndFree}
\desc{Sends \uw{msg} of length \uw{size} bytes to all members
of the specified group.  Uses \kw{CmiFree} to deallocate the
message buffer for \uw{msg} when the broadcast completes. Therefore
\uw{msg} must point to a buffer allocated with \kw{CmiAlloc}.
Group IDs are created using \kw{CmiEstablishGroup}.}

\function{CmiCommHandle CmiAsyncMulticast(CmiGroup grp, unsigned int size, void *msg)}
\index{CmiAsyncMulticast}
\desc{\note{Not yet implemented.} Initiates asynchronous broadcast of
message \uw{msg} of length \uw{size} bytes to all members of
the specified group.  It returns a communication handle which could
be used to check the status of this send using
\kw{CmiAsyncMsgSent}. If the returned communication handle is 0, 
message buffer can be reused immediately, thus saving a call to 
\kw{CmiAsyncMsgSent}. \uw{msg} should not be overwritten or
freed before the communication is complete.
Group IDs are created using \kw{CmiEstablishGroup}.}

\function{void CmiSyncListSend(int npes, int *pes, unsigned int size, void *msg)}
\index{CmiSyncListSend}
\desc{Sends \uw{msg} of length \uw{size} bytes to \uw{npes} processors in the
array \uw{pes}.}

\function{void CmiSyncListSendAndFree(int npes, int *pes, unsigned int size, void *msg)}
\index{CmiSyncListSendAndFree}
\desc{Sends \uw{msg} of length \uw{size} bytes to \uw{npes} processors in the
array \uw{pes}. Uses \kw{CmiFree} to deallocate the message buffer for \uw{msg}
when the multicast completes. Therefore, \uw{msg} must point to a buffer
allocated with \kw{CmiAlloc}.}

\function{CmiCommHandle CmiAsyncListSend(int npes, int *pes, unsigned int size, void *msg)}
\index{CmiAsyncListSend}
\desc{Initiates asynchronous multicast of message \uw{msg} of length \uw{size}
bytes to \uw{npes} processors in the array \uw{pes}. It returns a communication
handle which could be used to check the status of this send using
\kw{CmiAsyncMsgSent}. If the returned communication handle is 0, message buffer
can be reused immediately, thus saving a call to \kw{CmiAsyncMsgSent}. \uw{msg}
should not be overwritten or freed before the communication is complete.}

\section{Reducing Messaging}
\label{reduce}

Reductions are operations for which a message (or user data structure) is
contributed by each participant processor. All these contributions are merged
according to a merge-function provided by the user. A Converse handler is then
invoked with the resulting message. Reductions can be on the entire set of
processors, or on a subset of the whole.
Currently reductions are only implemented on processors sets. No equivalent
exists for SMP nodes.

There are eight functions used to deposit a message into the system, summarized
in Table~\ref{table:reductions}. Half of them receive as contribution a Converse
message (with a Converse header at its beginning). This message must have
already been set for delivery to the desired handler. The other half (ending
with ``Struct'') receives a pointer to a data structure allocated by the user. 
This second version may allow the user to write a simpler merging function. For
instance, the data structure could be a tree that can be easily expanded by
adding more nodes.

\begin{table}[h]
\begin{center}
\begin{tabular}{|l|llll|}
\hline
 & {\bf global} & {\bf global with ID} & {\bf processor set} & {\bf CmiGroup} \\
\hline
{\bf message} & CmiReduce & CmiReduceID & CmiListReduce & CmiGroupReduce \\
{\bf data} & CmiReduceStruct & CmiReduceStructID & CmiListReduceStruct & CmiGroupReduceStruct \\
\hline
\end{tabular}
\end{center}
\caption{Reductions functions in Converse}
\label{table:reductions}
\end{table}

The signatures for the functions in Table~\ref{table:reductions} are:

\function{void CmiReduce(void *msg, int size, CmiReduceMergeFn mergeFn);}
\function{void CmiReduceStruct(void *data, CmiReducePupFn pupFn,
                     CmiReduceMergeFn mergeFn, CmiHandler dest,
                     CmiReduceDeleteFn deleteFn);}
\function{void CmiReduceID(void *msg, int size, CmiReduceMergeFn mergeFn, CmiReductionID id);}
\function{void CmiReduceStructID(void *data, CmiReducePupFn pupFn,
                     CmiReduceMergeFn mergeFn, CmiHandler dest,
                     CmiReduceDeleteFn deleteFn, CmiReductionID id);}
\function{void CmiListReduce(int npes, int *pes, void *msg, int size, CmiReduceMergeFn mergeFn, CmiReductionID id);}
\function{void CmiListReduceStruct(int npes, int *pes,
                     void *data, CmiReducePupFn pupFn,
                     CmiReduceMergeFn mergeFn, CmiHandler dest,
                     CmiReduceDeleteFn deleteFn, CmiReductionID id);}
\function{void CmiGroupReduce(CmiGroup grp, void *msg, int size, CmiReduceMergeFn mergeFn, CmiReductionID id);}
\function{void CmiGroupReduceStruct(CmiGroup grp, void *data, CmiReducePupFn pupFn,
                     CmiReduceMergeFn mergeFn, CmiHandler dest,
                     CmiReduceDeleteFn deleteFn, CmiReductionID id);}

In all the above, \uw{msg} is the Converse message deposited by the local
processor, \uw{size} is the size of the message \uw{msg}, and \uw{data} is a
pointer to the user-allocated data structure deposited by the local processor.
\uw{dest} is the \kw{CmiHandler} where the final message shall be delivered. It
is explicitly passed in ``Struct'' functions only, since for the message
versions it is taken from the header of \uw{msg}. Moreover there are several
other function pointers passed in by the user:

\function{void * (*mergeFn)(int *size, void *local, void **remote, int count)}
\desc{Prototype for a \kw{CmiReduceMergeFn} function pointer argument.
This function is used in all the CmiReduce forms to merge the local
message/data structure deposited on a processor with all the messages incoming
from the children processors of the reduction spanning tree. The input parameters
are in the order: the size of the local data for message reductions (always zero
for struct reductions); the local data itself (the exact same pointer passed in as first
parameter of CmiReduce and similar); a pointer to an array of incoming messages;
the number of elements in the second parameter. The function returns a pointer
to a freshly allocated message (or data structure for the \kw{Struct} forms)
corresponding to the merge of all the messages. When performing message
reductions, this function is also responsible to updating the integer pointed by
\uw{size} to the new size of the returned message. All the messages in the
\uw{remote} array are deleted by the system; the data pointed by the first parameter
should be deleted by this function. If the data can be merged ``in-place'' by
modifying or augmenting \uw{local}, the function can return the same pointer to
\uw{local} which can be considered freshly allocated. Each element in \uw{remote}
is the complete incoming message (including the converse header) for message
reductions, and the data as it has been packed by the pup function (without any
additional header) for struct reductions.}

\function{void (*pupFn)(pup\_er p, void *data)}
\desc{Prototype for a \kw{CmiReducePupFn} function pointer argument.
This function will use the PUP framework to pup the \uw{data} passed in
into a message for sending across the network. The data can be either the same
data passed in as first parameter of any ``Struct'' function, or
the return of the merge function. It will be called for sizing and packing.
\note{It will not be called for unpacking.}}

\function{void (*deleteFn)(void *ptr)}
\desc{Prototype for a \kw{CmiReduceDeleteFn} function pointer argument.
This function is used to delete either the data stucture passed in as first
parameter of any ``Struct'' function, or the return of the merge
function. It can be as simple as ``free'' or as complicated as needed to delete
complex structures. If this function is NULL, the data structure will not be
deleted, and the program can continue to use it. Note: even if this function is
NULL, the input data structure may still be modified by the merge function.}

\uw{CmiReduce} and \uw{CmiReduceStruct} are the simplest reduction function, and
they reduce the deposited message/data across all the processors in the
system. Each processor must to call this function exactly once.
Multiple reductions can be invoked without waiting for previous ones to finish,
but the user is responsible to call CmiReduce/CmiReduceStruct in the same
order on every processor. \note{CmiReduce and CmiReduceStruct are not
interchangeable. Either every processor calls CmiReduce or every processor calls
CmiReduceStruct}.

In situations where it is not possible to guarantee the order of reductions, the
user may use \uw{CmiReduceID} or \uw{CmiReduceStructID}. These functions have an
additional parameter of type \kw{CmiReductionID} which will uniquely identify the
reduction, and match them correctly. \note{No two reductions can be active at
the same time with the same CmiReductionID. It is up to the user to guarantee
this.}

A \kw{CmiReductionID} can be obtained by the user in three ways, using one of
the following functions:

\function{CmiReductionID CmiGetGlobalReduction()}
\desc{This function must be called on every processor, and in the same order if
called multiple times. This would generally be inside initialization code, that
can set aside some CmiReductionIDs for later use.}

\function{CmiReductionID CmiGetDynamicReduction()}
\desc{This function may be called only on processor zero. It returns a unique
ID, and it is up to the user to distrubute this ID to any processor that needs
it.}

\function{void CmiGetDynamicReductionRemote(int handlerIdx, int pe, int dataSize, void *data)}
\desc{This function may be called on any processor. The produced CmiReductionID
is returned on the specified \uw{pe} by sending a message to the specified
\uw{handlerIdx}. If \uw{pe} is -1, then all processors will receive the
notification message. \uw{data} can be any data structure that the user wants to
receive on the specified handler (for example to differentiate between
requests). \uw{dataSize} is the size in bytes of \uw{data}. If \uw{dataSize} is
zero, \uw{data} is ignored.
The message received by \uw{handlerIdx} consists of the standard Converse
header, followed by the requested CmiReductionID (represented as a 4 bytes
integer the user can cast to a \kw{CmiReductionID}, a 4 byte integer containing
\uw{dataSize}, and the \uw{data} itself.}

The other four functions (CmiListReduce, CmiListReduceStruct, CmiGroupReduce,
CmiGroupReduceStruct) are used for reductions over subsets of processors. They
all require a CmiReductionID that the user must obtain in one of the ways
described above. The user is also responsible that no two reductions use the
same CmiReductionID simultaneously. The first two functions receive the subset
description as processor list (\uw{pes}) of size \uw{npes}. The last two receive
the subset description as a previously established CmiGroup
(see~\ref{sec:multicast}).

\section{Scheduling Messages}
\label{schedqueue}

The scheduler queue is a powerful priority queue.  The following
functions can be used to place messages into the scheduler queue.
These messages are treated very much like newly-arrived messages: when
they reach the front of the queue, they trigger handler functions,
just like messages transmitted with CMI functions.  Note that unlike
the CMI send functions, these cannot move messages across processors.

Every message inserted into the queue has a priority associated with
it.  \converse{} priorities are arbitrary-precision numbers between 0 and
1.  Priorities closer to 0 get processed first, priorities closer to 1
get processed last.  Arbitrary-precision priorities are very useful in
AI search-tree applications. Suppose we have a heuristic suggesting
that tree node N1 should be searched before tree node N2. We therefore
designate that node N1 and its descendants will use high priorities,
and that node N2 and its descendants will use lower priorities. We
have effectively split the range of possible priorities in two. If
several such heuristics fire in sequence, we can easily split the
priority range in two enough times that no significant bits remain,
and the search begins to fail for lack of meaningful priorities to
assign. The solution is to use arbitrary-precision priorities, aka
bitvector priorities.

These arbitrary-precision numbers are represented as bit-strings: for
example, the bit-string ``0011000101'' represents the binary number
(.0011000101).  The format of the bit-string is as follows: the
bit-string is represented as an array of unsigned integers. The most
significant bit of the first integer contains the first bit of the
bitvector.  The remaining bits of the first integer contain the next
31 bits of the bitvector.  Subsequent integers contain 32 bits
each. If the size of the bitvector is not a multiple of 32, then the
last integer contains 0 bits for padding in the least-significant bits
of the integer.

Some people only want regular integers as priorities.  For
simplicity's sake, we provide an easy way to convert integer
priorities to \converse{}'s built-in representation.

In addition to priorities, you may choose to enqueue a message
``LIFO'' or ``FIFO''.  Enqueueing a message ``FIFO'' simply pushes it
behind all the other messages of the same priority.  Enqueueing a
message ``LIFO'' pushes it in front of other messages of the same
priority.

Messages sent using the CMI functions take precedence over everything
in the scheduler queue, regardless of priority.

A recent addition\experimental{} to \converse{} scheduling mechanisms is 
the introduction of
node-level scheduling designed to support low-overhead programming for the
SMP clusters. These functions have ``Node'' in their names. All processors
within the node has access to the node-level scheduler's queue, and thus
a message enqueued in a node-level queue may be handled by any processor within
that node. When deciding about which message to process next, i.e. from
processor's own queue or from the node-level queue, a quick priority check
is performed internally, thus a processor views scheduler's queue as a single
prioritized queue that includes messages directed at that processor and
messages from the node-level queue sorted according to priorities.

\function{void CsdEnqueueGeneral(void *Message, int strategy, int priobits, int *prioptr)}
\index{CsdEnqueueGeneral}
\desc{This call enqueues a message to the processor's scheduler's queue, to
be sorted according to its priority and the queueing \param{strategy}.
The meaning of the \uw{priobits} and \uw{prioptr} fields depend
on the value of \uw{strategy}, which are explained below.}

\function{void CsdNodeEnqueueGeneral(void *Message, int strategy, int priobits, int *prioptr)}
\index{CsdNodeEnqueueGeneral}
\desc{This call enqueues a message to the node-level scheduler's queue, to
be sorted according to its priority and the queueing \uw{strategy}.
The meaning of the \uw{priobits} and \uw{prioptr} fields depend
on the value of \uw{strategy}, which can be any of the following:

\begin{itemize}
\item{\kw{CQS\_QUEUEING\_BFIFO}: the priobits and prioptr point to
a bit-string representing an arbitrary-precision priority.  The message
is pushed behind all other message of this priority.}

\item{\kw{CQS\_QUEUEING\_BLIFO}: the priobits and prioptr point to
a bit-string representing an arbitrary-precision priority.  The message
is pushed in front all other message of this priority.}

\item{\kw{CQS\_QUEUEING\_IFIFO}: the prioptr is a pointer to a
signed integer.  The integer is converted to a bit-string priority,
normalizing so that the integer zero is converted to the bit-string
``1000...'' (the ``middle'' priority).  To be more specific, the
conversion is performed by adding 0x80000000 to the integer, and then
treating the resulting 32-bit quantity as a 32-bit bitvector priority.
The message is pushed behind all other messages of this priority.}

\item{\kw{CQS\_QUEUEING\_ILIFO}: the prioptr is a pointer to a
signed integer.  The integer is converted to a bit-string priority,
normalizing so that the integer zero is converted to the bit-string
``1000...'' (the ``middle'' priority).  To be more specific, the
conversion is performed by adding 0x80000000 to the integer, and then
treating the resulting 32-bit quantity as a 32-bit bitvector priority.
The message is pushed in front of all other messages of this
priority.}

\item{\kw{CQS\_QUEUEING\_FIFO}: the prioptr and priobits are ignored.
The message is enqueued with the middle priority ``1000...'', and is
pushed behind all other messages with this priority.}

\item{\kw{CQS\_QUEUEING\_LIFO}: the prioptr and priobits are ignored.
The message is enqueued with the middle priority ``1000...'', and is
pushed in front of all other messages with this priority.}

\end{itemize}

Caution: the priority itself is {\em not copied} by the scheduler.
Therefore, if you pass a pointer to a priority into the scheduler, you
must not overwrite or free that priority until after the message has
emerged from the scheduler's queue.  It is normal to actually store
the priority {\em in the message itself}, though it is up to the user
to actually arrange storage for the priority.
}

\function {void CsdEnqueue(void *Message)}
\index{CsdEnqueue}
\desc{This macro is a shorthand for 
\begin{alltt}
CsdEnqueueGeneral(Message, CQS\_QUEUEING\_FIFO,0, NULL) 
\end{alltt}
provided here for backward compatibility.}

\function {void CsdNodeEnqueue(void *Message)}
\index{CsdNodeEnqueue}
\desc{This macro is a shorthand for 
\begin{alltt}
CsdNodeEnqueueGeneral(Message, CQS\_QUEUEING\_FIFO,0, NULL) 
\end{alltt}
provided here for backward compatibility.}

\function{void CsdEnqueueFifo(void *Message)}
\index{CsdEnqueueFifo}
\desc{This macro is a shorthand for 
\begin{alltt}
CsdEnqueueGeneral(Message, CQS\_QUEUEING\_FIFO,0, NULL)
\end{alltt}
provided here for backward compatibility.}

\function{void CsdNodeEnqueueFifo(void *Message)}
\index{CsdNodeEnqueueFifo}
\desc{This macro is a shorthand for 
\begin{alltt}
CsdNodeEnqueueGeneral(Message, CQS\_QUEUEING\_FIFO,0, NULL)
\end{alltt}
provided here for backward compatibility.}

\function{void CsdEnqueueLifo(void *Message)}
\index{CsdEnqueueLifo}
\desc{This macro is a shorthand for
\begin{alltt}
CsdEnqueueGeneral(Message, CQS\_QUEUEING\_LIFO,0, NULL)
\end{alltt}
provided here for backward compatibility.}

\function{void CsdNodeEnqueueLifo(void *Message)}
\index{CsdNodeEnqueueLifo}
\desc{This macro is a shorthand for
\begin{alltt}
CsdNodeEnqueueGeneral(Message, CQS\_QUEUEING\_LIFO,0, NULL) 
\end{alltt}
provided here for backward compatibility.}

\function{int CsdEmpty()}
\index{CsdEmpty}
\desc{This function returns non-zero integer when the scheduler's 
processor-level queue is empty, zero otherwise.}

\function{int CsdNodeEmpty()}
\index{CsdNodeEmpty}
\desc{This function returns non-zero integer when the scheduler's 
node-level queue is empty, zero otherwise.}

\section{Polling for Messages}
\label{polling}

As we stated earlier, \converse{} messages trigger handler functions when
they arrive.  In fact, for this to work, the processor must
occasionally poll for messages.  When the user starts \converse{}, he can
put it into one of several modes.  In the normal mode, the message
polling happens automatically.  However {\em user-calls-scheduler}
mode is designed to let the user poll manually.  To do this, the user
must use one of two polling functions: \kw{CmiDeliverMsgs}, or
\kw{CsdScheduler}.  \kw{CsdScheduler} is more general, it will notice any
\converse{} event.  \kw{CmiDeliverMsgs} is a lower-level function that ignores
all events except for recently-arrived messages.  (In particular, it
ignores messages in the scheduler queue).  You can save a tiny amount
of overhead by using the lower-level function.  We recommend the use
of \kw{CsdScheduler} for all applications except those that are using only
the lowest level of \converse{}, the CMI.  A third polling function,
\kw{CmiDeliverSpecificMsg}, is used when you know the exact event you want
to wait for: it does not allow any other event to occur.

In each iteration, a scheduler first looks for any message that 
has arrived from another processor, and delivers it.
If there isn't any, it selects a message from the locally enqueued
messages, and delivers it. 

\function {void CsdScheduleForever(void)} \index{CsdScheduleForever}
\desc{Extract and deliver messages until the scheduler is stopped.
Raises the idle handling converse signals.  This is the scheduler to
use in most \converse{} programs.}

\function {int CsdScheduleCount(int n)} \index{CsdScheduleCount}
\desc{Extract and deliver messages until $n$ messages
have been delivered, then return 0. If the scheduler is stopped
early, return $n$ minus the number of messages delivered so far.
Raises the idle handling converse signals.}

\function {void CsdSchedulePoll(void)} \index{CsdSchedulePoll}
\desc{Extract and deliver messages until no more messages
are available, then return.  This is useful for running
non-networking code when the networking code has nothing to do.}

\function {void CsdScheduler(int n)}
\index{CsdScheduler}
\desc{If $n$ is zero, call CsdSchedulePoll.  If $n$ is negative, call
CsdScheduleForever.  If $n$ is positive, call CsdScheduleCount($n$).}


\function{int CmiDeliverMsgs(int MaxMsgs)}
\index{CmiDeliverMsgs}
\desc{Retrieves messages from the network message queue and invokes 
corresponding handler functions for arrived messages. This function 
returns after either the network message queue becomes empty or after
\uw{MaxMsgs} messages have been retrieved and their handlers called. 
It returns the difference between total messages delivered and \uw{MaxMsgs}.
The handler is given a pointer to the message as  its parameter.}

\function{void CmiDeliverSpecificMsg(int HandlerId)}
\index{CmiDeliverSpecificMsg}
\desc{Retrieves messages from the network queue and delivers the first
message with its handler field equal to \uw{HandlerId}. This functions
leaves alone all other messages. It returns after the invoked handler
function returns.}

\function {void CsdExitScheduler(void)}
\index{CsdExitScheduler}
\desc{This call causes CsdScheduler to stop processing messages when
control has returned back to it. The scheduler then returns to its
calling routine.}


\zap{
\section{Global Pointer}

\function{int CmiGptrCreate(GlobalPtr *gptr, void *lptr, unsigned int size)}
\desc{This function creates a global pointer by initializing contents of
\param{*gptr} to point to memory on the local processor pointed to by
\param{lptr} of \param{size} bytes. \param{*gptr} could then be sent to other 
processors, and could be used by \param{CmiGet()} and \param{CmiPut()}
to read and write this memory by remote processors. This functions returns
a positive integer on success.}

\function{void *CmiGptrDref(GlobalPtr *gptr)}
\desc{This function returns the address of local memory associated
with global pointer \param{gptr}.}

\function{int CmiSyncGet(GlobalPtr *gptr, void *lptr, unsigned int size)}
\desc{Copies \param{size} bytes from 
memory pointed to by global pointer \param{gptr}
to local memory pointed to by \param{lptr}. 
This is a synchronous operation and the calling processor blocks until
the data is transferred to local memory. This function returns
a positive integer on success.}

\function{CommHandle CmiGet(GlobalPtr *gptr, void *lptr, unsigned int size)}
\desc{Initiates copying of \param{size} bytes from 
memory pointed to by global pointer \param{gptr}
to local memory pointed to by \param{lptr}. 
This function returns a  communication handle which could be used
to  enquire about the status of this operation.}

\function{CommHandle CmiPut(GlobalPtr *gptr, void *lptr, unsigned int size)}
\desc{Initiates copying of \param{size} bytes from a processor's local
memory pointed to by \param{lptr} to the memory pointed to by global
pointer \param{gptr}.  This function returns a  communication handle
which could be used to  enquire about the status of this operation.}
}

\section{The Timer}

\function{double CmiTimer(void)}
\index{CmiTimer}
\desc{Returns current value of the timer in seconds. This is
typically the time spent since the \kw{ConverseInit} call.
The precision of this timer is the best available on the particular machine,
and usually has at least microsecond accuracy.}

\section{Processor Ids}

\function{int CmiNumPe(void)}
\index{CmiNumPe}
\desc{Returns the total number of processors on which the 
parallel program is being run.}

\function{int CmiMyPe(void)}
\index{CmiMyPe}
\desc{Returns the logical processor identifier of processor on which the 
caller resides. A processor Id is between \texttt{0} and 
\texttt{\kw{CmiNumPe}()-1}.}

Also see the calls in Section~\ref{utility}.

%%%%%%%%%%%%%%%%%%%%%%%%%%%%%%%%%%%%%%%%%%%%%%%%%%%%%%%%%%%%%%%%%%%%%%%%%%%%
% RCS INFORMATION:
%
%       $RCSfile$
%       $Author$        $Locker$                $State$
%       $Revision$      $Date$
%
%%%%%%%%%%%%%%%%%%%%%%%%%%%%%%%%%%%%%%%%%%%%%%%%%%%%%%%%%%%%%%%%%%%%%%%%%%%%
% DESCRIPTION:
%
%%%%%%%%%%%%%%%%%%%%%%%%%%%%%%%%%%%%%%%%%%%%%%%%%%%%%%%%%%%%%%%%%%%%%%%%%%%%
% REVISION HISTORY:
%
% $Log$
% Revision 1.6  1997-06-25 06:53:48  jyelon
% Just trying to make this whole manual make more sense.
%
% Revision 1.5  1995/11/16 22:11:11  milind
% removed hardwired reference.
%
% Revision 1.4  1995/11/01  21:16:23  milind
% Added index entries.
%
% Revision 1.3  1995/10/30  17:33:54  milind
% Changed Cth variables macros to Ctv macros.
% Added "More on Shared Memory Machines" section.
%
% Revision 1.2  1995/10/27  21:02:54  milind
% Integrated into Manual.
%
% Revision 1.1  1995/10/26  21:08:23  gursoy
% Initial revision
%
%%%%%%%%%%%%%%%%%%%%%%%%%%%%%%%%%%%%%%%%%%%%%%%%%%%%%%%%%%%%%%%%%%%%%%%%%%%%

\chapter{Machine Model and Global Variables}

The machine model assumed by Converse consists of a collection of
nodes, where each node comprises of a number of processors that
share memory. In addition, each of the processors could have multiple
threads running on them which share code and data but have different
stacks. On networks of workstations or distributed memory MIMD
machines, number of processors in each node is assumed to be one and on
machines that do not have native thread support, there is only one
thread per processor. Converse provides functions to determine rank of
a processor within a node and a consistent access/modification
mechanism to manipulate global variables.

\section{Global Variables}

Global variables are allowed in Converse via specific macros.  These macros 
ensure that the program runs portably on various parallel machines. 
Three classes of global variables are supported: node-private,
process-private, and thread-private variables. 

\begin{description}
\item[Node-private global variables] are specific to a node. They are
     shared among all the processes within the node.
\item[Process-private global variables]  are specific to a process. They are
     shared among all the threads within the process.
\item[Thread-private global variables] are specific to a thread. They are 
     truely private.
\end{description}

There are five macros for each class. These macros are for
declaration, static declaration, extern declaration, initialization, and
access. The declaration, static and extern specifications have the same
meaning as in C. In order to support portability, however, the global
variables must be installed properly, by using the initialization macros.
For example, if the underlying machine is a simulator for the machine model 
supported by Converse, then the thread-private variables must be turned into 
arrays of variables.  Initialize and Access macros hide these details from 
the user.  It is possible to use global variables without these macros, 
as supported by the underlying machine, but at the expense of portability.


Macros for node-private variables:
\begin{verbatim}
CsvDeclare(type,variable)
CsvStaticDeclare(type,variable)
CsvExtern(type,variable)
CsvInitialize(type,variable)
CsvAccess(variable)
\end{verbatim}

\index{CsvDeclare}
\index{CsvStaticDeclare}
\index{CsvExtern}
\index{CsvInitialize}
\index{CsvAccess}

Macros for process-private variables:
\begin{verbatim}
CpvDeclare(type,variable)
CpvStaticDeclare(type,variable)
CpvExtern(type,variable)
CpvInitialize(type,variable)
CpvAccess(variable)
\end{verbatim}
\index{CpvDeclare}
\index{CpvStaticDeclare}
\index{CpvExtern}
\index{CpvInitialize}
\index{CpvAccess}

Macros for thread-private variables:
\begin{verbatim}
CtvDeclare(type,variable)
CtvStaticDeclare(type,variable)
CtvExtern(type,variable)
CtvInitialize(type,variable)
CtvAccess(variable)
\end{verbatim}
\index{CtvDeclare}
\index{CtvStaticDeclare}
\index{CtvExtern}
\index{CtvInitialize}
\index{CtvAccess}


    A sample code to illustrate the usage of the macros is provided
    in Figure~\ref{fig:cpv}.
    There are a few rules that user must pay attention: The
    {\tt type} and {\tt variable} fields of the macros must be a single
    word. Therefore, structures or pointer types can be used by defining
    new types with the {\tt typedef}. In the sample code, for example,
    a {\tt struct point} type is redefined with a {\tt typedef} as {\tt Point}
    in order to use it in the macros. Similarly,  the access macros contain
    only the name of the global variable. Any indexing or member access
    must be outside of the macro as shown in the sample code 
    (function {\tt func1}). Finally, all the global variables must be
    installed before they are used. One way to do this systematically is
    to provide a module-init function for each file (in the sample code - 
    {\tt ModuleInit()}. The module-init functions of each file, then, can be 
    called at the beginning of execution to complete the installations of 
    all global variables.
    
\begin{figure}
\begin{verbatim}
File Module1.c

    typedef struct point
    {
         float x,y;
    } Point;


    CpvDeclare(int, a);
    CpvDeclare(Point, p);

    void ModuleInit()
    {
         CpvInitialize(int, a)
         CpvInitialize(Point, p);

         CpvAccess(a) = 0;
    }

    int func1() 
    {
         CpvAccess(p).x = 0;
         CpvAccess(p).y = CpvAccess(p).x + 1;
    }
\end{verbatim}
\caption{An example code for global variable usage}
\label{fig:cpv}
\end{figure}

\section{Utility Functions}
\label{utility}

To further simplify programming with global variables on shared memory machines,
Converse provides following functions and/or macros. \note{These functions
are defined on machines other than shared-memory machines also, and
have the effect of only one processor per node and only one thread per 
processor.}

\function{int CmiMyRank()}
\index{CmiMyRank}
\desc{Returns the rank of the calling processor within a shared memory node.}

\function{void CmiNodeBarrier()}
\index{CmiNodeBarrier}
\desc{Provide barrier synchronization at the node level, i.e. all the 
processors belonging to the node participate in this barrier.}

\function{void *CmiSvAlloc(int size)}
\index{CmiSvAlloc}
\desc{Allocates a block of memory of \param{size} bytes from the heap in node's 
local memory and returns pointer to the start of this block. \note{On 
machines other than shared-memory machines, this is equivalent to 
\param{CmiAlloc}.}}


\section{Input/Output}

\function{void CmiPrintf(char *format, arg1, arg2, ...)}
\index{CmiPrintf}
\desc{This function does an atomic \texttt{printf()} on \texttt{stdout}. 
On machine with host, this is implemented on top of the messaging 
layer using asynchronous sends.}

\function{int CmiScanf(char *format, void *arg1, void *arg2, ...)}
\index{CmiScanf}
\desc{This function performs an atomic \texttt{scanf} from \texttt{stdin}.
The processor, on which the caller resides, blocks for input. On machines with
host, this is implemented on top of the messaging layer using asynchronous
send and blocking receive.}

\function{void CmiError(char *format, arg1, arg2, ...)}
\index{CmiError}
\desc{This function does an atomic \texttt{printf()} on \texttt{stderr}. 
On machines with host, this is implemented on top of the messaging 
layer using asynchronous sends.}

\zap{
\section{Processor Groups}

\function{void CmiPgrpCreate(Pgrp *group)}
\desc{Creates a processor-group with calling processsor as the root processor.}

\function{void CmiPgrpDestroy(Pgrp *group)}
\desc{Frees resources associated with a processor group \param{group}.}

\function{void CmiAddChildren(Pgrp *group, int penum, int size, int procs[])}
\desc{Adds \param{size} processors from array \param{procs[]} to the
processor-group \param{group} as children of processor penum. This function
could be called only by the root processor of processor-group \param{group}.}

\function{CommHandle CmiAsyncMulticast(Pgrp *group, unsigned int size, void *msg)}
\desc{Initiates asynchronous broadcast of message \param{msg} of
length \param{size} bytes to all processors belonging to \param{group}
excluding the processor on which the caller resides. It returns a
communication handle which could be used to check the status of this
send using \param{CmiAsyncMsgSent()}. If the returned communication handle 
is 0, message buffer can be reused immediately, thus saving a call to 
CmiAsyncMsgSent. \param{msg} should not be
overwritten or freed before the communication is complete. \note{Caller
need not belong to \param{group}.}} 

\function{int CmiPgrpRoot(Pgrp *group)}
\desc{Returns the processor id of root of processor-group \param{group}. }

\function{int CmiNumChildren(Pgrp *group, int penum)}
\desc{Returns  number of children of processor \param{penum} 
in the processor-group \param{group}.}

\function{int CmiParent(Pgrp *group, int penum)}
\desc{Returns  processor id of parent of processor \param{penum} 
in the processor-group \param{group}.}

\function{void CmiChildren(Pgrp *group, int node, int *children)}
\desc{Fills in array \param{children} with processor ids of all the
children processor \param{node} in processor-group \param{group}. This
array should at least be of size \param{CmiNumChildren()}.}
}

\section{Spanning Tree Calls}

Sometimes, it is convenient to view the processors/nodes of the machine as a
tree.  For this purpose, \converse{} defines a tree over processors/nodes.  We
provide functions to obtain the parent and children of each processor/node.  On
those machines where the communication topology is relevant, we
arrange the tree to optimize communication performance. The root of
the spanning tree (processor based or node-based) is always 0, thus
the \kw{CmiSpanTreeRoot} call has been eliminated.

\function{int CmiSpanTreeParent(int procNum)}
\index{CmiSpanTreeParent}
\desc{This function returns the processor number of the parent of
\uw{procNum} in the spanning tree.}

\function{int CmiNumSpanTreeChildren(int procNum)}
\index{CmiNumSpanTreeChildren}
\desc{Returns the number of children of \uw{procNum} in the spanning tree.}

\function{void CmiSpanTreeChildren(int procNum, int *children)}
\index{CmiSpanTreeChildren}
\desc{This function fills the array \uw{children} with processor
numbers of children of \uw{procNum} in the spanning tree.}

\function{int CmiNodeSpanTreeParent(int nodeNum)}
\index{CmiNodeSpanTreeParent}
\desc{This function returns the node number of the parent of
\uw{nodeNum} in the spanning tree.}

\function{int CmiNumNodeSpanTreeChildren(int nodeNum)}
\index{CmiNumNodeSpanTreeChildren}
\desc{Returns the number of children of \uw{nodeNum} in the spanning tree.}

\function{void CmiNodeSpanTreeChildren(int nodeNum, int *children)}
\index{CmiNodeSpanTreeChildren}
\desc{This function fills the array \uw{children} with node
numbers of children of \uw{nodeNum} in the spanning tree.}





