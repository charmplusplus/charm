\chapter{Initialization and Completion}


\function {void ConverseInit(void)}
This call initializes all Converse components, such as the scheduler,
machine interface, and other libraries. This must be the first Converse
call in the entire program.


\function {void ConverseExit(void)}
This call wraps up all Converse components. No Converse call may be made
after this call.




\chapter{Scheduler Calls}

%These are the calls/macros provided by the Converse scheduler.
%All declarations are in converse.h.

\internal{
\function {void CsdInit(void)}
This call initializes the Converse scheduler, it is called from
ConverseInit().
}

\function {void CsdScheduler(int NumberOfMessages)}
This call invokes the Converse scheduler. The {\tt NumberOfMessages}
parameter specifies how many messages should be processed (i.e. delivered
to their handlers). If set to -1, the scheduler continues processing
messages until CsdExitScheduler() is called from a message handler.

\function {void CsdExitScheduler(void)}
This call causes the scheduler
to stop processing messages when control has returned back to it.
The scheduler then returns to its calling routine.

\function {void CsdEnqueue(void *Message)}
This call enqueues a message in the scheduler's queue, to be processed
in accordance with the queueing strategy. This call is usually made from
a message handler when the message is not to be processed immediately,
but may be processed later (e.g. depending on the message's priority).
Also, it is used to enqueue local ready entities, such as threads.
