\chapter{Initialization and Completion}

The program utilizing Converse begins executing at {\tt main}, like
any other C program.  The initialization process is somewhat
complicated by the fact that vendors don't agree about which
processors should execute {\tt main}.  On some machines, every processor
executes {\tt main}.  On others, only one processor executes {\tt
main}.  All processors which don't execute {\tt main} are asleep when
the program begins.  The function ConverseInit is used to start the
converse system, and to wake up the sleeping processors:

\function{typedef void (*CmiStartFn)(int argc, char **argv);}
\function{void ConverseInit(int argc, char *argv[], CmiStartFn fn, int usched, int initret)}
\index{ConverseInit}
\desc{This function starts up the Converse system.  It can execute
in one of several modes, described below.

Normal Mode: {\tt usched=0, initret=0}

When you run your program, some of the processors automatically invoke
{\tt main}, others remain asleep.  All processors which automatically
invoked {\tt main} must must call ConverseInit.  This initializes the
entire Converse system.  Converse then initiates, on {\em all}
processors, the execution of the user-supplied start-function {\tt
fn(argc, argv)} followed by the Converse scheduler.  Once the
scheduler exits on all processors, the Converse system shuts down, and
your program terminates.  Note that in this case, ConverseInit never
returns.  The user is not allowed to call the Converse scheduler
manually.

ConverseInit-returns Mode: {\tt initret=1}

This option is used when you want ConverseInit to return.  All
processors which automatically invoked {\tt main} must call
ConverseInit.  This initializes the entire Converse System.  On all
processors which {\em did not} automatically invoke {\tt main}, Converse
initiates the user-supplied initialization function {\tt fn(argc,
argv)} followed by the Converse scheduler.  Meanwhile, on those
processors which {\em did} automatically invoke {\tt main}, ConverseInit
returns.  Shutdown is initiated when the processors that {\em did}
automatically invoke {\tt main} call ConverseExit, and when the other
processors exit the scheduler.  This option is not supported
by the sim version.

User-calls-scheduler Mode: {\tt usched=1}

This is how the user initializes Converse if he wants to perform all
the scheduling manually.  In normal mode, it is assumed that the
user-supplied start-function {\tt fn(argc, argv)} is just for
initialization, and that the remainder of the lifespan of the program
is spent in the (automatically-invoked) Converse scheduler.  In
user-calls-scheduler mode, however, it is assumed that the
user-supplied start-function will perform the {\em entire
computation}, including scheduling.  Thus, ConverseInit will not
automatically execute the scheduler for you.  When the user-supplied
start-function ends, Converse shuts down.  This mode is not supported
on the sim version.  This mode can be combined with ConverseInit
returns mode.}

\function{void ConverseExit(void)}
\index{ConverseExit}
\desc{This function is only used in ConverseInit returns mode, see
above.}

