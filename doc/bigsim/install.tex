\section{Blue Gene Simulator Installation and Usage}
\label{install}

\subsection{Installing Charm++ and Blue Gene}

Blue Gene Simulator now is integrated into Charm++ distribution as a runtime 
library. Unfortunately, the precompiled binary package distribution doesnot 
contain the Blue Gene Simulator, thus you need to download source code
and compile yourself. 

You begin by downloading Charm++ from our website:
http://charm.cs.uiuc.edu/beta.html

Please refer to "Charm++ Installation and Usage Manual" and also the README
in the source code for detailed instruction on how to compile Charm++.
In short, the "build" script is the main tool for compiling \charmpp{}.
You need to provide target and platform options:
\begin{verbatim}
./build <target> <platform> [options ...] [charmc-options ...]
\end{verbatim}

For example, to compile on a Linux machine, type:
\begin{verbatim}
./build charm++ net-linux
\end{verbatim}

which builds essential \charmpp{} kernel using UDP as communication method, 
alternatively, you can build Charm++ kernel on MPI:
\begin{verbatim}
./build charm++ mpi-linux
\end{verbatim}

For other platforms, change net-linux to whatever platform you are compiling 
on. See the charm/README file for a complete list of supported platforms.

However, those commands donot automatically compile for Blue Gene Simulator, 
thus in order to compile both \charmpp{} kernel and Blue Gene Simulator, 
you need to provide a special target "bluegene" and an extra option for 
"build" script, for example:
\begin{verbatim}
./build bluegene net-linux bluegene
\end{verbatim}

The first "bluegene" is the compilation target, it tells "build" to
compile Blue Gene Simulator libraries as well as \charmpp{} kernel;
The second "bluegene" is an option to platform "net-linux", which tells
"build" to compile the Charm++ kernel itself upon Blue Gene Emulator. 
To compile AMPI on Blue Gene, use "bgampi" as target, which subsumes target
"bluegene":
\begin{verbatim}
./build bgampi net-linux bluegene
\end{verbatim}

For the above "build" command, it creates a directory named 
"net-linux-bluegene" under charm, which contains all the header files and
libraries needed for compiling a user application.

\subsection{Compiling Blue Gene Applications}

\charmpp{} provides a compiler script {\tt charmc} to compile all programs.

There are three methods to write a Blue Gene applicaiton:

\subsubsection{Writing a Blue Gene application using low level machine API}
The low level machine API mimics the actual machine low level programming
API. It is defined in section~\ref{bgemulator}. Writing a program in the 
low level machine API, you just need to link \charmpp{}'s Blue Gene emulator
library, which provides the emulation of the machine API using Converse as
the communication layer.

In order to link against the Blue Gene library, specify 
\texttt{-language bluegene} as an argument to the {\tt charmc} linker, 
for example:
\begin{verbatim}
charmc -o hello hello.C -language bluegene
\end{verbatim}

Sample applications in low level machine API can be found under directory
charm/pgms/converse/bluegene.

\subsubsection{Writing a Blue Gene application using Charm++}

One can write a normal \charmpp{} application which can automatically 
run on the emulator after compilation. \charmpp{} implements
an object-based message-driven execution model. In \charmpp{} applications,
there are collections of C++ objects, which communicate by remotely invoking
methods on other objects by messages.

In order to compile a program written in \charmpp{} on Blue Gene simulator, 
specify \texttt{-language charm++} as an argument to the {\tt charmc} linker:
\begin{verbatim}
charmc -o hello hello.C -language charm++
\end{verbatim}
This will link both \charmpp{} runtime libraries and Blue Gene simulator 
library.

Sample applications in \charmpp{} can be found under directory
charm/pgms/charm++, specifically charm/pgms/charm++/littleMD.

\subsubsection{Writing a Blue Gene application using MPI}

One can also write a MPI application for Blue Gene Simulator.
The Adaptive MPI, or AMPI is implemented on top of Charm++ that supports
dynamic load balancing and multithreading for MPI applications. This is based
on the user-level migrating threads and load balancing capabilities provided
by the \charmpp{} framework. This allows legacy MPI program to run 
on top of Blue Gene \charmpp{} and take advantage of the \charmpp{}'s
virtualization and adaptive load balancing capability.

Current AMPI implements most features in the MPI version 1.0, with a few
extensions for migrating threads and asynchronous reduction.

In order to compile an AMPI application on Blue Gene simulator, you need 
to link against the AMPI library as well as Blue Gene \charmpp{} runtime
libraries by specifying \texttt{-language ampi} as an argument to 
the {\tt charmc} linker:
\begin{verbatim}
charmc -o hello hello.C -language ampi
\end{verbatim}

Sample applications in AMPI can be found under directory
charm/pgms/charm++/ampi, specifically charm/pgms/charm++/Cjacobi3D.

\subsection{Running a Blue Gene Application}

To run a parallel Blue Gene application, \charmpp{} provides a utility program
called {\tt charmrun} to start the parallel program. 
For detailed description on how to run a \charmpp{} application, 
refer to file charm/README in the source code distribution.

To run a Blue Gene application, you need to specify these parameters to 
{\tt charmrun} to define the simulated Blue Gene machine size:
\begin{enumerate}
\item {\tt +x, +y} and {\tt +z}:  define the size of of machine in three dimensions, these define the number of nodes along each dimension of the machine;
\item {\tt +wth} and {\tt +cth}:  For one node, these two parameters define the number of worker processors({\tt +wth}) and the number of communication processors({\tt +cth}).
\end{enumerate}

For example, to simulate a Blue Gene/L machine of size 64K in 40x40x40, with 
one worker processor and one I/O processor on each node, and use 100 
real processors to simulate:
\begin{verbatim}
./charmrun +p100 ./hello +x40 +y40 +z40 +cth1 +wth1
\end{verbatim}


