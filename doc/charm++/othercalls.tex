\subsection{Other Calls}

\label{other Charm++ calls}

The following calls provide information about the machines upon which the
parallel program is executing.  Processing Element refers to a single CPU.
Node refers to a single machine-- a set of processing elements which share
memory (i.e. an address space).  Processing Elements and Nodes are numbered,
starting from zero.

Thus if a parallel program is executing on one 4-processor workstation and one
2-processor workstation, there would be 6 processing elements (0, 1 ,2, 3, 4,
and 5) but only 2 nodes (0 and 1).  A given node's processing elements are
numbered sequentially.

{\bf int CkNumPes()} \index{CkNumPes} \\
returns the total number of processors, across all nodes.

{\bf int CkMyPe()} \index{CkMyPe} \\
returns the processor number on which the call was made.

{\bf int CkMyRank()} \index{CkMyRank} \\
returns the rank number of the processor on which the call was made.
Processing elements within a node are ranked starting from zero.

{\bf int CkMyNode()} \index{CkMyNode} \\
returns the address space number (node number) on which the call was made.

{\bf int CkNumNodes()} \index{CkMyNodes} \\
returns the total number of address spaces.

{\bf int CkNodeFirst(int node)} \index{CkNodeFirst} \\
returns the processor number of the first processor in this address space.

{\bf int CkNodeSize(int node)} \index{CkNodeSize} \\
returns the number of processors in the address space on which the call was made.

{\bf int CkNodeOf(int pe)} \index{CkNodeOf} \\
returns the node number on which the call was made.

{\bf int CkRankOf(int pe)} \index{CkRankOf} \\
returns the rank of the given processor within its node.

The following calls provide commonly needed functions.

{\bf void CkAbort(const char *message)} \index{CkAbort} \\
Cause the program to abort, printing the given error message.

{\bf void CkExit()} \index{CkExit} \\
This call informs the Charm kernel that computation on all processors should 
terminate.  After the currently executing entry method completes, no more 
messages or entry methods will be called.  \keyword{CkExit()} should be the 
last statement of the entry method from which it was called. 

{\bf double CkTimer()} \index{CkTimer} \index{timers} \\
returns the current value of the system timer in milliseconds. The system
timer is started when \\
the program begins execution. This timer measures
process time (user and system).

{\bf double CkWallTimer()} \index{CkWallTimer} \index{timers} \\
returns the elapsed time since the program has started from the wall clock 
timer.

