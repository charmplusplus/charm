\section{History}

The {\sc Charm} software was developed as a group effort of the Parallel
Programming Laboratory at the University of Illinois at Urbana-Champaign.
Researchers at the Parallel Programming Laboratory keep \charmpp\ updated for
the new machines, new programming paradigms, and for supporting and simplifying
development of emerging applications for parallel processing.  The earliest
prototype, Chare Kernel(1.0), was developed in the late eighties. It consisted
only of basic remote method invocation constructs available as a library.  The
second prototype, Chare Kernel(2.0), a complete re-write with major design
changes.  This included C language extensions to denote Chares, messages and
asynchronous remote method invocation.  {\sc Charm}(3.0) improved on this
syntax, and contained important features such as information sharing
abstractions, and chare groups (called Branch Office Chares).  {\sc Charm}(4.0)
included \charmpp\ and was released in fall 1993.  \charmpp\ in its initial
version consisted of syntactic changes to \CC\ and employed a special
translator that parsed the entire \CC\ code while translating the syntactic
extensions.  {\sc Charm}(4.5)  had a major change that resulted from a
significant shift in the research agenda of the Parallel Programming
Laboratory. The message-driven runtime system code of the \charmpp\ was
separated from the actual language implementation, resulting in an
interoperable parallel runtime system called {\sc
Converse}. The \charmpp\ runtime system was
retargetted on top of {\sc Converse}, and popular programming paradigms such as
MPI and PVM were also implemented on {\sc Converse}. This allowed
interoperability between these paradigms and \charmpp. This release also
eliminated the full-fledged \charmpp\ translator by replacing syntactic
extensions to \CC\ with \CC\ macros, and instead contained a small language and
a translator for describing the interfaces of \charmpp\ entities to the runtime
system.  This version of \charmpp, which, in earlier releases was known as {\em
Interface Translator \charmpp}, is the default version of \charmpp\ now, and
hence referred simply as {\bf \charmpp}.  In early 1999, the runtime system of
\charmpp\ 
%was formally named the Charm Kernel, and 
was rewritten in \CC.
Several new features were added. The interface language underwent significant
changes, and the macros that replaced the syntactic extensions in original
\charmpp, were replaced by natural \CC\ constructs. Late 1999, and early
2000 reflected several additions to \charmpp{}, when a load balancing
framework and migratable objects were added to \charmpp{}.

\section {Acknowledgements}

The Charm software was developed as a
group effort.  The earliest prototype, Chare Kernel(1.0), was
developed by Wennie Shu and Kevin Nomura working with Laxmikant
Kale.  The second prototype, Chare Kernel(2.0), a complete
re-write with major design changes, was developed by a team
consisting of Wayne Fenton, Balkrishna Ramkumar, Vikram Saletore,
Amitabh B. Sinha and Laxmikant Kale. The translator for Chare
Kernel(2.0) was written by Manish Gupta.  Charm(3.0), with
significant design changes, was developed by a team consisting of
Attila Gursoy, Balkrishna Ramkumar, Amitabh B.  Sinha and
Laxmikant Kale, with a new translator written by Nimish Shah.  The
\charmpp\ implementation was done by Sanjeev Krishnan.  Charm(4.0)
included \charmpp\ and was released in fall 1993.  Charm(4.5) was
developed by Attila Gursoy, Sanjeev Krishnan, Milind Bhandarkar,
Joshua Yelon, Narain Jagathesan and Laxmikant Kale.  Charm(4.8),
developed by the same team included Converse, a parallel runtime
system that allows interoperability among modules written using
different paradigms within a single application. \charmpp\ runtime
system was re-targetted at Converse. Syntactic extensions in
\charmpp\ were dropped, and a simple interface translator was
developed (by Sanjeev Krishnan and Jay DeSouza) that, along with
the \charmpp\ runtime, became the \charmpp\ language.  Charm
(5.4R1) included the following: a complete rewrite of the
\charmpp\ runtime system (using \CC) and the interface translator
(done by Milind Bhandarkar), several new features such as Chare
Arrays (developed by Robert Brunner and Orion Lawlor), various
libraries (written by Terry Wilmarth, Gengbin Zheng, Laxmikant
Kale, Zehra Sura, Milind Bhandarkar, Robert Brunner, and Krishnan
Varadarajan.) A coordination language ``Structured Dagger'' was
been implemented on top of \charmpp\ (Milind Bhandarkar), dynamic
seed-based load balancing (Terry Wilmarth and Joshua Yelon), a
client-server interface for Converse programs, and debugging
support by Parthasarathy Ramachandran, Jeff Wright, and Milind
Bhandarkar, Projections, the performance visualization and
analysis tool, was redesigned and rewritten using Java by Michael
Denardo. The test suite for \charmpp\ was developed by Michael
Lang, Jackie Wang, and Fang Hu. Converse was been ported to ASCI
Red (Joshua Yelon), Cray T3E (Robert Brunner), and SGI Origin2000
(Milind Bhandarkar). For the current version Charm 6.0 (R1),
Converse has been ported to new platforms including BlueGene/[LP]
(Kumar, Huang, Bhatele), Cray XT3/4 (Zheng), Apple G5, Myrinet
(Zheng), and Infiniband (Chakravorty).  Charm 6.0 introduces a
dedicated no network SMP multicore Converse layer for stand-alone
workstation experimenters (Zheng, Chakravorty, Kale, Jetley).
Charm 6.0 also includes cross platform network topology aware
chare placement for 3D tori and mesh networks (Kumar, Huang,
Bhatele, Bohm). The test suite was extended for automated testing
on all supported platforms by Gengbin Zheng.  The Projection tool
was substantially improved by Chee Wai Lee and Isaac Dooley. The
Control Point performance tuning framework was created by Isaac
Dooley. Debugging support was enhanced with memory inspection
features by Filippo Gioachin. The Charisma orchestration language
was implemented on top of Charm++ by Chao Huang and Sanjay Kale.
Sanjay Kale, Orion Lawlor, Gengbin Zheng, Terry Wilmarth, Filippo
Gioachin, Sayantan Chakravorty, Chao Huang, David Kunzman, Isaac
Dooley, Eric Bohm, Sameer Kumar, Chao Mei, Pritish Jetley, and
Abhinav Bhatele, have been responsible for the changes to the
system since the last release. 
