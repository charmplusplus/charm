To better utilize the multicore chip, it has become increasingly popular to
adopt shared-memory multithreading programming methods to exploit parallelism
on a node. For example, in hybrid MPI programs, OpenMP is the most popular
choice.  When launching such hybrid programs, users have to make sure there are
spare physical cores allocated to the shared-memory mulithreading runtime.
Otherwise, the runtime that handles distributed-memory programming may
interfere with resource contention because the two independent runtime systems
are not coordinated.  If spare cores are allocated, in the same way of
launching a MPI+OpenMP hybrid program, \charmpp{} will work perfectly with any
shared-memory parallel programming languages (e.g. OpenMP). As with ordinary
OpenMP applications, the number of threads used in the OpenMP parts of the
program can be controlled with the {\tt OMP\_NUM\_THREADS} environment
variable.  See Sec.~\ref{charmrun} for details on how to propagate such
environment variables.

If there are no spare cores allocated, to avoid resouce contention, a
\emph{unified runtime} is needed to support both intra-node shared-memory
multithreading parallelism and inter-node distributed-memory
message-passing parallelism. Additionally, considering that a parallel
application may have only a small fraction of its critical computation be
suitable for porting to shared-memory parallelism (the savings on critical
computation may also reduce the communication cost, thus leading to more
performance improvement), dedicating physical cores on every node to the
shared-memory multithreading runtime will waste computational power because
those dedicated cores are not utilized at all during most of the application's
execution time. This case indicates the necessity of a unified
runtime supporting both types of parallelism.

The \emph{CkLoop} library is an add-on to the \charmpp{} runtime to achieve such
a unified runtime.  The library implements a simple OpenMP-like shared-memory
multithreading runtime that reuses \charmpp{} PEs to perform tasks spawned by
the multithreading runtime. This library targets the SMP mode of \charmpp{}.

The \emph{CkLoop} library is built in
\$CHARM\_DIR/\$MACH\_LAYER/tmp/libs/ck-libs/ckloop by executing ``make''.
To use it for user applications, one has to include ``CkLoopAPI.h'' in
the source code. The interface functions of this library are as
follows:

\begin{itemize}
\item CProxy\_FuncCkLoop \textbf{CkLoop\_Init}(int
numThreads=0): This function initializes the CkLoop library, and it only needs
to be called once on a single PE during the initilization phase of the
application.  The argument ``numThreads'' is only used in non-SMP mode,
specifying the number of threads to be created for the single-node shared-memory
parallelism. It will be ignored in SMP mode.

\item void \textbf{CkLoop\_Exit}(CProxy\_FuncCkLoop ckLoop): This function is
intended to be used in non-SMP mode, as it frees the resources
(e.g. terminating the spawned threads) used by the CkLoop library. It should
be called on just one PE.

\item void \textbf{CkLoop\_Parallelize}( \\
HelperFn func, /* the function that finishes partial work on another thread */ \\
int paramNum, /* the number of parameters for func */\\
void * param, /* the input parameters for the above func */ \\
int numChunks, /* number of chunks to be partitioned */ \\
int lowerRange, /* lower range of the loop-like parallelization [lowerRange, upperRange] */ \\
int upperRange, /* upper range of the loop-like parallelization [lowerRange, upperRange] */ \\
int sync=1, /* toggle implicit barrier after each parallelized loop */ \\
void *redResult=NULL, /* the reduction result, ONLY SUPPORT SINGLE VAR of TYPE int/float/double */ \\
REDUCTION\_TYPE type=CKLOOP\_NONE /* type of the reduction result */ \\
) \\
The ``HelperFn'' is defined as ``typedef void (*HelperFn)(int first,int last, void *result, int paramNum, void *param);'' The ``result'' is the buffer for reduction result on a single simple-type variable.
\end{itemize}

Examples using this library can be found in \examplerefdir{ckloop} and the
widely used molecular dynamics simulation application
NAMD\footnote{http://www.ks.uiuc.edu/Research/namd}.



