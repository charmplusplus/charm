\charmpp\ supports C++ like inheritance among \charmpp\ objects such as
chares, groups, and messages, making it easier to keep applications
modular and allowing reuse of code.

\section{Chare Inheritance}

\index{inheritance}

Chare inheritance makes it possible to remotely invoke methods of a base
chare \index{base chare} from a proxy of a derived
chare.\index{derived chare} Suppose a base chare is of type 
\uw{BaseChare}, then the derived chare of type \uw{DerivedChare} needs to be
declared in the \charmpp\ interface file to be explicitly derived from
\uw{BaseChare}. Thus, the constructs in the \texttt{.ci} file should look like:

\begin{alltt}
  chare BaseChare \{
    entry BaseChare(someMessage *);
    entry void baseMethod(void);
    ...
  \}
  chare DerivedChare : BaseChare \{
    entry DerivedChare(otherMessage *);
    entry void derivedMethod(void);
    ...
  \}
\end{alltt}

Note that the access specifier \kw{public} is omitted, because \charmpp\
interface translator only needs to know about the public inheritance,
and thus \kw{public} is implicit. A Chare can inherit privately from other
classes too, but the \charmpp\ interface translator does not need to know
about it, because it generates support classes ({\em proxies}) to remotely
invoke only \kw{public} methods.

The class definitions of these chares should look like:

\begin{alltt}
  class BaseChare : public CBase\_BaseChare \{
    // private or protected data
    public:
      BaseChare(someMessage *);
      void baseMethod(void);
  \};
  class DerivedChare : public CBase\_DerivedChare \{
    // private or protected data
    public:
      DerivedChare(otherMessage *);
      void derivedMethod(void);
  \};
\end{alltt}

It is possible to create a derived chare, and invoke methods of base
chare from it, or to assign a derived chare proxy to a base chare proxy
as shown below:

\begin{alltt}
  ...
  otherMessage *msg = new otherMessage();
  CProxy_DerivedChare pd = CProxy_DerivedChare::ckNew(msg);
  pd.baseMethod();     // OK
  pd.derivedMethod();  // OK
  ...
  CProxy_BaseChare pb = pd;
  pb.baseMethod();    // OK
  pb.derivedMethod(); // COMPILE ERROR
\end{alltt}

To pass constructor arguments
from \uw{DerivedChare::DerivedChare(someMessage*)}
to \uw{BaseChare::BaseChare(someMessage*)}, they can be forwarded
through the \uw{CBase} type constructor as follows:

\begin{alltt}
DerivedChare::DerivedChare(someMessage *msg)
: CBase_DerivedChare(msg) // Will forward all arguments to BaseChare::BaseChare
{ }
\end{alltt}

If no arguments are provided, the generated \CC\ code for
the \uw{CBase\_DerivedChare} constructor calls the default
constructor \index{default constructor} of the base
class \uw{BaseChare}.

%Multiple inheritance \index{multiple inheritance} is also allowed for Chares
%and Groups. Often, one should make each of the base classes inherit
%``virtually'' from \kw{Chare} or \kw{Group}, so that a single copy of
%\kw{Chare} or \kw{Group} exists for each multiply derived class.

Entry methods are inherited in the same manner as methods of
sequential \CC{} objects.  To make an entry method virtual, just add
the keyword \kw{virtual} to the corresponding chare method declaration
in the class header-- no change is needed in the interface file.  Pure
virtual entry methods also require no special description in the
interface file.


\section{Inheritance for Messages}

\index{message inheritance}

Messages cannot inherit from other messages.  A message can, however,
inherit from a regular \CC\ class.  For example:

\begin{alltt}
//In the .ci file:
  message BaseMessage1;
  message BaseMessage2;

//In the .h file:
  class Base \{
    // ...
  \};
  class BaseMessage1 : public Base, public CMessage_BaseMessage1 \{
    // ...
  \};
  class BaseMessage2 : public Base, public CMessage_BaseMessage2 \{
    // ...
  \};
\end{alltt}

Messages cannot contain virtual methods
or virtual base classes unless you use a packed message.
Parameter marshalling has complete support for inheritance, virtual
methods, and virtual base classes via the PUP::able framework.


% ( I think the following is now a lie  OSL 7/5/2001 )  
%Similar to Chares, messages can also be derived from base messages. One needs
%to specify this in the \charmpp\ interface file similar to the Chare
%inheritance specification (that is, without the \kw{public} access specifier.)
%Message inheritance makes it possible to send a message of derived type to the
%method expecting a base class message.
