\section{Controlling Delivery Order}

By default, \charmpp\ processes the messages sent in roughly FIFO\index{message
  delivery order} order when they arrive at a PE.  For most programs, this
behavior is fine. However, for optimal performance, some programs need more
explicit control over the order in which messages are processed. \charmpp\
allows you to adjust delivery order on a per-message basis.

An example program demonstrating how to modify delivery order for messages and
parameter marshaling can be found in \examplerefdir{prio}.

\subsubsection{Queueing Strategies}
\label{queueing strategies}

The order in which messages are processed in the recipient's queue can be set
by explicitly setting the queuing strategy using one the following
constants. These constants can be applied when sending a message or invoking an
entry method using parameter marshaling:

\begin{itemize}
\item \texttt{CK\_QUEUEING\_FIFO}: FIFO ordering
\item \texttt{CK\_QUEUEING\_LIFO}: LIFO ordering
\item \texttt{CK\_QUEUEING\_IFIFO}: FIFO ordering with \emph{integer} priority
\item \texttt{CK\_QUEUEING\_ILIFO}: LIFO ordering with \emph{integer} priority
\item \texttt{CK\_QUEUEING\_BFIFO}: FIFO ordering with \emph{bitvector} priority
\item \texttt{CK\_QUEUEING\_BLIFO}: LIFO ordering with \emph{bitvector} priority
\item \texttt{CK\_QUEUEING\_LFIFO}: FIFO ordering with \emph{long integer} priority
\item \texttt{CK\_QUEUEING\_LLIFO}: FIFO ordering with \emph{long integer} priority
\end{itemize}

\subsubsection{Parameter Marshaling}

For parameter marshaling, the \kw{queueingtype} can be set for
\kw{CkEntryOptions}, which is passed to an entry method invocation as the
optional last parameter.

\begin{alltt}
  CkEntryOptions opts1, opts2;
  opts1.setQueueing(CK_QUEUEING_FIFO);
  opts2.setQueueing(CK_QUEUEING_LIFO);

  chare.entry_name(arg1, arg2, opts1);
  chare.entry_name(arg1, arg2, opts2);
\end{alltt}

When the message with \kw{opts1} arrives at its destination, it will be pushed onto the
end of the message queue as usual.  However, when the message with \kw{opts2} arrives, it will be
pushed onto the \emph{front} of the message queue.

\subsubsection{Messages}

For messages, the \kw{CkSetQueueing} function can be used to change the order
in which messages are processed, where \kw{queueingtype} is one of the above
constants.\\

\function{void CkSetQueueing(MsgType message, int queueingtype)}

\noindent The first two options,  \kw{CK\_QUEUEING\_FIFO} and
\kw{CK\_QUEUEING\_LIFO}, are used as follows:
%
\begin{alltt}
  MsgType *msg1 = new MsgType ;
  CkSetQueueing(msg1, CK_QUEUEING_FIFO);

  MsgType *msg2 = new MsgType ;
  CkSetQueueing(msg2, CK_QUEUEING_LIFO);
\end{alltt}

Similar to the parameter marshalled case described above, \kw{msg1}
will be pushed onto the end of the message queue, while \kw{msg2} will
be pushed onto the \emph{front} of the message queue.

\subsubsection{Prioritized Execution}
\label{prioritized message passing}
\index{prioritized execution}
\index{prioritized message passing}
\index{priorities}

The basic FIFO and LIFO strategies are sufficient to approximate parallel
breadth-first and depth-first explorations of a problem space, but they do not
allow more fine-grained control. To provide that degree of control, \charmpp\
also allows explicit prioritization of messages.

The other six queueing strategies involve the use of
priorities\index{priorities}.  There are two kinds of priorities which can be
attached to a message: \emph{integer priorities}\index{integer priorities} and
\emph{bitvector priorities}\index{bitvector priorities}. These correspond to
the \emph{I} and \emph{B} queueing strategies, respectively. In both cases,
numerically lower priorities will be dequeued and delivered before numerically
greater priorities. The FIFO and LIFO queueing strategies then control the
relative order in which messages of the same priority will be delivered.

To attach a priority field to a message, one needs to set aside space in the
message's buffer while allocating the message\index{message priority}.  To
achieve this, the size of the priority field\index{priority field} in bits
should be specified as a placement argument to the \kw{new} operator, as
described in section~\ref{memory allocation}.  Although the size of the
priority field is specified in bits, it is always padded to an integral number
of {\tt int}s.  A pointer to the priority part of the message buffer can be
obtained with this call:\\

\function{void *CkPriorityPtr(MsgType msg)}
\index{CkPriorityPtr}
\index{priority pointer}

Integer priorities are quite straightforward.  One allocates a message
with an extra integer parameter to ``new'' (see the first line of the
example below), which sets aside enough space (in bits) in the message
to hold the priority.  One then stores the priority in the message.
Finally, one informs the system that the message contains an integer
priority using \kw{CkSetQueueing}:

\begin{alltt}
  MsgType *msg = new (8*sizeof(int)) MsgType;
  *(int*)CkPriorityPtr(msg) = prio;
  CkSetQueueing(msg, CK_QUEUEING_IFIFO);
\end{alltt}

\subsubsection{Bitvector Prioritization}

Bitvector priorities are arbitrary-length bit-strings representing fixed-point
numbers in the range 0 to 1.  For example, the bit-string ``001001'' represents
the number .001001\raisebox{-.5ex}{\scriptsize binary}.  As with integer
priorities, higher numbers represent lower priorities.  However, bitvectors can
be of arbitrary length, and hence the priority numbers they represent can be
of arbitrary precision.

Arbitrary-precision priorities\index{arbitrary-precision priorities} are often
useful in AI search-tree applications.  Suppose we have a heuristic suggesting
that tree node $N_1$ should be searched before tree node $N_2$.  We therefore
designate that node $N_1$ and its descendants will use high priorities, and
that node $N_2$ and its descendants will use lower priorities.  We have
effectively split the range of possible priorities in two.  If several such
heuristics fire in sequence, we can easily split the priority range
\index{priority range splitting} in two enough times that no significant bits
remain, and the search begins to fail for lack of meaningful priorities to
assign.  The solution is to use arbitrary-precision priorities, i.e. bitvector
priorities.

To assign a bitvector priority, two methods are available.  The first is to
obtain a pointer to the priority field using \kw{CkPriorityPtr}, and then
manually set the bits using the bit-setting operations inherent to C.  To
achieve this, one must know the format \index{bitvector format} of the
bitvector, which is as follows: the bitvector is represented as an array of
unsigned integers.  The most significant bit of the first integer contains the
first bit of the bitvector.  The remaining bits of the first integer contain
the next 31 bits of the bitvector.  Subsequent integers contain 32 bits each.
If the size of the bitvector is not a multiple of 32, then the last integer
contains 0 bits for padding in the least-significant bits of the integer.

The second way to assign priorities is only useful for those who are using the
priority range-splitting\index{priority range splitting} described above.  The
root of the tree is assigned the null priority-string.  Each child is assigned
its parent's priority with some number of bits concatenated.  The net effect is
that the entire priority of a branch is within a small epsilon of the priority
of its root.

It is possible to utilize unprioritized messages, integer priorities, and
bitvector priorities in the same program.  The messages will be processed in
roughly the following order\index{multiple priority types}:

\begin{itemize}

\item Among messages enqueued with bitvector priorities, the messages are
  dequeued according to their priority.  The priority ``0000...'' is dequeued
  first, and ``1111...'' is dequeued last.

\item Unprioritized messages are treated as if they had the priority
  ``1000...'' (which is the ``middle'' priority, it lies exactly halfway
  between ``0000...'' and ``1111...'').

\item Integer priorities are converted to bitvector priorities.  They are
  normalized so that the integer priority of zero is converted to ``1000...''
  (the ``middle'' priority).  To be more specific, the conversion is performed
  by adding 0x80000000 to the integer, and then treating the resulting 32-bit
  quantity as a 32-bit bitvector priority.

\item Among messages with the same priority, messages are dequeued in FIFO
  order or LIFO order, depending upon which queuing strategy was used.

\end{itemize}

Additionally, {\sl long integer priorities} can be specified by the {\em L}
strategy.

A final reminder about prioritized execution: \charmpp\ processes messages in
{\it roughly} the order you specify; it never guarantees that it will deliver
the messages in {\it precisely} the order\index{message delivery order} you
specify. Thus, the correctness of your program should never depend on the order
in which the runtime delivers messages. However, it makes a serious attempt to
be ``close'', so priorities can strongly affect the efficiency of your program.

\subsubsection{Skipping the Queue}

Some operations that one might want to perform are sufficiently
latency-sensitive that they should never wait in line behind other
messages. The \charmpp\ runtime offers two attributes for entry
methods, {\kw expedited} and {\kw immediate}, to serve these
needs. For more information on these attributes, see
Section~\ref{attributes} and the example in
  \testreffile{megatest/immediatering.ci}.
