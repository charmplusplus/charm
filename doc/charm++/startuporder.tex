%describe the order in which entities are constructed on PE 0 and other PEs
%what assumptions can user program make about entity availability:
%ie groups are available in any chare array constructor, but not vice versa etc.

The \charmpp{} program starts with the following sequence:
\begin{enumerate}
\item Modules Registration: the modules are registered in the same order with
their specified order at the linking stage of the program compilation.
For example, if "-module A -module B" is specified for \charmpp{} program
linking, then module A is registered before module B at runtime.

\item \kw{initnode},\kw{initproc} Calls: all those calls are invoked before the
creation of other \charmpp{} data structures and the invocation of every
mainchares from different modules.

\item \kw{readonly} Variables: those variables are initialized in the mainchare following the program order as written on PE 0. After the initialization, they
are broadcasted to every other PEs making them available in the constructors
of \charmpp{} objects such as Group objects etc..

\item \kw{Group} and \kw{NodeGroup} Creation: on PE 0, constructors of these
objects are invoked in the program order. However, on every other PEs, their
creation are triggered by messages. Since the message order is not guaranteed
in \charmpp{} program, constructors should \textbf{not} depend on other Group
or NodeGroup objects on a PE. In addition, since those objects are initialized
after the initialization of readonly variabales, readonly variables can be used
in their constructors.

\item \charmpp{} Array Creation: the order of calling Array constructors follows
the same mechanism with that of Group and NodeGroup as described above.
Therefore, the creation of one array should \textbf{not} depend on other arrays.
As Array objects are initialized last, their constructors can use 
readonly variables and local branches to Group or NodeGroup objects.
\end{enumerate}
