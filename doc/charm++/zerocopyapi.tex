\section{Zero Copy Messaging API}

\label{nocopyapi}
Apart from using messages, \charmpp{} also provides APIs to avoid sender
and receiver side copies. On RDMA enabled networks like GNI, Verbs, PAMI and
OFI, these internally make use of one-sided communication by using the
underlying Remote Direct Memory Access (RDMA) enabled network.
For large arrays (few 100 KBs or more), the cost of copying during marshalling
the message can be quite high. Using these APIs can help not only save the
expensive copy operation but also reduce the application's memory footprint by
avoiding data duplication. Saving these costs for large arrays proves to be a
significant optimization in achieving faster message send and receive times in
addition to overall improvement in performance because of lower memory consumption.
On the other hand, using these APIs for small arrays can lead to a drop in
performance due to the overhead associated with one-sided communication. The overhead
is mainly due to additional small messages required for sending the metadata message
and the acknowledgment message on completion.

For within process data-transfers, this API uses regular memcpy to achieve zerocopy
semantics. Similarly, on CMA-enabled machines, in a few cases, this API takes
advantage of CMA to perform inter-process intra-physical host data transfers. This
API is also functional on non-RDMA enabled networks like regular ethernet, except that
it does not avoid copies and behaves like a regular Charm++ entry method invocation.

There are two APIs that provide Zero copy semantics in \charmpp{}:
\begin{itemize}
\item Zero Copy Direct API
\item Zero Copy Entry Method Send API
\end{itemize}

\subsection{Zero Copy Direct API}
The Zero copy Direct API allows users to explicitly invoke a standard set of
methods on predefined buffer information objects to avoid copies. Unlike the
Entry Method API which calls the zero copy operation for every zero copy entry
method invocation, the direct API provides a more flexible interface by allowing
the user to exploit the persistent nature of iterative applications to perform
zero copy operations using the same buffer information objects across iteration
boundaries. It is also more beneficial than the Zero copy entry method API because
unlike the entry method API, which avoids just the sender side copy, the Zero copy
Direct API avoids both sender and receiver side copies.

\vspace{0.1in}
\noindent
To send an array using the zero copy Direct API, define a \kw{CkNcpyBuffer}
object on the sender chare specifying the pointer, size, a CkCallback
object and an optional mode parameter.

\begin{alltt}
CkCallback srcCb(CkIndex_Ping1::sourceDone, thisProxy[thisIndex]);
// CkNcpyBuffer object representing the source buffer
CkNcpyBuffer source(arr1, arr1Size * sizeof(int), srcCb, CK_BUFFER_REG);
\end{alltt}

When used inside a \kw{CkNcpyBuffer} object that represents the source buffer
information, the callback is specified to notify about the safety of
reusing the buffer and indicates that the \kw{get} or \kw{put}
call has been completed. In those cases where the application can determine
safety of reusing the buffer through other synchronization mechanisms, the
callback is not entirely useful and in such cases, \texttt{CkCallback::ignore}
can be passed as the callback parameter. The optional mode operator is used to
determine the network registration mode for the buffer. It is only relevant
on networks requiring explicit memory registration for performing RDMA operations.
These networks include GNI, OFI and Verbs. When the mode is not specified by
the user, the default mode is considered to be
CK\textunderscore BUFFER\textunderscore REG.

Similarly, to receive an array using the Zero copy Direct API, define another
\kw{CkNcpyBuffer} object on the receiver chare object specifying the
pointer, the size, a CkCallback object and an optional mode parameter.
When used inside a \kw{CkNcpyBuffer} object that represents the destination
buffer, the callback is specified to notify the completion of data transfer
into the \kw{CkNcpyBuffer} buffer.
In those cases where the application can determine data transfer completion
through other synchronization mechanisms, the callback is not entirely useful
and in such cases, \texttt{CkCallback::ignore} can be passed as the callback
parameter.

\begin{alltt}
CkCallback destCb(CkIndex_Ping1::destinationDone, thisProxy[thisIndex]);
// CkNcpyBuffer object representing the destination buffer
CkNcpyBuffer dest(arr2, arr2Size * sizeof(int), destCb, CK_BUFFER_REG);
\end{alltt}

Once the source \kw{CkNcpyBuffer} and destination \kw{CkNcpyBuffer} objects have
been defined on the sender and receiver chares, to perform a \kw{get} operation, send
the source \kw{CkNcpyBuffer} object to the receiver chare. This can be done using a
regular entry method invocation as shown in the following code snippet,
where the sender, arrProxy[0] sends its source object to the receiver chare, arrProxy[1].

\begin{alltt}
// On Index 0 of arrProxy chare array
arrProxy[1].recvNcpySrcObj(source);
\end{alltt}

After receiving the sender's source \kw{CkNcpyBuffer} object, the receiver can perform a \kw{get}
operation on its destination \kw{CkNcpyBuffer} object by passing the source object as
an argument to the runtime defined \kw{get} method as shown in the following code snippet.

\begin{alltt}
// On Index 1 of arrProxy chare array
// Call get on the local destination object passing the source object
dest.get(source);
\end{alltt}

This call performs a \kw{get} operation, reading the remote source buffer into the local
destination buffer.

Similarly, a receiver's destination \kw{CkNcpyBuffer} object can be sent to the sender for
the sender to perform a \kw{put} on its source object by passing the source \kw{CkNcpyBuffer}
object as an argument to the runtime defined \kw{put} method as shown in in the code
snippet.

\begin{alltt}
// On Index 1 of arrProxy chare array
arrProxy[0].recvNcpyDestObj(dest);
\end{alltt}

\begin{alltt}
// On Index 0 of arrProxy chare array
// Call put on the local source object passing the destination object
source.put(dest);
\end{alltt}

After the completion of either a \kw{get} or a \kw{put}, the callbacks specified
in both the objects are invoked. Within the
\kw{CkNcpyBuffer} source callback, \texttt{sourceDone()}, the buffer can be safely modified or freed
as shown in the following code snippet.

\begin{alltt}
// Invoked by the runtime on source (Index 0)
void sourceDone() \{
    // update the buffer to the next pointer
    updateBuffer();
\}
\end{alltt}

Similarly, inside the \kw{CkNcpyBuffer} destination callback, \texttt{destinationDone()}, the user
is guaranteed that the data transfer is complete into the destination buffer and the user
can begin operating on the newly available data as shown in the following code snippet.


\begin{alltt}
// Invoked by the runtime on destination (Index 1)
void destinationDone() \{
    // received data, begin computing
    computeValues();
\}
\end{alltt}

The callback methods can also take a pointer to a \texttt{CkDataMsg} message. This message can be
used to access the original pointer passed into the buffer and another optional reference pointer that was
initially set for the \kw{CkNcpyBuffer} object using the method \texttt{setRef}. The following code snippet
illustrates the accessing of the original buffer pointer in the callback method by casting the \texttt{data}
field of the \texttt{CkDataMsg} object into a \texttt{CkNcpyAck} object.

\begin{alltt}
// Invoked by the runtime on source (Index 0)
void sourceDone(CkDataMsg *msg) \{
    // Cast msg->data to a CkNcpyAck
    CkNcpyAck *ack = (CkNcpyAck *)(msg->data);

    // access buffer pointer and free it
    free(ack->ptr);
\}
\end{alltt}

The following code snippet illustrates the usage of the \texttt{setRef} method.

\begin{alltt}
const void *refPtr = &index;
CkNcpyBuffer source(arr1, arr1Size * sizeof(int), srcCb, CK_BUFFER_REG);
source.setRef(refPtr);
\end{alltt}

Similar to the buffer pointer, the user set arbitrary reference pointer can be also accessed in the
callback method. This is shown in the next code snippet.

\begin{alltt}
// Invoked by the runtime on source (Index 0)
void sourceDone(CkDataMsg *msg) \{
    // update the buffer to the next pointer
    updateBuffer();

    // Cast msg->data to a CkNcpyAck
    CkNcpyAck *ack = (CkNcpyAck *)(msg->data);

    // access buffer pointer and free it
    free(ack->ptr);

    // get reference pointer
    const void *refPtr = ack->ref;
\}
\end{alltt}

The usage of \texttt{CkDataMsg} and \texttt{setRef} in order to access the original
pointer and the arbitrary reference pointer is illustrated in
\examplerefdir{zerocopy/direct\textunderscore api/unreg/simple\textunderscore get}

Both the source and destination buffers are of the same type i.e. \kw{CkNcpyBuffer}.
What distinguishes a source buffer from a destination buffer is the way the \kw{get} or
\kw{put} call is made. A valid \kw{get} call using two \kw{CkNcpyBuffer} objects \texttt{obj1} and
\texttt{obj2} is performed as \texttt{obj1.get(obj2)}, where \texttt{obj1} is the local destination
buffer object and \texttt{obj2} is the remote source buffer object that was passed to this PE.
Similarly, a valid \kw{put} call using two \kw{CkNcpyBuffer} objects \texttt{obj1} and
\texttt{obj2} is performed as \texttt{obj1.put(obj2)}, where \texttt{obj1} is the local source buffer
object and \texttt{obj2} is the remote destination buffer object that was passed to this PE.

Since callbacks in \charmpp{} allow to store a reference number, these
callbacks passed into \kw{CkNcpyBuffer} can be set with a
reference number using the method \texttt{cb.setRefNum(num)}. Upon callback
invocation, these reference numbers can be used to identify the buffers that
were passed into the \kw{CkNcpyBuffer} objects. This is
specifically useful where there is an indexed collection of buffers, where the
reference number can be used to index the collection.

Note that the \kw{CkNcpyBuffer} objects can be
either statically declared or be dynamically allocated.
Additionally, the objects are also reusable across iteration boundaries i.e.
after sending the \kw{CkNcpyBuffer} object, the remote PE can use
the same object to perform \kw{get} or \kw{put}. However, it is
important to note that in order to access them, it is required
to store them in a local variable or as a data member of the class. This pattern
of using the same objects across iterations is demonstrated in
\examplerefdir{zerocopy/direct\textunderscore api/reg/pingpong}.

This API is demonstrated in \examplerefdir{zerocopy/direct\textunderscore api}

\subsubsection{Memory Registration and Modes of Operation}

There are four modes of operation for the Zero Copy Direct API. These modes
act as control switches on networks that require memory registration like GNI,
OFI and Verbs, in order to perform RDMA operations . They dictate the functioning of the API
providing flexible options based on user requirement. On other networks, where
network memory management is not necessary (Netlrts) or is internally handled by the lower
layer networking API (PAMI, MPI), these switches are still supported to maintain API
consistency by all behaving in the similar default mode of operation.

\paragraph{\texttt{CK\textunderscore BUFFER\textunderscore REG}:}
\texttt{CK\textunderscore BUFFER\textunderscore REG} is the default mode that
is used when no mode is passed. This mode doesn't distinguish between
non-network and network data transfers. When this mode is passed, the buffer
is registered immediately and this can be used for both non-network sends (memcpy)
and network sends without requiring an extra message being sent by the runtime system
for the latter case. This mode is demonstrated in
\examplerefdir{zerocopy/direct\textunderscore api/reg}

\paragraph{\texttt{CK\textunderscore BUFFER\textunderscore UNREG}:}
When this mode is passed, the buffer is initially unregistered and it is
registered only for network transfers where registration is absolutely required.
For example, if the target buffer is on the same PE or same logical node (or process),
since the \kw{get} internally performs a memcpy, registration is avoided for non-network
transfers. On the other hand, if the target buffer resides on a remote PE on a different
logical node, the \kw{get} is implemented through an RDMA call requiring registration.
In such a case, there is a small message sent by the RTS to register and perform
the RDMA operation. Upon registration, the runtime modifies the state of \kw{CkNcpyBuffer}
object to \texttt{CK\textunderscore BUFFER\textunderscore REG} from
\texttt{CK\textunderscore BUFFER\textunderscore UNREG}. This mode is demonstrated in
\examplerefdir{zerocopy/direct\textunderscore api/unreg}

\paragraph{\texttt{CK\textunderscore BUFFER\textunderscore PREREG}:}
This mode is only beneficial by implementations that use pre-registered memory pools.
In \charmpp{}, GNI and Verbs machine layers use pre-registered memory pools for avoiding
registration costs. On other machine layers, this mode is supported, but it behaves similar
to \texttt{CK\textunderscore BUFFER\textunderscore REG}. A custom allocator, \texttt{CkRdmaAlloc}
can be used to allocate a buffer from a pool of pre-registered memory to avoid the expensive malloc
and memory registration costs. For a buffer allocated through \texttt{CkRdmaAlloc}, the mode
\texttt{CK\textunderscore BUFFER\textunderscore PREREG} should be passed to indicate the use of a
mempooled buffer to the RTS. A buffer allocated with \texttt{CkRdmaAlloc} can be deallocated by
calling a custom deallocator, \texttt{CkRdmaFree}. Although the allocator \texttt{CkRdmaAlloc} and
deallocator, \texttt{CkRdmaFree} are beneficial on GNI and Verbs, the allocators are functional on other
networks and allocate regular memory similar to a \texttt{malloc} call. This mode is demonstrated in
\examplerefdir{zerocopy/direct\textunderscore api/prereg}

\paragraph{\texttt{CK\textunderscore BUFFER\textunderscore NOREG}:}
This mode is used for just storing pointer information without any actual networking
or registration information. It cannot be used for performing any zerocopy operations.
This mode was added as it was useful for implementing a runtime system feature.

\subsubsection{Memory De-registration}
Similar to memory registration, there is a method available to de-register
memory after the completion of the operations. This allows for other buffers to use the
registered memory as machines/networks are limited by the maximum amount of registered or
pinned memory. Registered memory can be de-registered by calling the \texttt{deregisterMem()}
method on the \kw{CkNcpyBuffer} object.

\subsubsection{Other Methods}
In addition to \texttt{deregisterMem()}, there are other methods in a \kw{CkNcpyBuffer} object
that offer other utilities. The
\texttt{init(const void *ptr, size\textunderscore t cnt, CkCallback \&cb,
unsigned short int mode=CK\textunderscore BUFFER\textunderscore UNREG)}
method can be used to re-initialize the \kw{CkNcpyBuffer} object to new values similar to the ones that
were passed in the constructor. For example, after using a \kw{CkNcpyBuffer} object called \texttt{srcInfo},
the user can re-initialize the same object with other values. This is shown in the following code snippet.

\begin{alltt}
// initialize src with new values
src.init(ptr, 200, newCb, CK_BUFFER_REG);
\end{alltt}

Additionally, the user can use another method \texttt{registerMem} in order to register a buffer that has
been de-registered. Note that it is not required to call \texttt{registerMem} after a new initialization
using \texttt{init} as \texttt{registerMem} is internally called on every new initialization. The usage of
\texttt{registerMem} is illustrated in the following code snippet. Additionally, also note that following
de-registration, if intended to register again, it is required to call \texttt{registerMem} even in the
\texttt{CK\textunderscore BUFFER\textunderscore PREREG} mode when the buffer is allocated from a preregistered
mempool. This is required to set the registration memory handles and will not incur any registration costs.

\begin{alltt}
// register previously de-registered buffer
src.registerMem();
\end{alltt}

\subsection{Zero Copy Entry Method Send API}
The Zero copy Entry Method Send API only allows the user to only avoid the sender
side copy without avoiding the receiver side copy. It offloads the user from the
responsibility of making additional calls to support zero copy semantics.
It extends the capability of the existing entry methods with slight modifications
in order to send large buffers without a copy.

\vspace{0.1in}
\noindent
To send an array using the zero copy message send API, specify the array parameter
in the .ci file with the nocopy specifier.

\begin{alltt}
entry void foo (int size, nocopy int arr[size]);
\end{alltt}

While calling the entry method from the .C file, wrap the array i.e the
pointer in a CkSendBuffer wrapper.

\begin{alltt}
arrayProxy[0].foo(500000, CkSendBuffer(arrPtr));
\end{alltt}

Until the RDMA operation is completed, it is not safe to modify the buffer.
To be notified on completion of the RDMA operation, pass an optional callback object
in the CkSendBuffer wrapper associated with the specific nocopy array.

\begin{alltt}
CkCallback cb(CkIndex_Foo::zerocopySent(NULL), thisProxy[thisIndex]);
arrayProxy[0].foo(500000, CkSendBuffer(arrPtr, cb));
\end{alltt}

The callback will be invoked on completion of the RDMA operation associated with the
corresponding array. Inside the callback, it is safe to overwrite the buffer sent
via the zero copy entry method send API and this buffer can be accessed by dereferencing
the CkDataMsg received in the callback.

\begin{alltt}
//called when RDMA operation is completed
void zerocopySent(CkDataMsg *m)
\{
  //get access to the pointer and free the allocated buffer
  void *ptr = *((void **)(m->data));
  free(ptr);
  delete m;
\}
\end{alltt}

The RDMA call is associated with a nocopy array rather than the entry method.
In the case of sending multiple nocopy arrays, each RDMA call is independent of the other.
Hence, the callback applies to only the array it is attached to and not to all the nocopy
arrays passed in an entry method invocation. On completion of the RDMA call for each
array, the corresponding callback is separately invoked.

As an example, for an entry method with two nocopy array parameters, each called with the same
callback, the callback will be invoked twice: on completing the transfer of each of the two
nocopy parameters.

\vspace{0.1in}
\noindent
For multiple arrays to be sent via RDMA, declare the entry method in the .ci file as:

\begin{alltt}
entry void foo (int size1, nocopy int arr1[size1], int size2, nocopy double arr2[size2]);
\end{alltt}

In the .C file, it is also possible to have different callbacks associated with each nocopy array.
\begin{alltt}
CkCallback cb1(CkIndex_Foo::zerocopySent1(NULL), thisProxy[thisIndex]);
CkCallback cb2(CkIndex_Foo::zerocopySent2(NULL), thisProxy[thisIndex]);
arrayProxy[0].foo(500000, CkSendBuffer(arrPtr1, cb1), 1024000, CkSendBuffer(arrPtr2, cb2));
\end{alltt}

This API is demonstrated in \examplerefdir{zerocopy/entry\textunderscore method\textunderscore api}
and \testrefdir{pingpong}

\vspace{0.1in}
\noindent
It should be noted that calls to entry methods with nocopy specified parameters are
currently only supported for point to point operations and not for collective operations.
Additionally, there is also no support for migration of chares that have pending RDMA transfer
requests.

\vspace{0.1in}
\noindent
It should also be noted that the benefit of this API can be seen for large arrays on
only RDMA enabled networks. On networks which do not support RDMA and for within process sends
(which uses shared memory), the API is functional but doesn't show any performance benefit as it
behaves like a regular entry method that copies its arguments.

Table~\ref{tab:rdmathreshold} displays the message size thresholds for the zero copy entry method
send API on popular systems and build architectures. These results were obtained by running
\examplerefdir{zerocopy/entry\textunderscore method\textunderscore api/pingpong} in non-SMP mode
on production builds. For message sizes greater than or equal to the displayed thresholds,
the zero copy API is found to perform better than the regular message send API. For network layers
that are not pamilrts, gni, verbs, ofi or mpi, the generic implementation is used.

\begin{table}[ht]
\begin{tabular}{|p{4cm}|p{1.5cm}|p{4cm}|p{1.5cm}|p{1.5cm}|p{1.5cm}|}
\hline
Machine & Network & Build Architecture & Intra Processor & Intra Host & Inter Host
\\\hline
Blue Gene/Q (Vesta) & PAMI & \verb|pamilrts-bluegeneq| & 4 MB & 32 KB & 256 KB
%\\\hline
%Blue Gene/Q (Vesta) & PAMI & \verb|pamilrts-bluegeneq-smp| & & &
%\\\hline
%Blue Gene/Q (Vesta) & PAMI & \verb|pamilrts-bluegeneq-async-smp| & & &
%\\\hline
%Cray XK6 (Titan) & Gemini & \verb|gni-crayxe| & & &
%\\\hline
%Cray XE6 (Blue Waters) & Gemini & \verb|gni-crayxe-smp| & & &
%\\\hline
%Cray XK6 (Titan) & Gemini & \verb|mpi-crayxe| & & &
%\\\hline
%Cray XE6 (Blue Waters) & Gemini & \verb|mpi-crayxe-smp| & & &
\\\hline
Cray XC30 (Edison) & Aries & \verb|gni-crayxc| & 1 MB & 2 MB & 2 MB
%\\\hline
%Cray XC30 (Edison) & Aries & \verb|gni-crayxc-smp| & & &
\\\hline
Cray XC30 (Edison) & Aries & \verb|mpi-crayxc| & 256 KB & 8 KB & 32 KB
%\\\hline
%Cray XC30 (Edison) & Aries & \verb|mpi-crayxc-smp| & & &
\\\hline
Dell Cluster (Golub) & Infiniband &\verb|verbs-linux-x86_64| & 128 KB & 2 MB & 1 MB
%\\\hline
%Dell Cluster (Golub) & Infiniband &\verb|verbs-linux-x86_64-smp| & &  &
\\\hline
Dell Cluster (Golub) & Infiniband &\verb|mpi-linux-x86_64| & 128 KB & 32 KB & 64 KB
%\\\hline
%Dell Cluster (Golub) & Infiniband &\verb|mpi-linux-x86_64-smp| & & &
\\\hline
Intel Cluster (Bridges) & Intel Omni-Path &\verb|ofi-linux-x86_64| & 64 KB & 32 KB & 32 KB
\\\hline
Intel KNL Cluster (Stampede2) & Intel Omni-Path &\verb|ofi-linux-x86_64| & 1 MB & 64 KB & 64 KB
\\\hline
\end{tabular}
\caption{Message Size Thresholds for which Zero Copy Entry API performs better than Regular API}
\label{tab:rdmathreshold}
\end{table}
