\subsection{Chare Arrays}

Chare arrays \index{arrays} are arbitrarily-sized collections of chares,
referenced by a globally unique array identifier of type \keyword{CkArrayID}
\index{CkArrayID} and an index. 

\uw{ArrayType} is specified in the interface file as:

\begin{tabbing}
~~~~ \=~~~~ \=~~~~ \=~~~~ \=~~~~ \=~~~~ \=~~~~ \=~~~~ \=~~~~ \=~~~~ \kill
\> \kw{array} \uw{ArrayType} \{ \\
\> \> \kw{entry} \uw{ArrayType}(\uw{MessageType1} *); \\
\> \> \kw{entry void} \uw{EntryPointName2}(\uw{MessageType2} *); \\
\> \};
\end{tabbing}

The definition has the form:

\begin{tabbing}
~~~~ \=~~~~ \=~~~~ \=~~~~ \=~~~~ \=~~~~ \=~~~~ \=~~~~ \=~~~~ \=~~~~ \kill
\> \kw{class} \uw{ArrayType} : \kw{public ArrayElement} [: superclass
names] \{ \\
\> \> // Data and member functions as in C++ \\
\> \> // One or more {\it entry method} \index{entry method}
definitions of the form \\
\> \kw{public}: \\
\> \> \uw{ArrayType}(\uw{MessageType1} *{\it MsgPointer}) \\
\> \> \> \{ // C++ code block  \} \\
\> \> \kw{void} \uw{EntryPointName2}(\uw{MessageType2} *{\it MsgPointer}) \\
\> \> \> \{ // C++ code block  \} \\
\> \};
\end{tabbing}

In most respects, array elements are identical to chares.  Differences
include variations in the creation and remote invocation syntax, to
include the array size and array index.  Also, the messages handled by
array elements must be subclasses of the \keyword{ArrayMessage}
\index{ArrayMessage} type.


\subsubsection{Chare Array Creation}

Given the following code for a \index{array}\index{chare array}chare array declaration:
In the ``.ci'' file:

\begin{verbatim}
array G {
  entry G(M1 *);
  entry void someEntry(M2 *);
};
\end{verbatim}

and in the ``.h'' file:

\begin{verbatim}
class G : public ArrayElement {
  public:
    G(M1 *);
    void someEntry(M2 *);

    // Optional, for migratable elements
    G(ArrayElementMigrateMsg *);
    void packsize();
    void pack();
};
\end{verbatim}

An \index{array}array can be created as follows:

\begin{verbatim}
CProxy_A *pA = new CProxy_A(num_elements);
   // or
CkArrayID aid = CProxy_G::ckNew(num_elements);
CProxy_G g(aid);
\end{verbatim}


\subsubsection{Method Invocation on Chare Arrays}

To invoke an entry method f on a chare array, you need to have a proxy
for it.  If you have a global array id (of type CkArrayID) for the
array, you can construct the proxy as follows:

\begin{verbatim}
CkArrayID aid = CProxy_G::ckNew(num_elements);
// aid may be passed in messages to other objects
CProxy_G g(aid);

// Chare array message invocation
g[i].f(msg);

// Chare array broadcast
g.f(msg);
\end{verbatim}





