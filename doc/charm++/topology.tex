\section{Network Topology}
\section{Node Topology}
chao to write this section.

The following calls provide information about the machines upon which the
parallel program is executing.  Processing Element refers to a single CPU.
Node refers to a single machine-- a set of processing elements which share
memory (i.e. an address space).  Processing Elements and Nodes are numbered,
starting from zero.

Thus if a parallel program is executing on one 4-processor workstation and one
2-processor workstation, there would be 6 processing elements (0, 1 ,2, 3, 4,
and 5) but only 2 nodes (0 and 1).  A given node's processing elements are
numbered sequentially.

\function{int CkMyRank()} \index{CkMyRank}
\desc{returns the rank number of the processor on which the call was made.
Processing elements within a node are ranked starting from zero.}

\function{int CkMyNode()} \index{CkMyNode}
\desc{returns the address space number (node number) on which the call was made.}

\function{int CkNumNodes()} \index{CkMyNodes}
\desc{returns the total number of address spaces.}

\function{int CkNodeFirst(int node)} \index{CkNodeFirst}
\desc{returns the processor number of the first processor in this address space.}

\function{int CkNodeSize(int node)} \index{CkNodeSize}
\desc{returns the number of processors in the address space on which the call was made.}

\function{int CkNodeOf(int pe)} \index{CkNodeOf}
\desc{returns the node number on which the call was made.}

\function{int CkRankOf(int pe)} \index{CkRankOf}
\desc{returns the rank of the given processor within its node.}

