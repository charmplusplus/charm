\documentclass[11pt]{article}

\newif\ifpdf
\ifx\pdfoutout\undefined
  \pdffalse
\else
  \pdfoutput=1
  \pdftrue
\fi

\ifpdf
  \pdfcompresslevel=9
  \usepackage[pdftex,colorlinks=true,plainpages=false]{hyperref}
\else
\fi

\usepackage{fullpage}
\pagestyle{headings}
\setlength{\parskip}{0.1in}
\setlength{\textheight}{9.5in}
\setlength{\textwidth}{6.5in}
\setlength{\parindent}{0in}
\setlength{\topmargin}{-.5in}
\parskip 0.1in

%
% Constants
%
\newcommand{\version}{5.0}		%%% The current version number
\newcommand{\prevversion}{4.9}	%%% The previous version number

%
% Commands
%
\newcommand{\zap}[1]{ }
\newcommand{\fcmd}{\bf}		%%% Font for Charm commands
\newcommand{\fparm}{\it\sf}	%%% Font for parameters to Charm commands
\newcommand{\fexec}{\bf}	%%% Font for compile/execute cmds/options
\newcommand{\atitle}[1]{{\it #1}}
\newcommand{\keyword}[1]{{\textbf{#1}}}
\newcommand{\userword}[1]{{\fparm \textsf{#1}}}
\newcommand{\constraint}[1]{Note: {\it #1}}
\newcommand{\note}[1]{Note: {\it #1}}

%
% Conveniences
%
\newcommand{\uw}[1]{\userword{#1}}
\newcommand{\kw}[1]{\keyword{#1}}
\newcommand{\CLocalBranch}{\keyword{CLocalBranch}}

%
%       \CC gives "C++" that looks good.
%
\newcommand{\CC}{C\kern -0.0em\raise 0.5ex\hbox{\normalsize++}}
\newcommand{\emCC}{C\kern -0.0em\raise 0.4ex\hbox{\normalsize\em++}}
\newcommand{\charmpp}{{\sc Charm++}}

\makeindex

\begin{document}

\begin{titlepage}
\vspace*{2in}
\Huge
\begin{center}
The \\
\charmpp \\
Programming Language \\
Manual\\
\vspace*{0.5in}
Version 5.0\\
\vspace*{0.7in}
\today
\end{center}
\normalsize
\end{titlepage}

\section*{Acknowlegements}

\large
The Charm software was developed as a group effort.  The earliest
prototype, Chare Kernel(1.0), was developed by Wennie Shu and Kevin
Nomura working with Laxmikant Kale.  The second prototype, Chare
Kernel(2.0), a complete re-write with major design changes, was
developed by a team consisting of Wayne Fenton, Balkrishna Ramkumar,
Vikram Saletore, Amitabh B. Sinha and Laxmikant Kale. The translator
for Chare Kernel(2.0) was written by Manish Gupta.  Charm(3.0), with
significant design changes, was developed by a team consisting of
Attila Gursoy, Balkrishna Ramkumar, Amitabh B.  Sinha and Laxmikant
Kale, with a new translator written by Nimish Shah.  The \charmpp\ 
implementation was done by Sanjeev Krishnan.  Charm(4.0) included
\charmpp\ and was released in fall 1993.  Charm(4.5) was developed by
Attila Gursoy, Sanjeev Krishnan, Milind Bhandarkar, Joshua Yelon,
Narain Jagathesan and Laxmikant Kale.  Charm(4.8), developed by the
same team included Converse, a parallel runtime system that allows
interoperability among modules written using different paradigms
within a single application. \charmpp\ runtime system was re-targetted
at Converse. Syntactic extensions in \charmpp\ were dropped, and a
simple interface translator was developed that, along with the \charmpp\
runtime, became the \charmpp\ language.  The current version (5.0)
includes a complete rewrite of the \charmpp\ runtime system (using \CC)
and the interface translator (done by Milind Bhandarkar).  It also
includes several new features such as Chare Arrays (developed by
Robert Brunner), and various libraries (written by Terry Wilmarth,
Gengbin Zheng, Laxmikant Kale, Zehra Sura, Milind Bhandarkar, Robert
Brunner, and Krishnan Varadarajan.) A coordination language
``Structured Dagger'' has been implemented on top of \charmpp\ (Milind
Bhandarkar), and included in this version.  IDL bindings for \charmpp\ 
were developed by Jayant Desouza, Milind Bhandarkar and Gengbin
Zheng. Several features have also been added to Converse. Dynamic
seed-based load balancing has been implemented (Terry Wilmarth and
Joshua Yelon), a client-server interface for Converse programs, and
debugging support has been added (Parthasarathy Ramachandran, Jeff
Wright, and Milind Bhandarkar).  Converse has been ported to new
platforms including ASCI Red (Joshua Yelon), Cray T3E (Robert
Brunner), and SGI Origin2000 (Milind Bhandarkar).  The test suite for
\charmpp\ was developed by Michael Lang, Jackie Wang, and Fang
Hu. Projections, the performance visualization and analysis tool, was
redesigned and rewritten using Java by Michael Denardo.

\normalsize


\newpage
\tableofcontents

\newpage

\charmpp\ is a C++-based parallel programming system, founded on the
migratable-objects programming model, and supported by a novel and
powerful adaptive runtime system. It supports both irregular as well
as regular applications, and can be used to specify task-parallelism
as well as data parallelism in a single application. It automates
dynamic load balancing for task-parallel as well as data-parallel
applications, via separate suites of load-balancing strategies. Why
are its message-driven execution model, it supports automatic latency
tolerance, among other features. Charm++ also supports automatic
checkpoint/restart checkpoints, as well as fault tolerance based on
distributed checkpoints.
% {\sc Converse} interoperable runtime system for parallel
% programming.

Charm++ is a production-quality parallel programming system used by
multiple applications in science and engineering on supercomputers as
well as departmental clusters around the world.
Currently the parallel platforms supported by \charmpp\ are the
BlueGene/L,BlueGene/P, BlueGene/Q, Cray XT, XE and XK series
(including XK6 and XE6), 
% XT3/4, Cray X1, Cray T3E,
a single workstation or a network of
workstations (including x86 (running Linux, Windows, MacOS)), etc.
The communication protocols and infrastructures supported by
\charmpp\ are UDP, TCP, Myrinet, Infiniband, uGNI, and PAMI. 
\charmpp\ programs can run without changing the source
on all these platforms.  Please see the \charmpp{}/\converse{}
Installation and Usage
\htmladdnormallink{Manual}{http://charm.cs.uiuc.edu/manuals/html/install/manual.html}
for details about installing, compiling and running
\charmpp\ programs.



\subsubsection{Migratable Objects Programming Model} 
The key feature of the migratabl-objects model is {\em
over-decomposition}: The programmer decomposes the program into a
large number of work units and data units, and specifies the
computation in terms of creation of and interactions between these
units, without any direct reference to the processor on which any unit
resides. This empowers the runtime system to assign units to
processors, and to change the assignment at runtime as
necessary. Charm++ is the main (and early) exemplar of this
programming model. AMPI is another example within the Charm++ family
of the same model.


\subsubsection{\charmpp\ Execution Model}

% A \charmpp\ program consists of a number of \charmpp\ objects
% distributed across the available number of processors. Thus, 
A basic
unit of parallel computation in \charmpp\ programs is a {\em
chare}\index{chare}. 
% a \charmpp\ object that can be created on any
% available processor and can be accessed from remote processors.
A \index{chare}chare is similar to a process, an actor, an ADA task,
etc. At its most basic level, it is just a C++ object.
%  with some of its methods
% that can be invoked from remote objects. 
A \charmpp computation consists of a large number of chares
distributed on available processors of the system, and interacting
with each other via asynchronous method invocations.
Asynchronously invoking a method on a remote object can also be
thought of as
sending a ``message'' to it. So, these method invocations are
sometimes referred to as messages. (besides, in the implementation,
the method invocations are packaged as messages anyway).
\index{chare}Chares can be
created dynamically.
% , and many chares may be active simultaneously.
% Chares send \index{message}{\em messages} to one another to invoke
% methods asynchronously.  

Conceptually, the system maintains a
``work-pool'' consisting of seeds for new \index{chare}chares, and
\index{message}messages for existing chares. The Charm++ runtime system ({\em
Charm RTS}) may pick multiple items, non-deterministically, from this
pool and execute them, with the proviso that two different methods
cannot be simultaneously executing on the same chare object (say, on
different processors). Although one can define a reasonable
theoretical operational semantics of Charm++ in this fashion, a more
practical description of operation is useful to understand Charm++: On
each PE (``PE'' stands for a ``Processing Element''. PEs are akin to
cores, see later section for a precise description), there is a
scheduler operating with its own private pool of messages. Each
instantiated chare has one PE which is where it currently resides. The
pool on each PE includes messages meant for Chares residing on that
PE, and seeds for new Chares that are tentatively meant to be
instantiated on that PE. The scheduler picks a message, creates a new
chare if the message is a seed (i.e. a constructor invocation) for a
new Chare, and invokes the method specified by the message. When the
method returns control back to the scheduler, it repeats the
cycle. I.e. there is no pre-emptive scheduling of other invocations.

When a chare method executes, it may create  method invocatins for other
chares. The Charm Runtime System (RTS, sometimes referred to as the
Chare Kernl in the manual) locates the PE where the targeted chare
resides, and delivers the invocation to the scheduler on that PE. 

Methods of a \index{chare}chare that can be remotely invoked are called
\index{entry method}{\em entry} methods.  Entry methods may take marshalled
parameters, or a pointer to a message object.  Since \index{chare} chares can
be created on remote processors, obviously some constructor of a chare needs
to be an entry method.  Ordinary entry methods\footnote{``Threaded'' or
``synchronous'' methods are different.} are completely non-preemptive--
\charmpp\ will never interrupt an executing method to start any other work,
and all calls made are asynchronous.

\charmpp\ provides dynamic seed-based load balancing. Thus location (processor
number) need not be specified while creating a
remote \index{chare}chare. The Charm RTS will then place the remote
chare on a least loaded processor. Thus one can imagine chare creation
as generating only a seed for the new chare, which may {\em take root}
on some specific processor at a later time. Charm RTS identifies
a \index{chare}chare by a {\em ChareID}.  

% Since user code does not
% need to name a chares' processor, chares can potentially migrate from
% one processor to another.  (This behavior is used by the dynamic
% load-balancing framework for chare containers, such as arrays.)

Chares can be grouped into collections. The types of collections of
chares supported in Charm++ are: {\em chare-arrays}, \index{group}{\em
chare-groups}, and \index{nodegroup}{\em chare-nodegroups}, referred
to as {\em arrays}, {\em groups}, and {\em nodegroups} throughout this
manual for brevity. A Chare-array is a collection of arbitrary number
of migratable chares, indexed by some index type, and mapped to
processors according to a user-defined map group. A group (nodegroup)
is a collection of chares, with exactly one member element on each PE
(SMP node).

Charm++ does not allow global variables except readonly variables
(see \ref{readonly}). A chare can only access its own data directly.
However, each chare is accessible by a globally valid name. So, one
can think of Charm++ as supporting a {\em global object space}.



Every \charmpp\ program must have at least one \kw{mainchare}.  Each
\kw{mainchare} is created by the system on processor 0 when the \charmpp\
program starts up.  Execution of a \charmpp\ program begins with the
Charm Kernel constructing all the designated \kw{mainchare}s.  For
a \kw{mainchare} named X, execution starts at constructor X() or
X(CkArgMsg *) which are equivalent.  Typically, the
\kw{mainchare} constructor starts the computation by creating arrays, other
chares, and groups.  It can also be used to initialize shared \kw{readonly}
objects.

\charmpp\ program execution is terminated by the \kw{CkExit} call.  Like the
\kw{exit} system call, \kw{CkExit} never returns. The Charm RTS ensures
that no more messages are processed and no entry methods are called after a
\kw{CkExit}. \kw{CkExit} need not be called on all processors; it is enough
to call it from just one processor at the end of the computation.

\zap{
The only method of communication between processors in \charmpp\ is
asynchronous \index{entry method} entry method invocation on remote chares.
For this purpose, Charm RTS needs to know the types of
\index{chare}chares in the user program, the methods that can be invoked on
these chares from remote processors, the arguments these methods take as
input etc. Therefore, when the program starts up, these user-defined
entities need to be registered with Charm RTS, which assigns a unique
identifier to each of them. While invoking a method on a remote object,
these identifiers need to be specified to Charm RTS. Registration of
user-defined entities, and maintaining these identifiers can be cumbersome.
Fortunately, it is done automatically by the \charmpp\ interface translator.
The \charmpp\ interface translator generates definitions for {\em proxy}
objects. A proxy object acts as a {\em handle} to a remote chare. One
invokes methods on a proxy object, which in turn carries out remote method
invocation on the chare.}

As described so far, the execution of individual Chares is
``reactive'': When method A is invoked the chare executes this code,
and so on. But very often, chares have specific life-cycles, and the
sequence of entry methods they execute can be specified in a
structured manner, while even allowing for some localized
non-determinism (e.g. a pair of methods may execute in any order, but
when they both execute, the execution continues in a pre-dtermined
manner). To simplify expression of such control structures, Charm++
provides two methods: the structured dagger notation
(Sec \ref{sec:sdag}), which is the main notation we recommend you use.
Alternatively, you may use threaded entry methods, in combination with
futures and sync methods (See \ref{sec:threads}). The threaded methods
run in light-weight user-level threads, and can block waiting for data
in a variety of ways. Again, only the particular thread of a
particular chare is blocked, while the PE continues executing other
chares. 

The normal entry methods, being asynchronus, are not allowed to return
any value, and are declared with a void return type. However, the {\em
sync} methods are an exception to this. They must be called from a
threaded method, and so are allowed to return (certain types of)
values.  

\section{Proxies and Handles}
\label{proxies}

Those familiar with various component models (such as CORBA) in the
distributed computing world will recognize ``proxy'' to be a dummy, standin
entity that refers to an actual entity.  For each chare type, a ``proxy''
class exists.\footnote{The proxy class is generated by the ``interface
translator'' based on a description of the entry methods}  The methods of
this ``proxy'' class correspond to the remote methods of the actual class, and
act as ``forwarders''. That is, when one invokes a method on a proxy to a
remote object, the proxy forwards this method invocation to the actual
remote object. All entities that are created and manipulated remotely in
\charmpp\ have such proxies. Proxies for each type of entity in \charmpp\
have some differences among the features they support, but the basic syntax
and semantics remain the same -- that of invoking methods on the remote
object by invoking methods on proxies.

You can have several proxies that all refer to the same object.

\zap{
Historically, handles (which are basically globally unique
identifiers) were used to uniquely identify \charmpp\ objects.  Unlike
pointers, they are valid on all processors and so could be sent as
parameters in messages.  They are still available, but now proxies
also have the same feature.
}

(NOTE: I assume handles are not to be mentioned. Right?)
Handles (like CkChareID, CkArrayID, etc.) and 

Proxies (like
CProxy\_foo) are just bytes and can be sent in messages, pup'd, and
parameter marshalled.  This is now true of almost all objects in
Charm++: the only exceptions being entire Chares (Array Elements,
etc.) and, paradoxically, messages themselves.


The following sections provide detailed information about various features of the
\charmpp\ programming system.\footnote{For a description of the underlying design
philosophy please refer to the following papers :\\
    L. V. Kale and Sanjeev Krishnan,
    {\em ``\charmpp : Parallel Programming with Message-Driven Objects''},
    in ``Parallel Programming Using \CC'',
    MIT Press, 1995. \\
    L. V. Kale and Sanjeev Krishnan,
    {\em ``\charmpp : A Portable Concurrent Object Oriented System
    Based On \CC''},
    Proceedings of the Conference on Object Oriented Programming,
    Systems, Languages and Applications (OOPSLA), September 1993.
}.



\newpage
\section{Charm++ Overview}

\subsection{\charmpp\ Execution Model}

A \charmpp\ program consists of a number of \charmpp\ objects distributed across
the available number of processors. Thus,
the basic unit of parallel computation in \charmpp\ programs is the
{\em chare}\index{chare}, a \charmpp\ object that can be
created on any available processor and can be accessed from remote processors.
A \index{chare}chare is similar to a
process, an actor, an ADA task, etc.  \index{chare}Chares 
are created dynamically, and many chares
may be active.  Chares send \index{message}{\em messages} to one another 
to invoke methods asynchronously.  Conceptually, the system maintains a
``work-pool'' consisting of seeds for new \index{chare}chares, and 
\index{message}messages for
existing chares. The runtime system (called {\em Charm Kernel}) may pick 
multiple items,
non-deterministically, from this pool and execute them.  It will not
process two messages for the same \index{chare}chare concurrently, but 
otherwise it is free to schedule them in any way.

Methods of a \index{chare}chare that can be remotely invoked are called 
\index{entry method}{\em entry} methods.
Entry methods may take a pointer to a message object, or no parameters.
Since \index{chare}chares can be created on remote processors, obviously 
some constructor of a chare needs to be an entry method.

\charmpp\ provides dynamic seed-based load balancing. Thus location 
(processor number)
need not be specified while creating a remote \index{chare}chare. The Charm Kernel
will then place
the remote chare on a least loaded processor. Thus one can imagine chare 
creation as
generating only a seed for the new chare, which may {\em take root} on the most
{\em fertile} processor. Charm Kernel identifies a \index{chare}chare by a 
{\em ChareID}.
Since user code does not need to name a chares' processor, chares 
can potentially migrate from one processor to another.
(This behaviour is used by the dynamic load-balancing framework for 
chare containers, such as arrays.)

Other \charmpp\ objects are collections of chares. They are:
\index{group}{\em chare-groups}, \index{nodegroup}{\em chare-nodegroups}, and
\index{array}{\em chare-arrays}, referred to as {\em groups}, 
{\em nodegroups}, and
{\em arrays} throughout this manual. A group (nodegroup) is a collection of 
chares, one per processor (SMP node),
that is addressed using a unique system-wide name. An array is a collection 
of arbitrary
number of migratable chares, mapped to processors according to a user-defined 
map group.

Every \charmpp\ program must have at least one \kw{mainchare}.
There can be only one instance of this \index{chare}, which is created on 
processor 0 when the \charmpp\ program starts up.
Execution of a \charmpp\ program begins with Charm Kernel constructing all the
designated \kw{mainchare}s.  Typically, the \kw{mainchare} constructor
starts the computation by creating other chares and chare \index{group}
groups.  It can also be used to initialize shared \kw{readonly} objects.

The only method of communication between processors in \charmpp\ is 
asynchronous \index{entry method}
entry method invocation on remote chares. For this purpose, Charm Kernel needs
to know about the types of \index{chare}chares in the user program, the 
methods that can
be invoked on these chares from remote processors, the arguments these methods
take as input etc. Therefore, when the program starts up, these user-defined
entities need to be registered with Charm Kernel, which assigns a unique
identifier to each of them. While invoking a method on a remote object, these 
identifiers need to be specified to Charm Kernel. Registration of user-defind 
entities, and maintaining these identifiers can be cumbersome. Fortunately, 
it can be done
by the \charmpp\ interface translator. The \charmpp\ interface translator 
generates
definitions for {\em proxy} objects. A proxy object acts as a {\em handle} to 
a remote
chare. One invokes methods on a proxy object, which in turn carries out remote
method invocation on the chare.

In addition, the \charmpp\ interface translator provides ways to enhance the basic
functionality of Charm Kernel using user-level threads and futures. These allow
entry methods to be executed in separate user-level threads. 
These \index{threaded}
{\em threaded} entry methods may block waiting for data by making 
{\em synchronous} 
calls to remote object methods that return results in messages.

\charmpp\ program execution is terminated by the \index{CkExit}
\keyword{CkExit()} call. This call is not the same as \keyword{exit()} in Unix:
it merely informs the Charm Kernel that all computations on all
processors must be terminated, and then returns to the user program:
the programmer should make sure that no work is done by the method after
\keyword{CkExit()} is called. Charm Kernel ensures that no
messages are processed and no entry methods are called after the
currently executing entry method completes. \keyword{CkExit()} need
not be called on all processors; it is enough to call it from just one
processor at the end of the computation.

%The user can specify an entry method (using the \keyword{CkStartQD()}
%\index{CkStartQD} call), to be invoked on a chare when the computation
%has become quiescent (i.e., when no processor is executing an entry
%method and all messages that have been sent have also been
%consumed). This function is useful in computations where the user
%cannot forsee when the program is going to become quiescent. It is
%also useful for computations which proceed in phases --- the
%quiescence entry method can be used to start new phases of
%computation.


\subsection{Entities in \charmpp\ programs}

This section describes various entities in a typical \charmpp\ program.

\subsubsection{Sequential Objects}

A \charmpp\ program typically contains a number of sequential objects. These
objects are similar to the objects in C++, except that they {\em should 
not have any static data members}\footnote{
  This restriction makes \charmpp\ programs portable across distributed
  and shared memory architectures, as well as clusters of shared memory
  multiprocessors. If one feels that these restrictions are too severe,
  one is encouraged to look at the portability macros of Converse, that
  make it possible to write portable programs in presence of global or
  static variables.
}. These objects are created synchronously
on the local processor (using a {\tt new} operator in C++), and their
method could be synchronously invoked only from the local processor.
These objects are not known to the \charmpp\ runtime system, and
thus they need not be mentioned in the module interface files.

\subsubsection{Communication Objects}

Communication objects (messages) supply data arguments to the
asynchronous remote method invocation. Messages are instances of C++ classes
that are subclassed from a special class called {\tt comm\_object}.
These objects are treated differently from other objects in \charmpp\
by the runtime system, and therefore they should be specified in the
interface file of the module. They also have to be created differently
than normal C++ objects. Another variation of communication objects
is conditionally packed and unpacked. This variation should be used when
one wants to send messages that contain pointers to the data rather than
the actual data to other processors. This type of communication objects
contains two methods: {\tt pack}, and {\tt unpack}. 
These messages are specified differently in the interface file for the module.
The third variation of communication objects is called {\em varsize} messages. 
Varsize messages is an effective optimization on conditionally packed messages,
and have to be made known to the \charmpp\ runtime using a special keyword in
the interface file.

\subsubsection{Chares}

Chares are the most important entities in a \charmpp\ program. These 
concurrent objects are different from sequential C++ objects in many
ways. Syntactically, Chares are instances of C++ classes that are derived
from a system-provided class called {\tt chare\_object}. Also, in addition
to the usual C++ private and public data and method members, they contain
some public methods called {\em entry methods}. These entry methods do not
return anything (they are {\tt void} methods), and take exactly one argument,
which is a pointer to a communication object. Chares are {\em accessed} using
a handle (a structure defined in \charmpp), rather than a pointer as in C++.
Semantically, they are different
from C++ objects because they can be created asynchronously from remote
processors, and their entry methods also could be invoked asynchronously
from the remote processors. Since the constructor method is invoked
from remote processor (while creating a chare), every chare should have
its constuctors as entry methods (with one communication object pointer
parameter). These chares and their entry methods have to be specified
in the interface file.

\subsubsection{Branched Chares}

Branched chares\footnote{
  These were called Branch Office Chares (BOC) in earlier version of Charm.
} (or charegroups) are a special type of concurrent objects.
Each branched chare is a collection of chares, with one representative on
each processor. All the branches of a branched chare share a globally
unique name (handle). An entire branched chare could be addressed using
this global handle, and an individual branch of a branched chare can be
addressed using the global handle, and a processor number. Branched chares
are instances of C++ classes subclasses from a system-provided class called
{\tt groupmember}. The runtime system has to be notified that these
chares are semantically different, and therefore branched chares have 
a different declaration in the interface specification file.


\section{The \charmpp\ Language}
  A \charm program is essentially a \CC program where some components describe
its parallel structure. Sequential code can be written using any programming
technologies that cooperate with the \CC toolchain. This includes C and
Fortran. Parallel entities in the user's code are written in \CC{}. These
entities interact with the \charm framework via inherited classes and function
calls.


\section{.ci Files}
\index{ci}
All user program components that comprise its parallel interface (such as
messages, chares, entry methods, etc.) are granted this elevated status by
declaring or describing them in separate \emph{charm interface} description
files. These files have a \emph{.ci} suffix and adopt a \CC-like declaration
syntax with several additional keywords. In some declaration contexts, they
may also contain some sequential \CC source code.
%that is embedded unmodified into the generated code.
\charm parses these interface descriptions and generates \CC code (base
classes, utility classes, wrapper functions etc.) that facilitates the
interaction of the user program's entities with the framework.  A program may
have several interface description files.


\section{Modules}
\index{module}
The top-level construct in a \ci file is a named container for interface
declarations called a \kw{module}. Modules allow related declarations to be
grouped together, and cause generated code for these declarations to be grouped
into files named after the module. Modules cannot be nested, but each \ci file
can have several modules. Modules are specified using the keyword \kw{module}.

\begin{alltt}
module myFirstModule \{
    // Parallel interface declarations go here
    ...
\};
\end{alltt}


\section{Generated Files}
\index{decl}\index{def}

Each module present in a \ci file is parsed to generate two files. The basename
of these files is the same as the name of the module and their suffixes are
\emph{.decl.h} and \emph{.def.h}. For e.g., the module defined earlier will
produce the files ``myFirstModule.decl.h'' and ``myFirstModule.def.h''. As the
suffixes indicate, they contain the declarations and definitions respectively,
of all the classes and functions that are generated based on the parallel
interface description.

We recommend that the header file containing the declarations (decl.h) be
included at the top of the files that contain the declarations or definitions
of the user program entities mentioned in the corresponding module. The def.h
is not actually a header file because it contains definitions for the generated
entities. To avoid multiple definition errors, it should be compiled into just
one object file. A convention we find useful is to place the def.h file at the
bottom of the source file (.C, .cpp, .cc etc.) which includes the definitions
of the corresponding user program entities.

\experimental
It should be noted that the generated files have no dependence on the name of the \ci
file, but only on the names of the modules. This can make automated dependency-based
build systems slightly more complicated. We adopt some conventions to ease this process.
This is described in~\ref{AppendixSectionDescribingPhilRamsWorkOnCi.stampAndCharmc-M}.


\section{Module Dependencies}
\index{extern}

A module may depend on the parallel entities declared in another module. It can
express this dependency using the \kw{extern} keyword. \kw{extern}ed modules
do not have to be present in the same \ci file.

\begin{alltt}
module mySecondModule \{

    // Entities in this module depend on those declared in another module
    extern module myFirstModule;

    // More parallel interface declarations
    ...
\};
\end{alltt}

The \kw{extern} keyword places an include statement for the decl.h file of the
\kw{extern}ed module in the generated code of the current module. Hence,
decl.h files generated from \kw{extern}ed modules are required during the
compilation of the source code for the current module. This is usually required
anyway because of the dependencies between user program entities across the two
modules.

\section{The Main Module and Reachable Modules}
\index{mainmodule}

\charm software can contain several module definitions from several
independently developed libraries / components. However, the user program must
specify exactly one module as containing the starting point of the program's
execution. This module is called the \kw{mainmodule}. Every \charm program
has to contain precisely one \kw{mainmodule}.

All modules that are ``reachable'' from the \kw{mainmodule} via a chain of
\kw{extern}ed module dependencies are included in a \charm program. More
precisely, during program execution, the \charm runtime system will recognize
only the user program entities that are declared in reachable modules. The
decl.h and def.h files may be generated for other modules, but the runtime
system is not aware of entities declared in such unreachable modules.

\begin{alltt}
module A \{
    ...
\};

module B \{
    extern module A;
    ...
\};

module C \{
    extern module A;
    ...
\};

module D \{
    extern module B;
    ...
\};

module E \{
    ...
\};

mainmodule M \{
    extern module C;
    extern module D;
    // Only modules A, B, C and D are reachable and known to the runtime system
    // Module E is unreachable via any chain of externed modules
    ...
\};
\end{alltt}


\section{Including other headers}
\index{include}

There can be occasions where code generated from the module definitions
requires other declarations / definitions in the user program's sequential
code. Usually, this can be achieved by placing such user code before the point
of inclusion of the decl.h file. However, this can become laborious if the
decl.h file has to included in several places. \charm supports the keyword
\kw{include} in \ci files to permit the inclusion of any header directly into
the generated decl.h files.

\begin{alltt}
module A \{
    include "myUtilityClass.h"; //< Note the semicolon
    // Interface declarations that depend on myUtilityClass
    ...
\};

module B \{
    include "someUserTypedefs.h";
    // Interface declarations that require user typedefs
    ...
\};

module C \{
    extern module A;
    extern module B;
    // The user includes will be indirectly visible here too
    ...
\};
\end{alltt}


\section{The main() function}

The \charmpp framework implements its own main\(\) function and retains control
until the parallel execution environment is initialized and ready for executing
user code. Hence, the user program must not define a \emph{main()} function.
Control enters the user code via the \kw{mainchare} of the \kw{mainmodule}.
This will be discussed in further detail in~\ref{mainchare}.

Using the facilities described thus far, the parallel interface declarations
for a \charm program can be spread across multiple ci files and multiple
modules, permitting good control over the grouping and export of parallel API.
This aids the encapsulation of parallel software.

\section{Compiling \charm Programs}
\index{charmc}

\charm provides a compiler-wrapper called \kw{charmc} that handles all \ci, C,
\CC and fortran source files that are part of a user program. Users can invoke
charmc to parse their interface descriptions, compile source code and link
objects into binaries. It also links against the appropriate set of charm
framework objects and libraries while producing a binary. \kw{charmc} and its functionality
is described in~\ref{sec:compile}.

	
  \subsection{\charmpp{} Messages}

\subsubsection{What are messages?}

A bundle of data sent, via a proxy, to another chare. A message is
a special kind of heap-allocated C++ object.

\subsubsection{Should I use messages?}

It depends on the application. We've found parameter marshalling to be less
confusing and error-prone than messages for small parameters. Nevertheless,
messages can be more efficient, especially if you need to buffer incoming data,
or send complicated data structures (like a portion of a tree).

\subsubsection{What is the best way to pass pointers in a message?}

You can't pass pointers across processors. This is a basic fact of
life on distributed-memory machines.

You can, of course, pass a copy of an object referenced via a pointer
across processors--either dereference the pointer before sending, or use
a varsize message.

\subsubsection{Can I allocate a message on the stack?}

No. You must allocate messages with {\em new}.

\subsubsection{Do I need to delete messages that are sent to me?}

Yes, or you will leak memory! If you receive a message, you are responsible
for deleting it. This is exactly opposite of parameter marshalling,
and much common practice. The only exception are entry methods declared as
[nokeep]; for these the system will free the message automatically at the end of
the method.

\subsubsection{Do I need to delete messages that I allocate and send?}

No, this will certainly corrupt both the message and the heap! Once
you've sent a message, it's not yours any more. This is again exactly the
opposite of parameter marshalling.

\subsubsection{What can a variable-length message contain?}

Variable-length messages can contain arrays of any type, both primitive type or
any user-defined type. The only restriction is that they have to be 1D arrays.

\subsubsection{Do I need to delete the arrays in variable-length messages?}

No, this will certainly corrupt the heap! These arrays are allocated in a single
contiguous buffer together with the message itself, and is deleted when the
message is deleted.

\subsubsection{What are priorities?}

Priorities are special values that can be associated with messages, so that the
Charm++ scheduler will generally prefer higher priority messages when choosing a buffered message from the queue to invoke as an entry method.  Priorities are often respected by Charm++ scheduler, but for correctness, a program must never rely upon any particular ordering of message deliveries. Messages with priorities are typically used to encourage high performance behavior of an application.


For integer priorities, the smaller the priority value, the higher the priority
of the message. Negative value are therefore higher priority than positive ones. 
To enable and set a message's priority there is a special {\em new} syntax and
{\em CkPriorityPtr} function; see the manual for details. If no priority is set,
messages have a default priority of zero.

\subsubsection{Can messages have multiple inheritance in Charm++?}

Yes, but you probably shouldn't.  Perhaps you want to consider using \htmladdnormallink{generic or meta programming}{http://charm.cs.illinois.edu/manuals/html/charm++/15.html} techniques with templated chares, methods, and/or messages instead. 

%<br>Messages can't be inherited at all-- they don't even have *single*
%inheritance. This is a silly limitation; but the fact that messages need
%to be transmitted as flat byte streams puts strong limits on what we can
%do with them.

%\subsubsection{What is the difference between {\tt new} and {\tt alloc}?}


%My understanding is that </b><tt>new</tt><b>
%calls </b><tt>alloc</tt><b>, but what else is
%</b><tt>new</tt><b> doing?
%I.e. why do both exist?</b></li>

%<br><tt>new</tt> is an operator for the class, and
%<tt>alloc</tt> is a
%static method. <tt>alloc</tt> basically calls <tt>CkAlloc</tt> after calculating
%the sizes and the priority bits etc. <tt>new</tt> is just a wrapper around
%<tt>alloc</tt>.
%You should always call <tt>new</tt>.


  \section{Sequential Objects}
\index{sequential objects} 

These are the same as \CC{} classes and functions.  All \CC{} features can
be used.  However, care needs to be taken when sequential objects
interact with \charmpp\ objects.

  \section{Chare Objects}

\index{chare}Chares are concurrent objects with methods that can be invoked
remotely.  These methods are known as \index{entry method}entry methods, and 
must be specified in the interface ({\tt .ci}) file:

\begin{tabbing}
~~~~ \=~~~~ \=~~~~ \=~~~~ \=~~~~ \=~~~~ \=~~~~ \=~~~~ \=~~~~ \=~~~~ \kill
\> \kw{chare} \uw{ChareType} \{ \\
\> \> \kw{entry} \uw{ChareType}(\uw{MessageType1} *); \\
\> \> \kw{entry void} \uw{EntryMethodName2}(\uw{MessageType2} *); \\
\> \};
\end{tabbing}

A corresponding \index{chare}chare definition in the {\tt .h} file would 
have the form:

\begin{tabbing}
~~~~ \=~~~~ \=~~~~ \=~~~~ \=~~~~ \=~~~~ \=~~~~ \=~~~~ \=~~~~ \=~~~~ \kill
\> \kw{class} \uw{ChareType} : \kw{public Chare} [: superclass names] \{ \\
\> \>   // Data and member functions as in C++ \\
\> \>   // One or more {\it entry method} \index{entry method}
definitions of the form: \\
\> \kw{public}: \\
\> \> \uw{ChareType}(\uw{MessageType1} *{\it MsgPointer}) \\
\> \> \> \{ // C++ code block  \} \\
\> \> \kw{void} \uw{EntryMethodName2}(\uw{MessageType2} *{\it MsgPointer}) \\
\> \> \> \{ // C++ code block  \} \\
\> \};
\end{tabbing}

\index{chare}
Chares are concurrent objects encapsulating medium-grained units of
work.  Chares can be dynamically created on any processor; there may
be thousands of chares on a processor. The location of a chare is
usually determined by the dynamic load balancing strategy; however,
once a chare commences execution on a processor, it does not migrate
to other processors\footnote{Except when it is part of an array.}.  
Chares do not have a default ``thread of
control'': the entry methods \index{entry methods} in a
chare execute in a message driven fashion upon the arrival of a 
message\footnote{Threaded methods augment this behavior since they execute in
a separate user-level thread, and thus can block to wait for data.}.

The entry method definition specifies a function that is executed {\it
without interruption} when a message is received and scheduled for
processing. Only one message per chare is processed at a time.  Entry
methods are defined exactly as normal C++ function members, except
that they must have the return value \kw{void} (except for the
constructor entry method which may not have a return value, and for a {\em synchronous}
entry method, which is invoked by a {\em threaded} method in a remote chare) and they
must have exactly one argument which is a pointer to a
message.

Each chare instance is identified by a {\it handle} \index{handle}
which is essentially a global pointer, and is unique across all
processors.  The handle of a chare has type \kw{CkChareID}.  The
variable \kw{thishandle} holds the handle of the
chare whose entry function or public function is currently executing.
\kw{thishandle} is a public instance variable of the chare object
(it is inherited from the system-defined superclass for chares, \kw{Chare}).
\kw{thishandle} can be used to set fields in a message. This  
mechanism allows chares to send their handles to other chares.

\subsection{Chare Creation}
\label{chare creation}

First, a \index{chare}chare needs to be declared, both in {\tt .ci} file and in
{\tt .h} file, as stated earlier. The following is an example of
declaration for a \index{chare}chare of user-defined type \uw{C}, where \uw{M1}
and \uw{M2} are user-defined \index{message}message types, and \uw{someEntry}
is an entry method.

In the {\tt mod.ci} file we have:

\begin{verbatim}
module mod {
  chare C {
    entry C(M1 *);
    entry void someEntry(M2 *);
  };
}
\end{verbatim}

and in the {\tt mod.h} file:

\begin{verbatim}
#include "mod.decl.h"
class C : public Chare {
  public:
    C(M1 *);
    void someEntry(M2 *);
};
\end{verbatim}

Now one can use the class \kw{CProxy}\_\uw{chareType}\index{CProxy\_} to create
a new instance of a \index{chare}chare.  Here \uw{chareType} gets replaced with
whatever \index{chare}chare type we want.  For the above example, proxies would
be of type \kw{CProxy}\_\uw{C}. A number of \index{chare}chare creation calls
exist as static or instance methods of class \kw{CProxy}\_\uw{chareType}:

\begin{tabbing}
~~~~ \=~~~~ \=~~~~ \=~~~~ \=~~~~ \=~~~~ \=~~~~ \=~~~~ \=~~~~ \=~~~~ \kill
\> \kw{CProxy}\_\uw{chareType}::\kw{ckNew}(\uw{MessageType} *{\it
msgPtr}, \kw{CkChareID} *{\it vHdl}, \kw{int} {\it destPE});
\end{tabbing}

where each parameter above is optional and where

\begin{itemize}

\item \uw{chareType} is the name of the type of \index{chare}chare to be
created.

\item {\it msgPtr} is a pointer to a \index{message}message whose type must
correspond to the parameter for the \index{constructor}constructor entry method
\uw{chareType}(\uw{MessageType} *) in \index{chare}chare \uw{chareType}.  This
parameter may be omitted if the \index{constructor}constructor takes a void
parameter.

\item {\it vHdl} is a pointer to a \index{chare}chare handle of type
\kw{CkChareID}, which is filled by the \kw{ckNew} method. This optional
argument can be used if the user desires to have a {\em virtual} handle
\index{virtual handle} to the instance of the \index{chare}chare that will be
created. This handle is useful for sending \index{message}messages to the
\index{chare}chare, even though it has not yet been created on any processor.
Messages sent to this virtual handle are either queued up to be sent to the
\index{chare}chare after it has been created, or simply redirected if the
\index{chare}chare has already been created. For performance reasons,
therefore, virtual handles should be used only when absolutely necessary.
Virtual handles are otherwise like normal \index{handle}handles, and may be
sent to other processors in \index{message}messages.  

\item {\it destPE}: when a \index{chare}chare is to be created at a specific
processor, the {\it destPE} is used to specify that processor.  Note that, in
general, for good \index{load balancing}load balancing, the user should let
\charmpp\ determine the processor on which to create a \index{chare}chare.
Under unusual circumstances, however, the user may want to choose the
destination processor.  If a process replicated on every processor is desired,
then a \index{chare group}chare group should be used.  If no particular
processor is required, the parameter can be omitted, or \kw{CK\_PE\_ANY}.

\end{itemize}

The \index{chare}chare creation method deposits the \index{seed}{\em seed} for
a chare in a pool of seeds and returns immediately. The \index{chare}chare will
be created later on some processor, as determined by the dynamic \index{load
balancing}load balancing strategy. When a \index{chare}chare is created, it is
initialized by calling its   \index{constructor}constructor \index{entry
method}entry method with the \index{message}message parameter specified to the
\index{chare}chare creation method.  The method operator does not return any
value but fills in the \index{virtual handle}virtual handle to the newly
created \index{chare}chare if specified.

The following are some examples on how to use the \index{chare}chare creation
method to create chares.

\begin{enumerate}
\item{This will create a new \index{chare}chare of type \uw{C} on {\it any} processor:}
\begin{tabbing}
~~~~ \=~~~~ \=~~~~ \=~~~~ \=~~~~ \=~~~~ \=~~~~ \=~~~~ \=~~~~ \=~~~~ \kill
\> \uw{MessageType} *{\it MsgPtr} = \kw{new} \uw{MessageType}; \\
\> \kw{CProxy}\_\uw{C} *{\it pC} = \kw{new} \kw{CProxy}\_\uw{C}({\it
MsgPtr}); \\
\> \> // or \\
\> \uw{MessageType} *{\it MsgPtr} = \kw{new} \uw{MessageType}; \\
\> \kw{CProxy}\_\uw{C}::\kw{ckNew}({\it MsgPtr});
\end{tabbing} 

\item{This will create a new \index{chare}chare of type \uw{C} on processor {\it destPE}:}
\begin{tabbing}
~~~~ \=~~~~ \=~~~~ \=~~~~ \=~~~~ \=~~~~ \=~~~~ \=~~~~ \=~~~~ \=~~~~ \kill
\> \uw{MessageType} *{\it MsgPtr} = \kw{new} \uw{MessageType}; \\
\> \kw{CProxy}\_\uw{C}::\kw{ckNew}({\it MsgPtr}, {\it destPE});
\end{tabbing}

\item{The following first creates a \kw{CkChareID} {\it cid},
then creates a new \index{chare}chare of type \uw{C} on processor {\it destPE}:}

\begin{tabbing}
~~~~ \=~~~~ \=~~~~ \=~~~~ \=~~~~ \=~~~~ \=~~~~ \=~~~~ \=~~~~ \=~~~~ \kill
\> \uw{MessageType} *{\it MsgPtr} = \kw{new} \uw{MessageType}; \\
\> \kw{CkChareID} {\it cid}; \\
\> \kw{CProxy}\_\uw{C}::\kw{ckNew}({\it MsgPtr}, \&{\it cid}, {\it
destPE}); \\
\> \kw{CProxy}\_\uw{C} {\it c}({\it cid});
\end{tabbing}

\end{enumerate}

\subsection{Method Invocation on Chares}

Before sending a \index{message}message to a \index{chare}chare via an
\index{entry method}entry method, we need to get a \index{proxy}{\it proxy} of
that \index{chare}chare using either the chare identifier, or by creating the
proxy directly from the \index{chare}chare creation (see Section~\ref{chare
creation}. The following code creates a proxy from a \kw{CkChareID}.

\begin{tabbing} ~~~~ \=~~~~ \=~~~~ \=~~~~ \=~~~~ \=~~~~ \=~~~~ \=~~~~ \=~~~~
\=~~~~ \kill \> \kw{CProxy}\_\uw{chareType} {\it chareProxy}({\it chareID});
\end{tabbing}

This declares a proxy named {\it chareProxy} of \uw{chareType} using {\it
chareID}.  Alternatively, we can create a proxy pointer:

\begin{tabbing} ~~~~ \=~~~~ \=~~~~ \=~~~~ \=~~~~ \=~~~~ \=~~~~ \=~~~~ \=~~~~
\=~~~~ \kill \> \kw{CProxy}\_\uw{chareType} *{\it chareProxyPointer} = \kw{new
CProxy}\_\uw{chareType}({\it chareID}); \end{tabbing}

This creates a pointer named {\it chareProxyPointer} which points to the proxy. 

A message \index{message} may be sent to a \index{chare}chare using the
notation:

\begin{tabbing} ~~~~ \=~~~~ \=~~~~ \=~~~~ \=~~~~ \=~~~~ \=~~~~ \=~~~~ \=~~~~
\=~~~~ \kill \> {\it chareProxy}$.$\uw{EntryMethod}({\it MessagePointer}) \\ \>
\> or, \\ \> {\it chareProxyPointer}$->$\uw{EntryMethod}({\it MessagePointer})
\end{tabbing}

This sends the message pointed to by {\it MessagePointer} to the
\index{chare}chare whose proxy/proxy pointer is {\it chareProxy}/{\it
chareProxyPointer} at the \index{entry method}entry method \uw{EntryMethod},
which must be a valid entry method of that \index{chare}chare type. This call
is asynchronous and non-blocking; it returns immediately after sending the
message. 

  \subsection{Group Objects}

A {\sl group\footnote{Originally called {\em Branch Office Chare} or 
{\em Branched Chare}}} \index{group}is a collection of chares where 
there exists \index{chare}one chare (or {\sl branch}) on each
processor.   Each branch has its own data members.  Groups have
a definition syntax similar to normal chares, except that they must
inherit from the system defined class \keyword{Group}, rather than
\keyword{Chare}.

In the interface file, we declare

\begin{tabbing}
~~~~ \=~~~~ \=~~~~ \=~~~~ \=~~~~ \=~~~~ \=~~~~ \=~~~~ \=~~~~ \=~~~~ \kill
\> \kw{group} \uw{GroupType} \{ \\
\> \>  // Interface specifications as for normal chares \\
\> \};
\end{tabbing}

In the {\tt .h} file, we define \uw{GroupType} as follows:

\begin{tabbing}
~~~~ \=~~~~ \=~~~~ \=~~~~ \=~~~~ \=~~~~ \=~~~~ \=~~~~ \=~~~~ \=~~~~ \kill
\> \kw{class} \uw{GroupType} : \kw{public Group} [,other superclasses
] \{ \\
\> \> // Data and member functions as in C++ \\
\> \> // Entry functions as for normal chares \\
\> \};
\end{tabbing}

A group is identified by a globally unique group identifier, whose type is
\kw{CkGroupID}. \index{CkGroupID}This identifier is common to all of the 
group's branches and can be obtained from the variable \keyword{thisgroup},
\index{thisgroup}which is a public local variable of the \keyword{Group} 
superclass.  For groups, \kw{thishandle} \index{thishandle} is the handle of 
the particular branch in which the function is executing: it is 
a normal chare handle.

Groups can be used to implement data-parallel operations easily.  In
addition to sending messages to a particular branch of a group, one
can broadcast messages to all branches of a group.  There can be many
instances corresponding to a group type.  Each instance has a
different group handle, and its own set of branches.

\subsubsection{Group Creation}

\noindent {\bf Chare Group Declaration}:

\noindent Given a {\tt .ci} file as follows:

\begin{verbatim}
group G {
  entry G(M1 *);
  entry void someEntry(M2 *);
};
\end{verbatim}

\noindent and the following {\tt .h} file:

\begin{verbatim}
class G : public Group {
  public:
    G(M1 *);
    void someEntry(M2 *);
};
\end{verbatim}

we can create a \index{group}group in a manner similar to a regular \index{chare}chare.  Note
the difference in how the \index{virtual handle}virtual handle is created.

\begin{verbatim}
M1 *m1 = new M1;
CProxy_G *pG = new CProxy_G(m1);
  // or
CkGroupID gid = CProxy_G::ckNew(m1);
CProxy_G g(gid);
\end{verbatim}

\subsubsection{Method Invocation on Groups}

Before sending a message to a \index{group}group via an entry
method, we need to get a proxy of that group using the group identifier (a
\index{CkGroupID}\kw{CkGroupID}). The syntax for obtaining the proxy or a proxy
pointer is:

\begin{tabbing} ~~~~ \=~~~~ \=~~~~ \=~~~~ \=~~~~ \=~~~~ \=~~~~ \=~~~~ \=~~~~
\=~~~~ \kill \> \kw{CProxy}\_\uw{groupType} {\it groupProxy}({\it groupID}); \\
\> \> or, \\ \> \kw{CProxy}\_\uw{groupType} *{\it groupProxyPointer} = \kw{new
CProxy}\_\kw{groupType}({\it groupID}); \end{tabbing}

The first approach creates a proxy to the group represented by {\it groupID}
while the second creates a pointer named {\it groupProxyPointer} to a proxy to
the group represented by {\it groupID}. 

A message may be sent to a particular \index{branch}branch of group using the
notation:

\begin{tabbing} ~~~~ \=~~~~ \=~~~~ \=~~~~ \=~~~~ \=~~~~ \=~~~~ \=~~~~ \=~~~~
\=~~~~ \kill \> {\it groupProxy}$.$\uw{EntryMethod}({\it MessagePointer}, {\it
Processor}) \\ \> \> or, \\ \> {\it groupProxyPointer}$->$\uw{EntryMethod}({\it
MessagePointer}, {\it Processor}) \end{tabbing}

This sends the message in {\it MessagePointer} to the \index{branch}branch of
the group represented by {\it groupID} which is on processor number {\it
Processor} at the entry method \uw{EntryMethod}, which must be a valid entry
method of that group type. This call is asynchronous and non-blocking; it
returns immediately after sending the message.

A message may be broadcast \index{broadcast} to all branches of a branched
chare (i.e., to all processors) using the notation :

\begin{tabbing} ~~~~ \=~~~~ \=~~~~ \=~~~~ \=~~~~ \=~~~~ \=~~~~ \=~~~~ \=~~~~
\=~~~~ \kill \> {\it groupProxy}$.$\uw{EntryMethod}({\it MessagePointer}) \\ \>
{\it groupProxyPointer}$->$\uw{EntryMethod}({\it MessagePointer}) \end{tabbing}

This sends the message in {\it MessagePointer} to all branches of the group at
the entry method {\sf EntryMethod}, which must be a valid entry method of that
group type. This call is asynchronous and non-blocking; it returns immediately
after sending the message.

Note that the programmer relinquishes control of a message after sending it.
Further access to the message field can cause runtime errors.


Sequential objects, chares and other groups can access public members of the
\index{branch}branch of a group \index{group} {\it on their processor} using
the following notation:

((\uw{GroupType}*)\kw{CkLocalBranch}(\kw{CkGroupID} {\it groupID}))-$>${\it
DataMember}, and \\ ((\uw{GroupType}*)\kw{CkLocalBranch}(\kw{CkGroupID} {\it
groupID}))-$>$\uw{method}().  \index{CkLocalBranch}

Thus a dynamically created \index{chare}chare can call a public method of a
group without needing to know which processor it actually resides: the method
executes in the local \index{branch}branch of the group.  Once a proxy to the local branch of a group is obtained, that branch can be thought of as a regular object.  Its public methods can return values, and its public data is readily accessible.   

\index{CkLocalBranch}\kw{CkLocalBranch} returns a generic ({\tt void
*}) pointer.  It needs to be
cast to a pointer of appropriate classe before invoking methods or
accessing data members. \charmpp\ provides another way to do this using
the generated
\index{proxy}{\em proxy} classes. One may call the static method
\kw{ckLocalBranch} of the proxy class of appropriate
group to get the correct type of pointer.  For
example, method \uw{foo} can to be invoked on the local \index{branch}branch of
a group \uw{G} with \uw{gid} as CkGroupID as:

(\kw{CProxy}\_\uw{G}::\kw{ckLocalBranch}({\it gid}))-$>$\uw{foo}(...);\\
\index{ckLocalBranch}

One very nice use of Groups is to reduce the number of messages sent between processors by collecting the data from all the chares on a single processor before sending that data to the mainchare.  To do this, create basic chares to break up the work of a problem.  Also, create a group.  When a particular chare finishes its work, it reports its findings to the local branch of the group.  When all the chares on one processor are complete, the local branch of the group can then report to the main chare.  This reduces the number of messages sent to main from the number of chares created to the number of processors.     







  \section{NodeGroup Objects}

The {\em node group} construct \index{node groups} \index{nodegroup}
\index{Nodegroup} is similar to the group construct discussed
above. Rather than having one chare per PE, a node group is a
collection of chares with one chare per {\em process}, or {\em logical
  node}.  Therefore, each logical node hosts a single branch of the
node group.  As with groups, node groups can be addressed via globally
unique identifiers. Nonetheless, there are significant differences in 
the semantics of node groups as compared to groups and chare arrays. 
When an entry method of a node group is executed
on one of its branches, it executes on {\em some} PE within the
logical node. Also, multiple entry method calls can execute
concurrently on a single node group branch. This makes node groups
significantly different from groups and requires some care when using
them.

\subsection{NodeGroup Declaration} 

Node groups are defined in a similar way to groups.  \footnote{As with groups,
older syntax allows node groups to inherit from \kw{NodeGroup} instead of a
specific, generated ``\uw{CBase\_}'' class.} For example, in the interface file, we declare:

\begin{alltt}
 nodegroup NodeGroupType \{
  // Interface specifications as for normal chares
 \};
\end{alltt}

In the {\tt .h} file, we define \uw{NodeGroupType} as follows:

\begin{alltt}
 class NodeGroupType : public CBase_NodeGroupType \{
  // Data and member functions as in \CC{}
  // Entry functions as for normal chares
 \};
\end{alltt}

Like groups, NodeGroups are identified by a globally unique identifier of type
\index{CkGroupID}\kw{CkGroupID}.  Just as with groups, this identifier is
common to all branches of the NodeGroup, and can be obtained from the inherited
data member \index{thisgroup}\kw{thisgroup}.
There can be many instances corresponding to a single NodeGroup
type, and each instance has a different identifier, and its own set of
branches.


%, and once again, \index{thishandle}
%\kw{thishandle} is the handle of the particular branch in which the function is
%executing.


\subsection{Method Invocation on NodeGroups}

As with chares, chare arrays and groups, entry methods are invoked on
NodeGroup branches via proxy objects. 
An entry method may be invoked on a {\em particular} \index{branch}branch of a
\index{nodegroup}nodegroup by specifying a {\em logical node number} argument
to the square bracket operator of the proxy object. A broadcast is expressed
by omitting the square bracket notation. For completeness, example syntax for these
two cases is shown below:

\begin{alltt}
 // Invoke `someEntryMethod' on the i-th logical node of
 // a NodeGroup whose proxy is `myNodeGroupProxy':
 myNodeGroupProxy[i].someEntryMethod(\uw{parameters});

 // Invoke `someEntryMethod' on all logical nodes of
 // a NodeGroup whose proxy is `myNodeGroupProxy':
 myNodeGroupProxy.someEntryMethod(\uw{parameters});
\end{alltt}

%In the absence of such a parameter, the call is treated as a broadcast
%to all branches of the NodeGroup of the a \index{nodegroup}nodegroup, i.e. executed by all nodes. 
It is worth restating that when an entry method is
invoked on a particular \index{branch}branch of a \index{nodegroup}nodegroup,
it may be executed by {\em any} PE in that logical node. Thus two invocations of
a single entry method on a particular \index{branch}branch of a
\index{nodegroup}NodeGroup may be executed {\em concurrently} by two
different PEs in the logical node. If this could cause data races in your
program, please consult \S~\ref{sec:nodegroups/exclusive} (below).

%If that method contains code that should be
%executed by only one processor at a time, the method should be flagged
%\index{exclusive}\kw{exclusive} in the interface file. 

\subsection{NodeGroups and \kw{exclusive} Entry Methods}
\label{sec:nodegroups/exclusive}

Node groups may have \index{exclusive}\kw{exclusive} entry methods.  The
execution of an \kw{exclusive} entry method invocation is {\em mutually
exclusive} with those of all other \kw{exclusive} entry methods invocations.
That is, an \kw{exclusive} entry method invocation is not executed on a logical
node as long as another \kw{exclusive} entry method is executing on it.  More
explicitly, if a method \uw{M} of a nodegroup \uw{NG} is marked exclusive, it
means that while an instance of \uw{M} is being executed by a PE within a
logical node, no other PE within that logical node will execute any other {\em
exclusive} methods.
%of that \index{nodegroup}nodegroup \index{branch}branch.  
However, PEs in the logical node may still execute {\em non-exclusive} entry
method invocations.
%on that l \index{branch}branch, however.  of that node group are running on
%the same node.  
An entry method can be marked exclusive by tagging it with the \kw{exclusive}
attribute, as explained in \S~\ref{attributes}.


\subsection{Accessing the Local Branch of a NodeGroup}

The local \index{branch}branch of a \kw{NodeGroup} \uw{NG}, and hence its
member fields and methods, can be accessed through the method \kw{NG*
CProxy\_NG::ckLocalBranch()} of its proxy. Note that accessing data members of
a NodeGroup branch in this manner is {\em not} thread-safe by default, although
you may implement your own mutual exclusion schemes to ensure safety.
%accesses are {\em not} thread-safe by default.  Concurrent invocation of a
%method on a \index{nodegroup}nodegroup by different processors within a node
%may result in unpredictable runtime behavior.  
One way to ensure safety is to use node-level locks, which are described in the
Converse manual.

%For certain applications, node groups can be used in the place of regular
%groups to mitigate messaging overhead when sharing of address spaces between 
%PEs is possible.
%For example, consider a parallel program that does one calculation that can be
%decomposed into several mutually exclusive subcalculations.  The program
%distributes the work amongst all of the processors, the subresults are all
%stored in the local branch of a group, and when the local branch has recieved
%all of its results, it relays everything to one particular processor where the
%subresults are put together into the final result.  When normal groups are
%used, the number of messages sent is $O$(\# of processors).  However, if node
%groups are used, a number of message sends will be replaced by local memory
%accesses if there is more than one processor per node.  Instead, the number of
%messages sent is $O$(\# of nodes).
NodeGroups can be used in a similar way to groups so as to implement lower-level
optimizations such as data sharing and message reduction.



  \section{\charmpp{} Arrays}

\subsection{How do I know which processor a chare array element is running on?}

At any given instant, you can call {\tt CkMyPe()} to find out where
you are. There is no reliable way to tell where another array element is;
even if you could find out at some instant, the element might immediately
migrate somewhere else!

\subsection{Should I use Charm++ Arrays in my program?}

Yes! Most of your computation should happen inside array elements.
Arrays are the main way to automatically balance the load using one of the
load balancers available.

\subsection{How many array elements should I have per processor?}

To do load balancing, you need more than one array element per processor.
To keep the time and space overheads reasonable, you probably don't want
more than a few thousand array elements per processor. The optimal
value depends on the program, but is usually between 10 and 100.
If you come from an MPI background, this may seem like a lot.

\subsection{What does the term reduction refer to?}

You can {\em reduce} a set of data to a single value. For example,
finding the sum of values, where each array element contributes a value
to the final sum. Reductions are supported directly by Charm++ arrays, and some
operations most commonly used are predefined. Other more complicated reductions
can implement if needed.

\subsection{Can I do multiple reductions on an array?}

You {\em can} have several reductions happen one after another; but
you {\em cannot} mix up the execution of two reductions over the same
array. That is, if you want to reduce A, then B, every array element has
to contribute to A, then contribute to B; you cannot have some elements
contribute to B, then contribute to A.

\subsection{Does Charm++ do automatic load balancing without the user asking
for it?}

No. You only get load balancing if you explicitly ask for it by
linking in one or more load balancers with {\em -balancer} link-time
option. If you link in more than one load balancer, you can select
from the available load balancers at runtime with the {\em +balancer}
option. In addition, you can use Metabalancer with the {\em +MetaLB} option to
automatically decide when to invoke the load balancer, as described in
\htmladdnormallink{Load Balancing Strategies}{http://charm.cs.illinois.edu/manuals/html/charm++/7.html\#lbStrategy}
section.

\subsection{What is the migration constructor and why do I need it?}

The migration constructor (a constructor that takes {\tt
  CkMigrateMessage *} as parameter) is invoked when an array element
migrates to a new processor, or when chares or group instances are
restored from a checkpoint. If there is anything you want to do when
you migrate, you could put it here.

A migration constructor need not be defined for any given chare
type. If you try to migrate instances of a chare type that lacks a
migration constructor, the runtime system will abort the program with
an error message.

The migration constructor should not be declared in the {\em .ci} file. Of
course the array element will require also at least one regular constructor so
that it can be created, and these must be declared in the {\em .ci} file.

\subsection{What happens to the old copy of an array element after it migrates?}

After sizing and packing a migrating array element, the array manager
{\tt delete}s
the old copy. As long as all the array element destructors in the non-leaf
nodes of your inheritance hierarchy are {\em virtual destructors}, with
declaration syntax:
\begin{alltt}
class foo : ... \{
  ...
  virtual ~foo(); // <- virtual destructor
\};
\end{alltt}
then everything will get deleted properly.\\
Note that deleting things in a packing pup happens to work for the
current array manager, but {\em WILL NOT} work for checkpointing, debugging,
or any of the (many) other uses for packing puppers we might dream up -
so DON'T DO IT!

\subsection{Is it possible to turn migratability on and off for an individual array
element?}

Yes, call {\em setMigratable(false);} in the constructor.

\subsection{Is it possible to insist that a particular array element gets migrated
at the next {\em AtSync()}?}

No, but a manual migration can be triggered using {\em migrateMe}.


\subsection{When not using {\tt AtSync} for LB, when does the LB start
up? Where is the code that periodically checks if load balancing can be
done?}

If not using {\tt usesAtSync}, the load balancer can start up at
anytime. There is a dummy {\tt AtSync} for each array element which
by default tells the load balancer that it is always ready. The LDBD manager
has a syncer ({\tt LBDB::batsyncer}) which periodically calls {\tt AtSync}
roughly every 1ms to trigger the load balancing (this timeout can be changed
with the {\em +LBPeriod} option). In this load balancing
mode, users have to make sure all migratable objects are always ready to
migrate (e.g. not depending on a global variable which cannot be migrated).

\subsection{Should I use AtSync explicitly, or leave it to the system?}

You almost certainly want to use AtSync directly. In most cases there are
points in the execution where the memory in use by a chare is bigger due to
transitory data, which does not need to be transferred if the migration happens
at predefined points.

%<b>Who calls </b><tt>void staticAtSync(void*)</tt><b>?</b></li>

%<br><tt>staticAtSync</tt> is an internal function for load balancer strategy
%modules, and is called by the load balancing framework. For each object
%participating in migration, you register a local barrier like this:
%<pre>theLbdb->AddLocalBarrierClient(...);</pre>
%and this registers a callback with the LB framework (this can have any
%name, not just <tt>staticAtSync</tt>). When you think it is time to migrate,
%AtSync is called and the local barrier is reached via:
%<pre>theLbdb->AtLocalBarrier(LdBarrierhandle);</pre>
%by all local registered clients. Then the <tt>staticAtSync</tt> callbacks
%will be executed.
%<br>&nbsp;</ol>

  \subsection{Read-only Variables, Messages and Arrays}

Since \charmpp\ does not allow global variables for keeping
programs portable across a wide range of machines, it provides a special
mechanism for sharing data amongst all objects. {\it Read-only}
variables, messages and arrays are used to share information that 
is obtained only after the program begins execution and does not
change after they are initialized in the dynamic scope of 
{\tt main::main()} function. They
can be accessed from any \index{chare}chare on any processor as ``global''
variables. Large data structures containing pointers can be made
available as read-only variables using read-only messages or
read-only arrays. Read-only variables, messages and arrays can
be used just like local variables for each processor, but the user has
to allocate space for read-only messages using \kw{new} to create
the message in the {\tt main} function of the \kw{mainchare}. 

Read-only variables, messages, and arrays are declared by using the type
modifier \kw{readonly}, which is similar to \kw{const} in
\CC. Read-only data is specified in the {\tt .ci} file (the interface
file) as: 

\begin{tabbing}
~~~~ \=~~~~ \=~~~~ \=~~~~ \=~~~~ \=~~~~ \=~~~~ \=~~~~ \=~~~~ \=~~~~ \kill
\> \kw{readonly} \uw{Type} {\it ReadonlyVarName};
\end{tabbing}

The variable {\it ReadonlyVarName} is declared to be a read-only
variable of type \uw{Type}. \uw{Type} must be a single token and not a
type expression.

\begin{tabbing}
~~~~ \=~~~~ \=~~~~ \=~~~~ \=~~~~ \=~~~~ \=~~~~ \=~~~~ \=~~~~ \=~~~~ \kill
\> \kw{readonly} \uw{MessageType} *{\it ReadonlyMsgName};
\end{tabbing}

The variable {\it ReadonlyMsgName} is declared to be a read-only
message of type \uw{MessageType}. Pointers are not allowed to be
readonly variables unless they are pointers to message types. In this
case, the message will be initialized on every processor.

\begin{tabbing}
~~~~ \=~~~~ \=~~~~ \=~~~~ \=~~~~ \=~~~~ \=~~~~ \=~~~~ \=~~~~ \=~~~~ \kill
\> \kw{readonly} \uw{Type} {\it ReadonlyArrayName} [{\it arraysize}];
\end{tabbing}

The variable {\it ReadonlyArrayName} is declared to be a read-only
array of type \uw{Type}. \uw{Type} must be a single token and not a
type expression.

Read-only variables, messages and arrays must be declared either as
global or as public class static data, and these declarations have the
usual form:

\begin{tabbing}
~~~~ \=~~~~ \=~~~~ \=~~~~ \=~~~~ \=~~~~ \=~~~~ \=~~~~ \=~~~~ \=~~~~ \kill
\> \uw{Type} {\it ReadonlyVarName}; \\
\> \uw{MessageType} *{\it ReadonlyMsgName}; \\
\> \uw{Type} {\it ReadonlyArrayName} [{\it arraysize}];
\end{tabbing}

Similar declarations preceded by \kw{extern} would appear in the {\tt
.h} file. 

{\it Note:}  The current \charmpp\ compiler cannot prevent
assignments to read-only variables.  The user must make sure that no
assignments occur in the program.




    
  \subsection{Quiescence Detection}

In \charmpp, \index{quiescence}quiescence is defined as the state in which no
processor is executing an entry point, and no messages are awaiting processing.

\charmpp\ provides two facilities for detecting quiescence: \kw{CkStartQd} and
\kw{CkWaitQd}. \index{CkStartQd} \index{CkWaitQd}

\kw{CkStartQd} registers with the system a callback that should be made the
next time \index{quiescence}quiescence is detected.  \kw{CkStartQd} takes two
parameters: an index corresponding to the entry function that is to be called,
and a handle to the chare on which that entry function should be called.  The
syntax of this call looks like this:

\begin{tabbing}
~~~~ \=~~~~ \=~~~~ \=~~~~ \=~~~~ \=~~~~ \=~~~~ \=~~~~ \=~~~~ \=~~~~ \kill
\> \kw{CkStartQd}(\kw{int} {\it Index}, \kw{CkChareID} {\it chareID});
\end{tabbing}

To retrieve the corresponding index of a particular \index{entry method}entry
method, you must use a static method contained within the
\index{CProxy}\kw{CProxy} object corresponding to the \index{chare}chare
containing that entry method.  The syntax of this call is as follows:

\begin{tabbing}
~~~~ \=~~~~ \=~~~~ \=~~~~ \=~~~~ \=~~~~ \=~~~~ \=~~~~ \=~~~~ \=~~~~ \kill
\kw{CProxy}\_\uw{ChareName}::\kw{ckIdx}\_\uw{EntryMethod}(\uw{Msg}
*{\it Message});
\end{tabbing}

where {\it chareID} is the name of the chare identifier of the chare containing
the desired entry method, \uw{EntryMethod} is the name of that entry method,
and {\it Message} is a pointer to the kind of message that the desired entry
method takes as a parameter. To make this look a little cleaner, we have
provided a simple macro called \kw{EntryIndex}, which can be used in the place
of this convoluted looking static method call.
\kw{EntryIndex}\index{EntryIndex} takes as parameters the type of chare in
which the entry method is located, the name of the entry method itself, and the
type of message that the entry method takes as a parameter. For example:

\begin{tabbing}
~~~~ \=~~~~ \=~~~~ \=~~~~ \=~~~~ \=~~~~ \=~~~~ \=~~~~ \=~~~~ \=~~~~ \kill
\> \kw{EntryIndex}(\uw{ChareName}, \uw{EntryName}, \uw{MsgName});
\end{tabbing}

Note that ChareName, EntryName, and MsgName are {\bf NOT} variables or
constants. This is text that the preprocessor uses to fill in portions of the
previously mentioned static method call.  Additionally, this macro method will
not work with templated chares (refer to Section ~\ref{inheritance and
templates} for details on templated chares).

\index{CkWaitQd}\kw{CkWaitQd}, however, does not register a callback.  Rather,
\kw{CkWaitQd} blocks and does not return until \index{quiescence}quiescence is
detected.  It takes no parameters and returns no value.  A call to
\kw{CkWaitQd} simply looks like this: 

\begin{tabbing}
~~~~ \=~~~~ \=~~~~ \=~~~~ \=~~~~ \=~~~~ \=~~~~ \=~~~~ \=~~~~ \=~~~~ \kill
\> \kw{CkWaitQd}();
\end{tabbing}

Keep in mind that \kw{CkWaitQd} should only be called from threaded
\index{entry method}entry methods because a call to \kw{CkWaitQd} suspends the
current thread of execution, and if it were called outside of a threaded entry
method it would suspend the main thread of execution of the processor from
which \kw{CkWaitQd} was called and the entire program would come to a grinding
halt on that processor.
    
  \subsection{Terminal I/O}

\index{input/output}
\charmpp\ provides both C and \CC\ style methods of doing terminal I/O.  

In place of C-style printf and scanf, \charmpp\ provides
\kw{CkPrintf} and \kw{CkScanf}.  These functions have
interfaces that are identical to their C counterparts, but there are some
differences in their behavior that should be mentioned.

\function{int CkPrintf(format [, arg]*)} \index{CkPrintf} \index{input/output}
\desc{This call is used for atomic terminal output. Its usage is similar to
\texttt{printf} in C.  However, \kw{CkPrintf} has some special properties
that make it more suited for parallel programming on networks of
workstations.  \kw{CkPrintf} routes all terminal output to the \kw{charmrun},
which is running on the host computer.  So, if a
\index{chare}chare on processor 3 makes a call to \kw{CkPrintf}, that call
puts the output in a TCP message and sends it to host
computer where it will be displayed.  This message passing is an asynchronous
send, meaning that the call to \kw{CkPrintf} returns immediately after the
message has been sent, and most likely before the message has actually
been received, processed, and displayed. \footnote{Because of
communication latencies, the following scenario is actually possible:
Chare 1 does a \kw{Ckprintf} from processor 1, then creates chare 2 on
processor 2.  After chare 2's creation, it calls \kw{CkPrintf}, and the
message from chare 2 is displayed before the one from chare 1.}
}

\function{void CkError(format [, arg]*))} \index{CkError} \index{input/output} 
\desc{Like \kw{CkPrintf}, but used to print error messages on \texttt{stderr}.}

\function{int CkScanf(format [, arg]*)} \index{CkScanf} \index{input/output}
\desc{This call is used for atomic terminal input. Its usage is similar to
{\tt scanf} in C.  A call to \kw{CkScanf}, unlike \kw{CkPrintf},
blocks all execution on the processor it is called from, and returns
only after all input has been retrieved.
}

For \CC\ style stream-based I/O, \charmpp\ offers \kw{ckin},
\kw{ckout}, and \kw{ckerr} in the place of cin, cout, and cerr.  The
\CC\ streams and their \charmpp\ equivalents are related in the same
manner as printf and scanf are to \kw{CkPrintf} and \kw{CkScanf}.  The
\charmpp\ streams are all used through the same interface as the \CC\ 
streams, and all behave in a slightly different way, just like C-style
I/O. \kw{ckout} and \kw{ckerr} both have the same idiosyncratic
behavior as \kw{CkPrintf}, and \kw{ckin} behaves in the same way as
\kw{CkScanf}.

  \subsection{\kw{initnode} and \kw{initproc} routines}

\index{initcall}
\label{initcall}
Some registration routines need be executed exactly once
before the computation begins. You may choose to 
declare a regular  \CC\ subroutine \kw{initnode} in the .ci file
to ask \charmpp to execute the routine exactly once on {\em every node} 
before the computation begins, or to declare a regular  \CC\ subroutine 
\kw{initproc} to be executed exactly once on {\em every processor}.

\begin{alltt}
module foo \{
    initnode void fooNodeInit(void);
    initproc void fooProcInit(void);
    chare bar \{
        ...
        initnode void barNodeInit(void);
        initproc void barProcInit(void);
    \};
\};
\end{alltt}

This code will execute the routines \uw{fooNodeInit} and static 
\uw{bar::barNodeInit} once on every node and \uw{fooProcInit}
and \uw{bar::barProcInit} on every processor before the main computation 
starts.
Initnode calls are always executed before initproc calls.
Both init calls (declared as static member function) can be used in chare, 
group or chare arrays.

Note that these routines should only do registration, not computation since
Charm run-time initialization does not start yet ---
use a mainchare instead, which gets executed on only processor 0,
to begin the computation.  Initcall routines are typically
used to do special registrations and global variable setup
before the computation actually begins.


\subsection{Other Calls}

\label{other Charm++ calls}

The following calls provide commonly needed functions.

\function{double CkCpuTimer()} \index{CkCpuTimer} \index{timers}
\desc{Returns the current value of the system timer in seconds. The system
timer is started when the program begins execution. This timer measures process
time (user and system).}

\function{double CkTimer()} \index{CkTimer} \index{timers}
\desc{This is an alias for either \kw{CkWallTimer} on dedicated machines (such as ASCI Red) or 
\kw{CkCpuTimer} for machines with multiple user processes per CPU (such as workstation cluster.)}

    

\newpage
\section{Inheritance and Templates in Charm++}
\label{inheritance and templates}

\charmpp\ 5.0 supports inheritance among \charmpp\ objects such as
chares, groups, and messages. This, along with facilities for generic
programming using \CC\ style templates for \charmpp\ objects, is a
major enhancement over the previous version of \charmpp.

\subsection{Chare Inheritance}
\index{inheritance}
Chare inheritance makes it possible to remotely invoke methods of a base
chare \index{base chare} from a proxy of a derived
chare.\index{derived chare} Suppose a base chare is of type 
\uw{BaseChare}, then the derived chare of type \uw{DerivedChare} needs to be
declared in the \charmpp\ interface file to be explicitly derived from
\uw{BaseChare}. Thus, the constructs in the {\tt .ci} file should look like:

\begin{verbatim}
  chare BaseChare {
    BaseChare(someMessage *);
    baseMethod(void);
    ...
  }
  chare DerivedChare : BaseChare {
    DerivedChare(otherMessage *);
    derivedMethod(void);
    ...
  }
\end{verbatim}

Note that the access specifier \kw{public} is omitted, because \charmpp\
interface translator only needs to know about the public inheritance,
and thus \kw{public} is implicit. A Chare can inherit privately from other
classes too, but the \charmpp\ interface translator does not need to know
about it, because it generates support classes ({\em proxies}) to remotely
invoke only \kw{public} methods.

The class definitions of both these chares should look like:

\begin{verbatim}
  class BaseChare : public Chare {
    // private or protected data
    public:
      BaseChare(someMessage *);
      baseMethod(void);
  };
  class DerivedChare : public BaseChare {
    // private or protected data
    public:
      DerivedChare(otherMessage *);
      derivedMethod(void);
  };
\end{verbatim}

Now, it is possible to create a derived chare, and invoke methods of base
chare from it, or to assign a derived chare proxy to a base chare proxy
as shown below:

\begin{verbatim}
  ...
  otherMessage *msg = new otherMessage();
  CProxy_DerivedChare *pd = new CProxy_DerivedChare(msg);
  pd->baseMethod();     // OK
  pd->derivedMethod();  // OK
  ...
  Cproxy_BaseChare *pb = pd;
  pb->baseMethod();    // OK
  pb->derivedMethod(); // ERROR
\end{verbatim}

Note that \CC\ calls the default constructor \index{default
constructor} of the base class from any
constructor for the derived class where base class constructor is not
called explicitly. Therefore, one should always provide a default constructor
for the base class to avoid explicitly calling another base class constructor.

Multiple inheritance \index{multiple inheritance} is also allowed for
Chares and Groups. However, one should make sure that each of the base
classes inherit ``virtually'' from \kw{Chare} or \kw{Group}, so that a
single copy of \kw{Chare} or \kw{Group} exists for each multiply
derived class.

One may have {\em virtual} entry methods \index{virtual} in base chare
classes. They are specified by the entry attribute \kw{virtual} in the
interface file. Also, such a method could be pure virtual \index{pure
virtual} (where implementation of this method is not given.) It is
noted with the usual \CC\ syntax of ``{\tt =0}'' even in the interface
file. Base classes with pure virtual methods are called {\em abstract}
base classes \index{abstract base class} in OO-lingo. Note that,
though instances of abstract base chares cannot be created, a proxy
objects of such a type can be created by constructing it with the
chareID of the concrete derived chare. Invoking methods on this proxy
object would invoke those methods remotely on the concrete derived
class instance.

\subsection{Inheritance for Messages}
\index{message inheritance}

Similar to Chares, messages can also be derived from base messages. One needs
to specify this in the \charmpp\ interface file similar to the Chare
inheritance specification (that is, without the \kw{public} access specifier.)
Message inheritance makes it possible to send a message of derived type to
the method expecting a base class message.

\subsection{Generic Programming Using Templates}
\index{templates}

One can write ``templated'' code for Chares, Groups, Messages and other 
\charmpp\  entities using familiar \CC\ template syntax (almost). The
\charmpp\ interface translator now recognizes most of the \CC\
templates syntax, including a variety of formal parameters, default
parameters, etc. However, not all \CC\ compilers currently recognize
templates in ANSI drafts, therefore the code generated by \charmpp\
for templates may not be acceptable to some current \CC\
compilers\footnote{
Most modern \CC\ compilers belong to one of the two camps. One that 
supports Borland style template instantiation, and the other that
supports AT\&T Cfront style template instantiation. In the first,
code is generated for the source file where the instantiation is seen.
GNU \CC\ falls in this category.
In the second, which template is to be instantiated, and
where the templated code is seen is noted in a separate area 
(typically a local directory), and then just before linking all the
template instantiations are generated. Solaris CC 5.0 belongs to this
category. For templates to work for compilers in the first category
such as for GNU \CC\ all the templated code needs to be visible to the
compiler at the point of instantiation, that is, while compiling the 
source file containing the template instantiation. For a variety of
reasons, \charmpp\ interface translator cannot generate all the templated
code in the declarations file {\tt *.decl.h}, which is included in the
source file where templates are instantiated. Thus, for \charmpp\ generated
templates to work for GNU \CC\ even the definitions file {\tt *.def.h}
should be included in the \CC\ source file. However, this file may
contain other definitions apart from templates that will be duplicated if
the same file is included in more than one source files. To alleviate 
this problem, we have to do a little trick. Fortunately, this trick works for
compilers supporting both Borland-style and Cfront-style template 
instantiation, therefore, code using this trick will be portable. The trick
is to include {\tt *.def.h} with a preprocessor symbol 
\kw{CK\_TEMPLATES\_ONLY} defined, whenever templates defined in an
\kw{extern} module are instantiated. If your
interface file does not contain template declarations or definitions,
you need not bother about including {\tt *.def.h} for \kw{extern} modules.
For example, if module {\tt stlib} contains template definitions, 
that you may want to instantiate in another module called {\tt pgm},
then {\tt pgm.C} should include {\tt stlib.def.h} 
with \kw{CK\_TEMPLATES\_ONLY} defined. Of course, {\tt stlib.decl.h} needs
to be included at the top of {\tt pgm.C}. 
}. 

The \charmpp\ interface file should contain the template
definitions as well as the instantiation. For example, if a message
class \uw{TMessage} is templated with a formal type parameter 
\uw{DType}, then every instantiation of \uw{TMessage} should be specified
in the \charmpp\ interface file. An example will illustrate this better:
\index{template}

\begin{verbatim}
  template <class DType=int, int N=3> message TMessage;
  message TMessage<>; // same as TMessage<int,3>
  message TMessage<double>; // same as TMessage<double, 3>
  message TMessage<UserType, 1>;
\end{verbatim}

Note the use of default template parameters. It is not necessary for
template definitions and template instantiations to be part of the
same module.  Thus, templates could be defined in one module, and
could be instantiated in another module \index{module}, as long as the
module defining a template is imported into the other module using the
\kw{extern module} construct. Thus it is possible to build a standard
\charmpp\ template library. Here we give a flavor of possibilities:

\begin{verbatim}
module SCTL {
  template <class dtype> message  Singleton;
  template <class dtype> group Reducer {
    entry Reducer(void);
    entry void submit(Singleton<dtype> *);
  }
  template <class dtype> chare ReductionClient {
    entry void recvResult(Singleton<dtype> *);
  }
};

module User {
  extern module SCTL;
  message Singleton<int>;
  group Reducer<int>;
  chare RedcutionClient<int>;
  chare UserClient : ReductionClient<int> {
    entry UserClient(void);
  }
};
\end{verbatim}

The \uw{Singleton} message is a template for storing one element of
any \uw{dtype}. The \uw{Reducer} is a group template for a
spanning-tree reduction, which is started by submitting data to the
local branch. It also contains a public method to register the
\uw{ReductionClient} (or any of its derived types), which acts as a
callback to receive results of a reduction.


\newpage
\appendix

\section{Compiling, Running and Debugging Charm++/Converse Programs}


\section{\charmpp\ Keywords}
The following is the complete list of keywords in \charmpp:

\begin{enumerate}
\item array
\item char
\item chare
\item class
\item double
\item entry
\item exclusive
\item extern
\item float
\item group
\item int
\item long
\item mainchare
\item mainmodule
\item message
\item module
\item nodegroup
\item packed
\item readonly
\item short
\item stacksize
\item sync 
\item template
\item thisgroup
\item thishandle
\item threaded
\item unsigned 
\item varsize
\item virtual
\item void
\end{enumerate}

The following is the complete list of system operators and calls in \charmpp.
Currently no user-defined functions may have one of these names.

\begin{enumerate}
\item CkAllocBuffer
\item CkAllocMsg
\item CkAllocSysMsg
\item CkArgMsg
\item CkArrayID
\item CkBroadcastMsgBranch
\item CkBroadcastMsgNodeBranch
\item CkChareID
\item CkCopyMsg
\item CkCreateChare
\item CkCreateGroup
\item CkCreateNodeGroup
\item CkError
\item CkExit
\item CkFreeMsg
\item CkFreeSysMsg
\item CkGetChareID
\item CkGetGroupID
\item CkGetNodeGroupID(void);
\item CkGetRefNum
\item CkGetSrcNode
\item CkGetSrcPe
\item CkLocalBranch(int groupID);
\item CkLocalNodeBranch(int groupID);
\item CkMyPe
\item CkMyRank
\item CkMyNode
\item CkNodeFirst
\item CkNodeOf
\item CkNodeSize
\item CkNumNodes
\item CkNumPes
\item CkPrintf
\item CkPriorityPtr
\item CkQdMsg
\item CkRankOf
\item CkRegisterChare
\item CkRegisterEp
\item CkRegisterMainChare
\item CkRegisterMsg
\item CkRegisterReadonly
\item CkRegisterReadonlyMsg
\item CkRemoteCall
\item CkRemoteBranchCall
\item CkRemoteNodeBranchCall
\item CkScanf
\item CkSendMsg
\item CkSendMsgBranch
\item CkSendMsgNodeBranch
\item CkSendToFuture
\item CkSetQueueing
\item CkSetRefNum
\item CkStartQD
\item CkTimer
\item CkWaitQd
\item CkWallTimer
\end{enumerate}


\section{Syntax Changes from \charmpp\ 4.9}

The following changes are required to make older \charmpp\ 4.9
programs run with the new translator and runtime system in
\charmpp\ 5.0.

\begin{itemize}

\item Replace all references to {\tt *.top.h} and {\tt *.bot.h} to
{\tt *.decl.h} and {\tt *.def.h} respectively. This should be done in
Makefile, and everywhere these two types of files are included.

\item Change all X.ci files to include a top-level enclosure of module X {...}.

\item Change \kw{chare} \uw{main} to \kw{mainchare} \uw{main}.

\item Replace \kw{chare\_object} by \kw{Chare}.

\item Replace \kw{groupmember} by \kw{Group}.

\item Replace \kw{comm\_object} by \uw{CMessage\_<msgName>} in all
message declarations in {\tt *.h} files. 

\item Remove all \kw{operator new} methods of messages.

\item Remove \kw{MsgIndex(..)} parameter to \kw{new} for message allocation.

\item Remove all the empty messages (used only for triggering
computations.) from {\tt *.h}, {\tt *.ci}, {\tt *.C} files. All the
entry methods that take these empty messages as parameters should be
made methods with \kw{void} parameter. This should be done in all {\tt
*.ci}, {\tt *.h}, and {\tt *.C} files.  In these methods, there may be
a {\tt delete msg}. Remove that. 

\item Check for \kw{mainhandle} in the source. If it is there, declare
it as a \kw{readonly} variable in {\tt .ci} file, and initialize it in
the \kw{mainchare}'s constructor, so that it is available to all the
processors during the run.

\item All the \kw{packmessage} declarations in {\tt *.ci} files should
be changed to \kw{message [packed]}. 

\item All \kw{CPrintf}, \kw{CScanf}, and \kw{CError} should be changed
to \kw{CkPrintf}, \kw{CkScanf}, and \kw{CkError}. 

\item All \kw{CharmExit} should be changed to \kw{CkExit}.

\item Replace \kw{CMyPe} by \kw{CkMyPe}. Replace \kw{CNumPes} by \kw{CkNumPes}.

\item Change \kw{ChareIDType} to \kw{CkChareID}.

\item Change signature of \uw{M::pack} and \uw{M::unpack} for all
messages \uw{M} in {\tt *.h}, {\tt *.C} to\\
\verb+static void *pack(M* msg)+\\
and\\
\verb+static M* unpack(void *buf)+\\
and change the code
accordingly. This is a significant change because {\tt pack} and {\tt unpack}
used to be instance methods, and now they are class static
methods.  Avenues of optimizations open with this change, but one need
not explore those in the interest of time immediately. Further, one
should make sure that the performance of the new scheme is at least as
good as the old one. 

\item Replace all \kw{new\_group} by \uw{CProxy\_<grpName>::ckNew}.

\item Replace \kw{new\_chare2} by proxy creation.

\item Replace \kw{CSendMsg} by temporary proxy creation based on
ChareID and invoking appropriate method on it. One optimization is to
create a proxy immediately after a ChareID is received, and reusing it
everytime.

\item Similarly replace \kw{CSendMsgBranch} by temporary proxy
creation based on the groupID, and invoking appropriate method on it,
with processor number as the second parameter. Once again, there is an
opportunity for optimization here. 

\item Similarly replace \kw{CBroadcastMsgBranch} by temporary proxy
creation based on the groupID, and invoking appropriate method on it,
without any second parameter. Once again, there is an opportunity for
optimization here. 

\item Replace \kw{CStartQuiescence} by \kw{CkStartQD}.

\item Replace \kw{GetEntryPtr} by appropriate static method
(\kw{ckIdx\_*}) calls. 

\item Replace \kw{CLocalBranch} macros by \kw{ckLocalBranch} instance
method on temporarily created proxy. 

\item Change \kw{CPriorityPtr} to \kw{CkPriorityPtr}, also cast it
explicitly to {\tt (int *)}. 

\item Replace \kw{QuiescenceMessage} by \kw{CkQDMsg}. Remove all
extern declarations of \kw{CkQDMsg} from {\tt *.ci} files.

\end{itemize}


\charmpp\ is based on the message-driven parallel programming paradigm. In
contrast to many other approaches, \charmpp\ programmers encode entry points to
their parallel objects, but do not explicitly wait (i.e. block) on the runtime
to indicate completion of posted `receive' requests. Thus, a \charmpp\ object's
overall flow of control can end up fragmented across a number of separate
methods, obscuring the sequence in which code is expected to
execute. Furthermore, there are often constraints on when different pieces of
code should execute relative to one another, related to data and
synchronization dependencies.

Consider one way of expressing these constraints using flags, buffers, and
counters, as in the following example:
%
%\begin{figure}[ht]
\begin{center}
\begin{alltt}
// in .ci file
chare ComputeObject \{
  entry void ComputeObject();
  entry void startStep();
  entry void firstInput(Input i);
  entry void secondInput(Input j);
\};

// in C++ file
class ComputeObject : public CBase_ComputeObject \{
  int   expectedMessageCount;
  Input first, second;

public:
  ComputeObject() \{
    startStep();
  \}
  void startStep() \{
    expectedMessageCount = 2;
  \}

  void firstInput(Input i) \{
    first = i;
    if (\verb|--expectedMessageCount == 0|)
      computeInteractions(first, second);
    \}
  void recv_second(Input j) \{
    second = j;
    if (\verb|--expectedMessageCount == 0|)
      computeInteractions(first, second);
  \}

  void computeInteractions(Input a, Input b) \{
    // do computations using a and b
    . . .
    // send off results
    . . .
    // reset for next step
    startStep();
  \}
\};
\end{alltt}
\end{center}
%\caption{Compute Object in a Molecular Dynamics Application}
%\label{figchareexample}
%\end{figure}
In each step, this object expects pairs of messages, and waits to process the
incoming data until it has both of them. This sequencing is encoded across 4
different functions, which in real code could be much larger and more numerous,
resulting in a spaghetti-code mess.

Instead, it would be preferable to express this flow of control using
structured constructs, such as loops. \charmpp\ provides such constructs for
structured control flow across an object's entry methods in a notation called
Structured Dagger. The basic constructs of Structured Dagger (SDAG) provide for
\emph{program-order execution} of the entry methods and code blocks that they
define. These definitions appear in the {\tt .ci} file definition of the
enclosing chare class as a `body' of an entry method following its signature.

The most basic construct in SDAG is the {\tt serial} (aka the {\tt atomic}) block.
Serial blocks
contain sequential \CC code.  They're also called atomic because the code within
them executes without returning control to the \charmpp\ runtime scheduler, and
thus avoiding interruption from incoming messages. The keywords atomic and serial
are synonymous, and you can find example programs that use atomic. However, we
recommend the use of serial and are considering the deprecation of the atomic keyword.
Typically serial blocks hold
the code that actually deals with incoming messages in a {\tt when} statement,
or to do local operations before a message is sent or after it's received. The
earlier example can be adapted to use serial blocks as follows:
\begin{center}
\begin{alltt}
// in .ci file
chare ComputeObject \{
  entry void ComputeObject();
  entry void startStep();
  entry void firstInput(Input i) \{
    serial \{
      first = i;
      if (\verb|--expectedMessageCount == 0|)
        computeInteractions(first, second);
    \}
  \};
  entry void secondInput(Input j) \{
    serial \{
      second = j;
      if (\verb|--expectedMessageCount == 0|)
        computeInteractions(first, second);
    \}
  \};
\};

// in C++ file
class ComputeObject : public CBase\_ComputeObject \{
  ComputeObject\_SDAG\_CODE
  int   expectedMessageCount;
  Input first, second;

public:
  ComputeObject() \{
    startStep();
  \}
  void startStep() \{
    expectedMessageCount = 2;
  \}

  void computeInteractions(Input a, Input b) \{
    // do computations using a and b
    . . .
    // send off results
    . . .
    // reset for next step
    startStep();
  \}
\};
\end{alltt}
\end{center}
Note that chare classes containing SDAG code must include a few additional declarations
in addition to inheriting from their {\tt CBase\_Foo} class, by incorporating the
{\tt Foo\_SDAG\_CODE} generated-code macro in the class.

Serial blocks can also specify a textual `label' that will appear in traces, as
follows:
\begin{center}
\begin{alltt}
  entry void firstInput(Input i) \{
    serial "process first" \{
      first = i;
      if (\verb|--expectedMessageCount == 0|)
        computeInteractions(first, second);
    \}
  \};
\end{alltt}
\end{center}

In order to control the sequence in which entry methods are processed, SDAG
provides the {\tt when} construct. These statements, also called triggers,
indicate that we expect an incoming message of a particular type, and provide
code to handle that message when it arrives. From the perspective of a chare
object reaching a {\tt when} statement, it is effectively a `blocking
receive.'

Entry methods defined by a {\tt when} are
not executed immediately when a message targeting them is delivered, but
instead are held until control flow in the chare reaches a corresponding {\tt
  when} clause. Conversely, when control flow reaches a {\tt when} clause, the
generated code checks whether a corresponding message has arrived: if one has
arrived, it is processed; otherwise, control is returned to the
\charmpp\ scheduler. 

The use of {\tt when} substantially simplifies the example from above:
\begin{center}
\begin{alltt}
// in .ci file
chare ComputeObject \{
  entry void ComputeObject();
  entry void startStep() \{
    when firstInput(Input first)
      when secondInput(Input second)
        serial \{
          computeInteractions(first, second);
        \}
  \};
  entry void firstInput(Input i);
  entry void secondInput(Input j);
\};

// in C++ file
class ComputeObject : public CBase_ComputeObject \{
  ComputeObject_SDAG_CODE

public:
  ComputeObject() \{
    startStep();
  \}

  void computeInteractions(Input a, Input b) \{
    // do computations using a and b
    . . .
    // send off results
    . . .
    // reset for next step
    startStep();
  \}
\};
\end{alltt}
\end{center}
Like an {\tt if} or {\tt while} in C code, each {\tt when} clause has a body
made up of the statement or block following it. The variables declared as
arguments to the entry method triggering the when are available in the scope of
the body. By using the sequenced execution of SDAG code and the availability of
parameters to when-defined entry methods in their bodies, the counter {\tt
  expectedMessageCount} and the intermediate copies of the received input are
eliminated. Note that the entry methods {\tt firstInput} and {\tt secondInput}
are still declared in the {\tt .ci} file, but their definition is in the SDAG
code. The interface translator generates code to handle buffering and
triggering them appropriately.

For simplicity, {\tt when} constructs can also specify multiple expected entry
methods that all feed into a single body, by separating their prototypes with
commas:
\begin{center}
\begin{alltt}
entry void startStep() \{
  when firstInput(Input first),
       secondInput(Input second)
    serial \{
      computeInteractions(first, second);
    \}
\};
\end{alltt}
\end{center}

A single entry method is allowed to appear in more than one {\tt when} statement.
If only one of those {\tt when} statements has been triggered when the runtime
delivers a message to that entry method, that {\tt when} statement is guaranteed
to process it. If there is no trigger waiting for that entry method, then the
next corresponding {\tt when} to be reached will receive that message. If there is
more than one {\tt when} waiting on that method, which one will receive it is not
specified, and should not be relied upon. For an example of multiple {\tt when}
statements handling the same entry method without reaching the unspecified case,
see the CharmLU benchmark.

To more finely control the correspondence between incoming messages and {\tt when}
clauses associated with the target entry method, SDAG supports \emph{matching} on
reference numbers. Matching is typically used to denote an iteration of a program
that executes asynchronously (without any sort of barrier or other synchronization
between steps) or a particular piece of the problem being solved.
Matching is requested by placing an expression denoting the desired reference
number in square brackets between the entry method name and its parameter list.
For parameter marshalled entry methods, the reference number expression will be
compared for equality with the entry method's first argument. For entry methods
that accept an explicit message (\S~\ref{messages}), the reference number on
the message can be set by calling the function
\verb|CkSetRefNum(void *msg, CMK_REFNUM_TYPE ref)|.
Matching is used in the loop example below, and in
\examplereffile{jacobi2d-sdag/jacobi2d.ci}. Multiple {\tt when} triggers for
an entry method with different matching reference numbers will not conflict - each
will receive only corresponding messages.

SDAG supports the {\tt for} and {\tt while} loop constructs mostly as if they
appeared in plain C or C++ code. In the running example, {\tt
  computeInteractions()} calls {\tt startStep()} when it is finished to start
the next step. Instead of this arrangement, the loop structure can be made
explicit:
\begin{center}
\begin{alltt}
// in .ci file
chare ComputeObject \{
  entry void ComputeObject();
  entry void runForever() \{
    while(true) \{
      when firstInput(Input first),
           secondInput(Input second) serial \{
          computeInteractions(first, second);
      \}
    \}
  \};
  entry void firstInput(Input i);
  entry void secondInput(Input j);
\};

// in C++ file
class ComputeObject : public CBase_ComputeObject \{
  ComputeObject_SDAG_CODE

public:
  ComputeObject() \{
    runForever();
  \}

  void computeInteractions(Input a, Input b) \{
    // do computations using a and b
    . . .
    // send off results
    . . .
  \}
\};
\end{alltt}
\end{center}
If this code should instead run for a fixed number of iterations, we can
instead use a for loop:
\begin{center}
\begin{alltt}
// in .ci file
chare ComputeObject \{
  entry void ComputeObject();
  entry void runForever() \{
    for(iter = 0; iter < n; ++iter) \{
      // Match to only accept inputs for the current iteration
      when firstInput[iter](int a, Input first),
           secondInput[iter](int b, Input second) serial \{
        computeInteractions(first, second);
      \}
    \}
  \};
  entry void firstInput(int a, Input i);
  entry void secondInput(int b, Input j);
\};

// in C++ file
class ComputeObject : public CBase_ComputeObject \{
  ComputeObject_SDAG_CODE
  int n, iter;

public:
  ComputeObject() \{
    n = 10;
    runForever();
  \}

  void computeInteractions(Input a, Input b) \{
    // do computations using a and b
    . . .
    // send off results
    . . .
  \}
\};
\end{alltt}
\end{center}
Note that {\tt int iter;} is declared in the chare's class definition and not
in the {\tt .ci} file. This is necessary because the \charmpp\ interface
translator does not fully parse the declarations in the {\tt for} loop header,
because of the inherent complexities of C++.

SDAG also supports conditional execution of statements and blocks with {\tt if}
statements. The syntax of SDAG {\tt if} statements matches that of C and
C++. However, if one encounters a syntax error on correct-looking code in a
loop or conditional statement, try assigning the condition expression to a
boolean variable in a serial block preceding the statement and then testing that
boolean's value. This can be necessary because of the complexity of parsing C++
code.

In cases where multiple tasks must be processed before execution continues, but
with no dependencies or interactions among them, SDAG provides the {\tt
  overlap} construct. Overlap blocks contain a series of SDAG statements within
them which can occur in any order. Commonly these blocks are used to hold a
series of {\tt when} triggers which can be received and processed in any
order. Flow of control doesn't leave the overlap block until all the statements
within it have been processed.

In the running example, suppose each input needs to be preprocessed independently
before the call to {\tt computeInteractions}. Since we don't care which order
they get processed in, and want it to happen as soon as possible, we can apply
{\tt overlap}:
\begin{center}
\begin{alltt}
// in .ci file
chare ComputeObject \{
  entry void ComputeObject();
  entry void startStep() \{
    overlap \{
      when firstInput(Input i)
        serial \{ first = preprocess(i); \}
      when secondInput(Input j)
        serial \{ second = preprocess(j); \}
     \}
     serial \{
       computeInteractions(first, second);
     \}
  \};
  entry void firstInput(Input i);
  entry void secondInput(Input j);
\};

// in C++ file
class ComputeObject : public CBase_ComputeObject \{
  ComputeObject_SDAG_CODE

public:
  ComputeObject() \{
    startStep();
  \}

  void computeInteractions(Input a, Input b) \{
    // do computations using a and b
    . . .
    // send off results
    . . .
    // reset for next step
    startStep();
  \}
\};
\end{alltt}
\end{center}

Another construct offered by SDAG is the {\tt forall} loop. These loops are
used when the iterations of a loop can be performed independently and in any
order. This is in contrast to a regular {\tt for} loop, in which each iteration
is executed sequentially. The {\tt forall} loop can be seen as an overlap with
an indexed set of otherwise identical statements in the body. Its syntax is 
\begin{center}
\begin{alltt}
forall [IDENT] (MIN:MAX,STRIDE) BODY
\end{alltt}
\end{center}
The range from MIN to MAX is inclusive. Its use is demonstrated through
distributed parallel matrix-matrix multiply shown in
\examplereffile{matmul/matmul.ci}

\subsection{The \texttt{case} Statement}

The \texttt{case} statement in SDAG expresses a disjunction over a set of
\texttt{when} clauses. In other words, if it is known that one dependency out
of a set will be satisfied, but which one is not known, this statement allows
the set to be specified and will execute the corresponding block based on which
dependency ends up being fulfilled.

The following is a basic example of the \texttt{case} statement. Note that the
trigger \texttt{b(), d()} will only be fulfilled if both \texttt{b()} and
\texttt{d()} arrive. If only one arrives, then it will partially match, and the
runtime will not ``commit'' to this branch until the second arrives. If another
dependency fully matches, the partial match will be ignored and can be used to
trigger another \texttt{when} later in the execution.

\begin{verbatim}
case {
  when a() { }
  when b(), d() { }
  when c() { }
}
\end{verbatim}

A full example of the \texttt{case} statement can be found
\testreffile{sdag/case/caseTest.ci}.

\section{Usage Notes}

If you've added \sdag\ code to your class, you must link in the code
by adding ``{\it className}\_SDAG\_CODE'' inside the class declaration
in the .h file. This macro defines the entry points and support code
used by \sdag{}. Forgetting this results in a compile error (undefined
SDAG entry methods referenced from the .def.h file).

For example, an array named ``Foo'' that uses sdag code might contain:
\begin{center}
\begin{alltt}
class Foo : public CBase_Foo \{
public:
    Foo_SDAG_CODE
    Foo(...) \{
       ...
    \}
    Foo(CkMigrateMessage *m) \{ \}
    
    void pup(PUP::er &p) \{
       ...
    \}
    . . .
\};
\end{alltt}
\end{center}

\zap{
\section{Relationship to Threads}

Threads are typically used to perform the abovementioned sequencing.
Lets us code our previous example using threads.

%\begin{figure}[ht]
\begin{center}
\begin{alltt}
void compute_thread(int first_index, int second_index)
\{
    getPatch(first_index);
    getPatch(second_index);
    threadId[0] = createThread(recvFirst);
    threadId[1] = createThread(recvSecond);
    threadJoin(2, threadId);
    computeInteractions(first, second);
  \}
  void recvFirst(void)
  \{
    recv(first, sizeof(Patch), ANY_PE, FIRST_TAG);
    filter(first);
  \}
  void recvSecond(void)
  \{
    recv(second, sizeof(Patch), ANY_PE, SECOND_TAG);
    filter(second);
  \}
\end{alltt}
\end{center}
%\caption{Compute Thread in a Molecular Dynamics Application}
%\label{figthreadexample}
%\end{figure}

Contrast the compute chare-object example in figure~\ref{figchareexample} with
a thread-based implementation of the same scheme in
figure~\ref{figthreadexample}. Functions \uw{getFirst}, and \uw{getSecond} send
messages asynchronously to the PatchManager, requesting that the specified
patches be sent to them, and return immediately. Since these messages with
patches could arrive in any order, two threads, \uw{recvFirst} and
\uw{recvSecond}, are created. These threads block, waiting for messages to
arrive. After each message arrives, each thread performs the filtering
operation. The main thread waits for these two threads to complete, and then
computes the pairwise interactions. Though the programming complexity of
buffering the messages and maintaining the counters has been eliminated in this
implementation, considerable overhead in the form of thread creation, and
synchronization in the form of {\em join} has been added. Let us now code the
same example in \sdag. It reduces the parallel programming complexity without
adding any significant overhead.

%\begin{figure}[ht]
\begin{center}
\begin{alltt}
  array[1D] compute_object \{
    entry void recv_first(Patch *first);
    entry void recv_second(Patch *first);
    entry void compute_object(MSG *msg)\{
      serial \{
         PatchManager->Get(msg->first_index,\dots);
         PatchManager->Get(msg->second_index,\dots);
      \}
      overlap \{
        when recv_first(Patch *first) serial \{ filter(first); \}
        when recv_second(Patch *second) serial \{ filter(second); \}
      \}
      serial \{ computeInteractions(first, second); \}
    \}
  \}
\end{alltt}
\end{center}
%\caption{\sdag\ Implementation of the Compute Object}
%\label{figsdagexample}
%\end{figure}
}


\zap{
\section{Grammar}

\paragraph{Tokens}

\begin{alltt}
  <ident> = Valid \CC{} identifier 
  <int-expr> = Valid \CC{} integer expression 
  <\CC{}-code> = Valid \CC{} code 
\end{alltt}

\paragraph{Grammar in EBNF Form}

\begin{alltt}
<sdag> := <class-decl> <sdagentry>+ 

<class-decl> := "class" <ident> 

<sdagentry> := "sdagentry" <ident> "(" <ident> "*" <ident> ")" <body> 

<body> := <stmt> 
        | "\{" <stmt>+ "\}" 

<stmt> := <overlap-stmt> 
        | <when-stmt> 
        | <atomic-stmt> 
        | <if-stmt> 
        | <while-stmt> 
        | <for-stmt> 
        | <forall-stmt> 

<overlap-stmt> := "overlap" <body> 

<atomic-stmt> := "serial" "\{" <\CC-code> "\}" 

<if-stmt> := "if" "(" <int-expr> ")" <body> [<else-stmt>] 

<else-stmt> := "else" <body> 

<while-stmt> := "while" "(" <int-expr> ")" <body> 

<for-stmt> := "for" "(" <c++-code> ";" <int-expr> ";" <c++-code> ")" <body> 

<forall-stmt> := "forall" "[" <ident> "]" "(" <range-stride> ")" <body> 

<range-stride> := <int-expr> ":" <int-expr> "," <int-expr> 

<when-stmt> := "when" <entry-list>  <body> 

<entry-list> := <entry> 
              | <entry> [ "," <entry-list> ] 

<entry> := <ident> [ "[" <int-expr> "]" ] "(" <ident> "*" <ident> ")" 
  
\end{alltt}
}

\zap{
\sdag\ is a coordination language built on top of \charmpp\ that supports the
sequencing mentioned above, while overcoming limitations of thread-based
languages, and facilitating a clear expression of flow of control within the
object without losing the performance benefits of adaptive message-driven
execution.  In other words, \sdag\ is a structured notation for specifying
intra-process control dependences in message-driven programs. It combines the
efficiency of message-driven execution with the explicitness of control
specification. \sdag\ allows easy expression of dependences among messages and
computations and also among computations within the same object using
when-blocks and various structured constructs.  \sdag\ is adequate for
expressing control-dependencies that form a series-parallel control-flow graph.
\sdag\ has been developed on top of \charmpp\. \sdag\ allows \charmpp\ entry
methods (in chares, groups or arrays) to specify code (a when-block body) to be
executed upon occurrence of certain events.  These events (or guards of a
when-block) are entry methods of the object that can be invoked remotely. While
writing a \sdag\ program, one has to declare these entries in \charmpp\
interface file. The implementation of the entry methods that contain the
when-block is written using the \sdag\ language. Grammar of \sdag\ is given in
the EBNF form below.

\subsubsection{Usage}

You can use SDAG to implement entry methods for any chare, chare array, group,
or nodegroup. Any entry method implemented using SDAG must be implemented in the
interface (.ci) file for its class. An SDAG entry method consists of a series of
SDAG constructs of the following kinds:

\begin{itemize}
    \item {\tt forall} loops: 
    \item {\tt if}, {\tt for}, and {\tt while} statements: these statements have
        the same meaning as the normal {\tt if}, {\tt for}, and {\tt while}
        loops in sequential \CC programs. This allows the programmer to use
        common control flow constructs outside the context of serial blocks.
\end{itemize}

\sdag{} code can be inserted into the .ci file for any array, group, or chare's entry methods.

For more details regarding \sdag{}, look at \examplerefdir{hello/sdag}
}



\section{Related Publications}
\label{publications}

For starters, see the publications, reports, and manuals 
on the Parallel Programming Laboratory website: \texttt{http://charm.cs.uiuc.edu/}. 

\subsection{Associated Tools and Libraries}

Several tools and libraries are provided for \charmpp{}. \projections{} 
is an automatic performance analysis tool which provides
the user with information about the parallel behavior of \charmpp\ programs. 
The purpose of implementing \charmpp{} standard
libraries is to reduce the time needed to develop parallel
applications with the help of a set of efficient and re-usable modules.
Most of the libraries have been described in a separate manual.

\subsection{\projections}

\projections{} is a performance visualization and feedback tool. The system has
a much more refined understanding of user computation than is possible in
traditional tools.

\projections{} displays information about the request for creation and the
actual creation of tasks in \charmpp\ programs. Projections also provides the
function of post-mortem clock synchronization. Additionally, it can also
automatically partition the execution of the running program into logically
separate units, and automatically analyzes each individual partition. 

Future versions will be able to provide recommendations/suggestions for
improving performance as well.

\section{Contacts}
\label{Distribution}

While we can promise neither bug-free software nor immediate solutions   
to all problems, \charmpp\ is a stable system and it is our intention to
keep it as up-to-date and usable as our resources will allow
by responding quickly to questions and bug reports.  To that
end, there are mechanisms in place for contacting Charm users
and developers. 

Our software is made available for research use and evaluation.
For the latest software distribution, further information about
\converse{}/\charmpp\ and information on how to contact the Parallel
Programming laboratory, see our website at \texttt{http://charm.cs.uiuc.edu/}.

If retrieval of a publication via these channels is not possible,
please send electronic mail to \texttt{kale@cs.uiuc.edu} or postal mail to:

\begin{alltt}
   Laxmikant Kale
   Department of Computer Science 
   University of Illinois 
   201 N. Goodwin Ave.
   Urbana, IL 61801 
\end{alltt}



\newpage
\addtocontents{toc}{\contentsline {section}{\numberline {}References}{46}}
\bibliographystyle{plain}
\bibliography{group}

\newpage
\addtocontents{toc}{\contentsline {section}{\numberline {}Index}{47}}
\include{index}

\end{document}
